


 
\documentclass[11pt]{article}
\usepackage[utf8]{inputenc}
\usepackage{mathtools,amsmath,amssymb}
\usepackage{graphicx}
\usepackage{lmodern}
\usepackage[T1]{fontenc}
%\usepackage{microtype}
\usepackage[pagebackref=false]{hyperref}
\usepackage{bbm}
\usepackage{bm}
\usepackage{appendix}
\usepackage{comment}
\numberwithin{equation}{section}
\numberwithin{equation}{subsection}





%\usepackage[textsize=tiny]{todonotes}
 
% \usepackage[notref,notcite]{showkeys}
\usepackage{xspace}
%\usepackage{bibentry}
%\usepackage{easybmat}

\usepackage{graphicx}

\usepackage{subcaption} 

%\usepackage[numbers,sort&compress]{natbib}
%\usepackage{hypernat}
 
\usepackage{tikz}
\usetikzlibrary{decorations.pathreplacing}
% MACROS
% % 
% \newcommand{\cristian}[1]{{\bf (* {\color{ForestGreen}V:{} \small #1}*)}}
% \newcommand{\jorge}[1]{{\bf (* {\color{BlueViolet}G:{} \small #1}*)}}
% \newcommand{\rouven}[1]{{ (* {\color{red}{} \small #1}*)}}
 
\frenchspacing

\newcommand{\changelocaltocdepth}[1]{%
  \addtocontents{toc}{\protect\setcounter{tocdepth}{#1}}%
  \setcounter{tocdepth}{#1}%
}
\newcommand{\id}{I}
\newcommand{\Xt}{\tilde{X}}




\newcommand{\len}{L}
\newcommand{\spe}{N}


\newcommand{\com}[1]{{ (* {\color{red}\small #1}*)}}


\setcounter{tocdepth}{2}

\usepackage[margin=2.5cm]{geometry} 

%%%%%  oscillators
\newcommand{\osc}[1]{\mathbf{#1}}
\newcommand{\oscgreek}[1]{\boldsymbol{#1}}
\newcommand{\dagg}[1]{\bar{#1}}
\newcommand{\ep}{\epsilon}
\newcommand{\ob}{\osc{b}}
\newcommand{\oc}{\osc{c}}
\newcommand{\od}{\osc{d}}
\newcommand{\oab}{\dagg{\osc{a}}}
\newcommand{\obb}{\dagg{\osc{b}}}
\newcommand{\ocb}{\dagg{\osc{c}}}
\newcommand{\odb}{\dagg{\osc{d}}}
\newcommand{\ox}{\oscgreek{\xi}}
\newcommand{\oxb}{\dagg{\oscgreek{\xi}}}
\newcommand{\och}{\oscgreek{\chi}}
\newcommand{\ochb}{\dagg{\oscgreek{\chi}}}
\newcommand{\NN}{\mathbf{N}}
\newcommand{\ccc}{\mathbf{C}}
\newcommand{\qop}{Q}
\newcommand{\sop}{Y}
\newcommand{\q}{\mathbf{\epsilon}}
\newcommand{\ID}{I}
\newcommand{\per}{P}

\newcommand{\EE}{\mathrm{E}}

\newmuskip\pFqmuskip

\newcommand*\pFq[6][8]{%
  \begingroup % only local assignments
  \pFqmuskip=#1mu\relax
  \mathchardef\normalcomma=\mathcode`,
  % make the comma math active
  \mathcode`\,=\string"8000
  % and define it to be \pFqcomma
  \begingroup\lccode`\~=`\,
  \lowercase{\endgroup\let~}\pFqcomma
  % typeset the formula
  {}_{#2}\phi_{#3}{\left[\genfrac..{0pt}{}{#4}{#5};#6\right]}%
  \endgroup
}
\newcommand{\pFqcomma}{{\normalcomma}\mskip\pFqmuskip}


\DeclareMathOperator{\tr}{tr}
\DeclareMathOperator{\Li}{Li}
\DeclareMathOperator{\diag}{diag}


\newcommand{\KL}{\mathcal{K}}
\newcommand{\KR}{\hat{\mathcal{K}}}
\newcommand{\oa}{\mathbf{a}}
\newcommand{\dd}{\mathcal{D}_{\widehat{x}}}
\newcommand{\oad}{\mathbf{a}^{\dagger}}

\newcommand{\phil}{\beta_-}
\newcommand{\phir}{\beta_+}
\newcommand{\ma}{n}

\newcommand{\oaq}{\mathbf{a}_\q}

\newcommand{\oadq}{\mathbf{\bar{a}}_\q}

\newcommand{\N}{\mathbf{N}}
\newcommand{\twoj}{\nu}
\allowdisplaybreaks


\begin{document} 
 
\begingroup
% \centering
\begin{center}
 \begingroup\LARGE
\bf Multispecies stirring process
\par\endgroup
 \vspace{3.5em}
 \begingroup\large \bf
% {Rouven Frassek}
 \par\endgroup
\vspace{2em}

\begingroup\sffamily 
%  
% Università degli Studi di Modena e Reggio Emilia
%University of Modena and Reggio Emilia, 
%\\Department of Physics, Informatics and Mathematics,\\
%Via G. Campi 213/b, 41125 Modena, Italy\\ 
\par\endgroup
\vspace{2em}

 Version: \today
\end{center}
%  

\thispagestyle{empty}

\begin{abstract}
\noindent
...
\end{abstract}

%\vfill 

\tableofcontents

% \newpage

\section{Introduction} 
In the past years duality has been developed to study stochastic processes, especially in the boundary driven case, i.e. when the system is put in contact with boundaries that generate a non-equilibrium current. The literature is huge. We can address , for instance, to \cite{giardina2009duality,carinci2013duality,frassek2020duality}. One application of duality is the attempt to characterize the non equilibrium steady state of boundary driven process via the so called absorption probabilities. Indeed, there are many cases where the dual process has absorbing boundaries. This means that the the chain is eventually empty in the long time scale. Moreover, there are cases in which the steady state of a stochastic process can be computed explicitly, thanks to the integrability property of the system (quantum inverse scattering method, matrix product ansatz). In this direction, some examples can be found in \cite{derrida1993exact,frassek2019non,frassek2020eigenstates}. In the past, the focus has been on processes where one can distinguish, in each site of a certain graph, a (or many) particles or an empty state. However, this particles are indistinguishable, in the sense that you can pile up in a site just one type of particles. Some natural questions arise: what happens if we imagine to assign different colours to the particles? Is the duality property still holding? How far can we push our analysis for the characterization of the non equilibrium steady state?\\
The way one can define a multi-type (or multi-species) interacting particle system is not unique. In the literature there are many. Two of them are the following. On one hand, the \textit{multi-layer models}, in which the process is constructed on a many layer graph where each graph correspond to a type and perform a kind of dynamic (independent jump, exclusion/inclusion walkers) and the layers communicate each other with some "switching probabilities". The state space of the Markov process is the Cartesian product of the state spaces of each layer, i.e. of the two species. For example see \cite{floreani2022switching,redig2022ergodic}. On the other hand, the \textit{single layer models}. Here, there is just one layer, where particles of each type can pile up, in a way that the exclusion dynamic concerns many species together. In other words, now the species can interact "directly", not just by switching layer. Here the state space contains all the species and it is not product anymore. Some examples of such a dynamic can be found in \cite{casini2022uphill,vanicat2017exact,zhou2021orthogonal}. \\
In this work we concentrate our effort in this last "single layer" case with a repulsive process, i.e. a dynamic in which a maximum amount of particles can be present in each site at each time. We will call this amount, the "maximal occupancy" and it will indicated with $\twoj$ where $\twoj\in\mathbb{N}$. Particle of any species can exchange their position in the graph, under this exclusion constraint. The system will be put out of equilibrium and some absorbing duality result are investigated. Moreover, we will use this duality, to study the out of equilibrium steady state. In particular, in case of $\twoj=1$ we retrieve the model studied in \cite{vanicat2017exact}. We will then exploit this last result and the duality, to write explicitly all the correlations of this \textit{hard-core exclusion} case. The main tool that we will use is the Lie algebraic description of the process, that allows to find the duality as a proper symmetry of the generator. 
\paragraph{Organization of the paper}
\section{The stirring process}
\subsection{Informal description}
The process studied in this paper is the {\em multi-species stirring process}. 
In words, the dynamics is described as follows. Each site
of a connected graph can host up to $\nu\in \mathbb{N}$ particles.
The particles have a {\em type} (sometimes called {\em species} or {\em color})
which can take $N$ values.
Then, the dynamics has two parts: on each edge of the graph, 
any two types are swapped at rate $1$; additionally, on each site $x$, 
a particles of a give type (if present) is replaced with a
particle of type $k$ at rate $\alpha_k^x >0$.
The swap dynamics taking place on the edges is of Kawasaki-type 
with $N$ conservation laws
(the total number of particles of each type). 
The site-dynamics is instead of Glauber-type 
and, in the long-time limit, so-called non-equilibrium
steady state sets in.


\vspace{.5cm}
\noindent
{\bf Remark:} in the following we will often consider type $1$ as a distinguished type
(called `holes'), whose number at each site is fixed by the numbers of particles 
of the remaining $N-1$ types (at the same site). It is interesting to observe that, in case we stop distinguishing the colors of the ``true'' particles of types $\{2,\ldots,N\}$, we retrieve the 
boundary-driven version of the partial symmetric exclusion process \cite{schutzSandow,carinci2013duality}.
The choice of type $1$ for the distinguished type is arbitrary and does not affect
the results. In particular, we shall see in Section \ref{sectionDuality} about duality, that
the dual process eventually has only particles of type $1$ on the graph.
Interpreting type $1$ as an hole, one can then say that the dual process voids the graph.  









\subsection{The infinitesimal generator}
We now give the mathematical description of the multi-species stirring process.
We consider a connected graph $G=(V,\mathcal{E})$ with vertex set $V$ and edge set $\mathcal{E}$.
At each site $x\in V$, we describe the occupation with an $N-$dimensional vector $n^{x}=(n_{1}^{x},\ldots,n_{N}^{x})$ in which the value of the $k$-th component $n_{k}^{x}$ denotes the number of particles of species $k\in \{1,\ldots,N\}$ at site $x\in V$. On each vertex we allow a total  maximal occupancy $\nu\in \mathbb{N}$. Then, the configuration space of the process on the graph $G$ is 
\begin{equation}\label{stateSpace}
    \Omega:=\bigtimes_{x\in V} \Omega_{x}
\end{equation}
where
\begin{equation}
\Omega_{x}:=\left\{(n_{1},\ldots,n_{N})\in\mathbb{N}_0^{N}\;:\; \sum_{k=1}^{N}n_{k}=\twoj\right\}
\end{equation}
We denote a particle configuration as $\mathbf{n}\in \Omega$, where $\mathbf{n}=(n_{k}^{x})_{x\in V,\,k\in\{1,\ldots,N\}}$ .\\
The dynamics consists in an exchange of particles (at Poissonian times) between connected vertices. Moreover, each vertex  exchanges particles with the external environement. The infinitesimal generator of the process reads
\begin{equation}\label{Generator}
    \mathcal{L}=\sum_{(x,y)\in \mathcal{E}}\omega_{x,y}\mathcal{L}_{x,y}+\sum_{x\in V}\Gamma_{x}\mathcal{L}_{x}
\end{equation}
where the quantities $ \omega_{x,y}\geq 0$ are called conductances and $\Gamma_{x}\geq 0$ local inhomogeneities. The generator $\mathcal{L}_{x,y}$ is called the \textit{edge generator}, while $\mathcal{L}_{x}$ is called the \textit{site generator}. These linear operators act on functions $f:\Omega\to \mathbb{R}$ as follows
\begin{equation}\label{edgeGenerator}
\mathcal{L}_{x,y}f(\bm{n})=\sum_{k,\ell=1}^{N}n_{k}^{x}n_{l}^{y}\left[f(\bm{n}-\delta_{x}^{k}+\delta_{\ell}^{x}+\delta_{k}^{y}-\delta_{\ell}^{y})-f(\bm{n})\right]
\end{equation}
\begin{equation}\label{siteGenerator}
    \mathcal{L}_{x}f(\bm{n})=\sum_{k,\ell=1}^{N}\alpha_{k}^{x}n_{\ell}^{x}\left[f(\bm{n}+\delta_{k}^{x}-\delta_{\ell}^{x})-f(\bm{n})\right]
\end{equation}
where 
\begin{equation}
(\delta_{k}^{x})^{y}_{\ell}=\begin{cases}
1\qquad &\text{if}\quad y=x,\;\ell=k\\
0\qquad &\text{otherwise}
\end{cases}
\end{equation}
This means that on the edge $(x,y)\in \mathcal{E}$ at rate $\omega_{x,y}n_{k}^{x}n_{\ell}^{y}$ a particle of type $k$ at site $x$ is exchanged with a particle of tipe $l$ at site $y$; furthermore, on each site $x\in V$ at rate $\Gamma_{x}\alpha_{k}^{x}n_{\ell}^{x}$ a particle of type $\ell$ is replaced with a particle of type $k$. 
See Figure \ref{fig:1}, Figure \ref{fig:2}for a visual description of the dyamics. 
\begin{figure}
    \centering
    \includegraphics[scale=0.6]{Dyn_stir.eps}
    \caption{The edge dynamics}
    \label{fig:1}
\end{figure}
\begin{figure}
    \centering
    \includegraphics[scale=0.88]{Dyn_stir_bordo.eps}
    \caption{The site dynamics}
    \label{fig:2}
\end{figure}
\subsection{Reversible measures}
The process described by the generator \eqref{Generator} is reversible with respect to the homogeneus product measure \begin{equation}\label{reversibleMeasure}
\mu_{rev}=\bigtimes_{x\in V}\mu_{rev}^{x}
\end{equation}
when for any $k\in\{1,\ldots,N\}$
\begin{equation}\label{reversibilityCondition}
\alpha_{k}^{x}=\alpha_{k}\qquad \forall x\in V
\end{equation}
This measure has marginals $\mu_{rev}^{x}$ distributed as 
\begin{equation}
 \mu^{x}_{rev}\sim \text{Multinomial}\left(\twoj,\rho_{1},\ldots,\rho_{N}\right)\quad \text{where}\quad \rho_{k}=\frac{\alpha_{k}}{\sum_{i=1}^{N}\alpha_{i}}
\end{equation}
namely
\begin{equation}
\mu_{rev}^{x}(n^{x})=\frac{\nu!}{\prod_{k=1}^{N}n_{k}^{x}!}\prod_{k=1}^{N}\rho_{k}^{n_{k}^{x}}
\end{equation}
This can be proved just by imposing the detailed balance conditions for the edge and for the site generators. If condition \eqref{reversibilityCondition} is not met, then reversibility is lost: each reservoir at site $x$ wants to fix its own density $\alpha^{x}=(\alpha_{1}^{x},\ldots,\alpha_{N}^{x})$ and currents will arise when at least two of them are different. 
\subsection{The Lie algebraic description}


Consider the Lie algebra $gl(N)$ with generators denoted by $\EE_{ab}$ with $a,b\in \{1,\ldots,N\}$ and commutation relations
\begin{equation}\label{eq:comgl}
\left[\EE_{ab},\EE_{cd}\right]=\EE_{ad}\delta_{bc}-\EE_{cb}\delta_{ad}\qquad \forall a,b\in \{1,\ldots,N\}
\end{equation}
The finite-dimensional representations are labelled by partitions $\lambda=(\lambda_1,\lambda_2,\ldots,\lambda_N)$ of $\nu$ with $\lambda_i\in \mathbb{N}$ and $\sum_{i=1}^N \lambda_i = \nu$. We assume $\lambda_i\geq \lambda_{i+1}$ without loss of generality.
We are interested in the {\em symmetric} finite-dimensional representations with 
\begin{equation}\label{eq:dynkin}
    \lambda=(\twoj,0,\ldots,0) \qquad\text{where}\qquad \twoj\in\mathbb{N}
\end{equation} 
The dimension $M$ of the symmetric representations is given by the combination of $N$ objects in $\twoj$ positions with repetition, namely
\begin{equation}
	M= \frac{(N+\twoj-1)!}{\twoj  !(N-1)!}
\end{equation} 
The basis elements of the vector space $\mathbb{C}^{M}$ are the column vectors 
\begin{equation}
  |n\rangle=  |n_{1},\ldots,n_{N}\rangle,\quad \text{with}\quad n_{i}\in\mathbb{N}_{0}\quad \text{sucht that}\quad \sum_{i=1}^{N}n_{i}=\nu
\end{equation}
%where $\sum_{k=1}^{N}n_{i}=\nu$. 
This basis vector satisfy the orthogonality relation with respect to the Euclidean scalar product 
\begin{equation}\label{ortho}
    \langle m_{1},\ldots,m_{N}|n_{1},\ldots,n_{N}\rangle=\prod_{k=1}^{N}\delta_{m_{k},n_{k}}
\end{equation}
where  $ \langle m_{1},\ldots,m_{N}|$ is the row vector obtained by transposing $|m_{1},\ldots,m_{N}\rangle$ and $\delta_{m_{k},n_{k}}$ is the Kronhecker delta. 
The explicit actions of the algebra generators on the vectors $|n\rangle$ are the following:
\begin{equation}\label{actionE}
	\begin{cases}
		E_{ab}|n_{1},\ldots,n_{a},\ldots,n_{b},\ldots,n_{N}\rangle =n_{b}|n_{1},\ldots,n_{a}+1,\ldots,n_{b}-1,\ldots,n_{N}\rangle\quad a\neq b\\[0.1cm]
		E_{aa}|n_{1},\ldots,n_{a},\ldots,n_{b},\ldots,n_{N}\rangle = n_{a} |n_{1},\ldots,n_{a},\ldots,n_{b},\ldots,n_{N}\rangle\quad a=b
	\end{cases}
\end{equation}  
One can check that the matrices defined in this way satisfy the commutation relations \eqref{eq:comgl} and yield Dynkin weight \eqref{eq:dynkin}.

\noindent
\textbf{Remark}: The symmetric representations $\lambda=(\twoj,0,\ldots,0)$ are dual to the representations $\lambda=(\twoj,\ldots,\twoj,0)$ that can be obtained via 
\begin{equation}
   \bar E_{ab}=\nu\delta_{ab}-E_{N-b+1,N-a+1}
\end{equation}
{\color{red} This needs to be checked and give alternative form of Hamiltonian?}\\
By the Lie algebra above we describe the process with generator \eqref{Generator}. The state space \eqref{stateSpace} is the vector space with basis elements 

%We now define the equivalent of \eqref{stateSpace} in the vector notation
% \begin{equation}
%	\Omega':=\left\{|n\rangle=|n_{1},\ldots,n_{N}\rangle \;:\;n\in\mathbb{N}_0^N,\;\;|n|=\twoj\right\}^{\otimes|V|}
%	\end{equation}
%where we denote 
\begin{equation}
|{\bf n}\rangle=\left(\,\bigotimes_{x\in V}	|n_{1}^{x},\ldots,n_{N}^{x}\rangle\right)
\end{equation}
where we require that $n_{k}^{x}\in \mathbb{N}_{0}$ and $\forall x\in V$ we have $\sum_{k=1}^{N}n_{k}^{x}=\nu$. Sometimes it will bw convenient to write $|n^{x}\rangle$ to denote, for a fixed $x\in V$, the vector $|n_{1}^{x},\ldots,n_{N}^{x}\rangle$. The following orthogonality relation is a consequence of the single site relation \eqref{ortho}
\begin{equation}
    \langle {\bf n}|{\bf m}\rangle =\prod_{x\in V}\prod_{i=1}^N\delta_{n^x_i,m^{x}_i}
\end{equation}
We introduce the Hamiltonian operator

\begin{equation}\label{OriginalHamiltonian}
	\begin{split}
		H=\sum_{x,y\in \mathcal{E}}\omega_{x,y}H_{x,y}+\sum_{x\in V}\Gamma_{x}H_{x}
	\end{split}
\end{equation}
where the edge Hamiltonian is
\begin{equation}\label{edgeHamiltonian}
H_{x,y}=\sum_{k,\ell=1}^{N}\Big(E_{k\ell}^{x} E_{\ell k}^{y}-E_{\ell\ell}^{x} E_{kk}^{y}\Big)
 \end{equation}
  and where the site Hamiltonian is
 \begin{equation}\label{siteHamiltonian}
H_{x}=\sum_{k,\ell=1}^{N}\alpha_{k}^{x}\left(E_{k\ell}^{x}-E_{\ell\ell}^{x}\right)
\end{equation}
Here $E_{kl}^{x}$ is a copy of $E_{kl}$ defined in \eqref{actionE} acting on site $x$. 
Following  \cite{belitsky2015self}, the Hamiltonian and the genertor are linked by
\begin{equation}\label{Hamiltonian-Generator}
H=\mathcal{L}^{T}
\end{equation}
The action of the generator can also be expressed as 
\begin{equation}
    \mathcal{L}f( {\bf n})=\langle f|H| {\bf n}\rangle
\end{equation}
where 
\begin{equation}
    \langle f|=\sum_{ {{\bf n}\in \Omega}}f( {\bf n})\langle  {\bf n}|
\end{equation}

We can write the edge Hamiltonian \eqref{edgeHamiltonian} as a function of the coproduct of the second Casimir of $gl(N)$
\begin{equation}\label{secondCasimir}
    C_{2}=\sum_{a,b=1}^{N}E_{ab}E_{ba}
\end{equation}
that acts diagonally as $C_{2}|\mathbf{n}\rangle=\twoj(\twoj+N)|\mathbf{n}\rangle$ and belogs to the center of $gl(N)$ (i.e. it commutes with all the algebra elements).  
More precisely,  considering the standard coproduct 
\begin{equation}
\begin{split}
\Delta:gl(N)&\to gl(N)\otimes gl(N)\\
E_{ab}&\to E_{ab}\otimes \mathbbm{1}+\mathbbm{1}\otimes E_{ab}
\end{split}
\end{equation}
we have 
\begin{equation}
\Delta(C_{2})=\sum_{a,b=1}^{N}\Delta(E_{ab})\Delta(E_{ba})=2\sum_{a,b=1}^{N}E_{ab}\otimes E_{ba}+C_{2}\otimes \mathbbm{1}+\mathbbm{1}\otimes C_{2}
\end{equation}
Then, one can check that 
\begin{equation}\label{hamiltonianCasimir}
	H_{x,y}=\frac{1}{2}\Delta^{x,y}(C_{2})-\twoj(2\twoj+N)
\end{equation}
where $\Delta^{x,y}(C_2)$ denotes a copy of $\Delta(C_2)$ acting on  
the edge $(x,y)\in \mathcal{E}$.


\begin{comment}
We introduce the Hamiltonian operator

\begin{equation}\label{OriginalHamiltonian}
	\begin{split}
		H=\sum_{x,y\in \mathcal{E}}\omega_{x,y}H_{x,y}+\sum_{x\in V}\Gamma_{x}H_{x}
	\end{split}
\end{equation}
where the edge Hamiltonian is
\begin{equation}\label{edgeHamiltonian}
H_{x,y}=\sum_{k,l=1}^{N}E_{kl}^{x}\otimes E_{lk}^{y}-E_{ll}^{x}\otimes E_{kk}^{y}
 \end{equation}
 and where the site Hamiltonian is
 \begin{equation}\label{siteHamiltonian}
H_{x}=\sum_{k,l=1}^{N}\alpha_{k}^{x}\left(E_{kl}^{x}-E_{ll}^{x}\right)
\end{equation}
Here $E_{kl}^{x}$ is a copy of $E_{kl}$ defined in \eqref{actionE} acting on site $x$. 

We can write this Hamiltonian in function of the coproduct of the second Casimir
\begin{equation}
    C_{2}=\sum_{a,b=1}^{N}E_{ab}E_{ba}
\end{equation}
It reads
\begin{equation}
	H_{x,y}=\left\{\frac{1}{2}\Delta^{xy}(C_{2})-\twoj(2\twoj+N)\frac{1}{2}\mathbbm{1}^{x}\otimes\mathbbm{1}^{y}\right\}
\end{equation}
Here we introduced the standard coproduct
\begin{equation}
\begin{split}
\Delta:gl(N)&\to gl(N)\otimes gl(N)\\
E_{ab}&\to E_{ab}\otimes \mathbbm{1}+\mathbbm{1}\otimes E_{ab}
\end{split}
\end{equation} acting on sites $x,y$ and used that $C_{2}|n\rangle=\twoj(\twoj+N)|n\rangle$. 
\end{comment}
 



\section{Duality}\label{sectionDuality}
\subsection{Definition}
Consider two Markov processes $(\eta_{t})_{t\geq 0}$ defined on a state space $\Omega$ and $(\xi_{t})_{t\geq 0}$ defined on a state space $\widetilde{\Omega}$. We say that they are dual, with respect to a duality function $D:\Omega\times \widetilde{\Omega}\to \mathbb{R}$, if $\forall \eta\in\Omega$, $\forall \xi\in\widetilde{\Omega}$ and $\forall t> 0$ we have 
\begin{equation}
    \mathbb{E}_{\eta}\left[D(\eta_{t},\xi)\right]=\mathbb{E}_{\xi}\left[D(\eta,\xi_{t})\right]
\end{equation}
where $\mathbb{E}_{\eta}$ denotes the expectation with respect the law of the Markov process $(\eta_{t})_{t\geq 0}$ initialized with the particle configuration $\eta$, whereas $\mathbb{E}_{\xi}$ denotes the expectation with respect to the law of the Markov process $(\xi_{t})_{t\geq 0}$ initialized with the particle configuration $\xi$.
The duality definition can also be formulated as a relation between the generators. Call $\mathcal{L}$ the generator of $(\eta_{t})_{t\geq0}$ and $\widetilde{\mathcal{L}}$ the generator of $(\xi_{t})_{t\geq 0}$, then we say that these two processes are dual with respect to the duality function $D:\Omega\times \widetilde{\Omega}\to \mathbb{R}$ if $\forall \eta\in\Omega$ and $\forall \xi\in\widetilde{\Omega}$
\begin{equation}\label{dualityRelationGenerator}
    \left(\mathcal{L}D(\cdot,\xi)\right)(\eta)=\left(\widetilde{\mathcal{L}}D(\eta,\cdot)\right)(\xi)
\end{equation}
In the specific case where $\mathcal{L}=\widetilde{\mathcal{L}}$ we say that the process is self-dual.
\newline
\newline
\textbf{Remark}:
when the state spaces of the dual processes is finite, the generators and the duality function are matrices with elements $\mathcal{L}(\eta,\eta^{'})$, $\widetilde{\mathcal{L}}(\xi,\xi^{'})$ and $D(\eta,\xi)$ for arbitrary $\eta,\eta^{'}\in\Omega$ and $\xi,\xi^{'}\in \widetilde{\Omega}$. Therefore, we can write the duality relation \eqref{dualityRelationGenerator} as 
\begin{equation}
    \sum_{\eta^{'}\in\,\Omega}\mathcal{L}(\eta,\eta^{'})D(\eta^{'},\xi)=\sum_{\xi^{'}\in\, \widetilde{\Omega}}\widetilde{\mathcal{L}}(\xi,\xi^{'})D(\eta,\xi^{'})
\end{equation}
that can be read as
\begin{equation}\label{dualityIntertwines}
    \mathcal{L}D=D\widetilde{\mathcal{L}}^{\,T}
\end{equation}
where the superscript $T$ denotes the matrix transposition. Therefore, the duality relation \eqref{dualityIntertwines} is an intertwining between two linear operators $\mathcal{L}$ and $\widetilde{\mathcal{L}}$. Working with the Hamiltonian operators the duality relation \eqref{dualityIntertwines} reads 
\begin{equation}\label{DualityRelation}
    H^{T}D=D\widetilde{H}
\end{equation}
\subsection{Duality for the multi-species stirring process}\label{statementDualitySubsection}
In this section we show that the process $(\mathbf{n}(t))_{t\geq 0}$, defined by the generator \eqref{Generator}, is dual to a process $(\bm{\xi}(t))_{t\geq 0}$ taking values in the enlarged state space
\begin{equation}\label{dualStateSpace}
    \widetilde{\Omega}= \bigtimes_{x\in V} \widetilde{\Omega}_{x}\ = \bigtimes_{x\in V} (\Omega_{x}\times \mathbb{N}_{0}^{N-1})
\end{equation}
To each site $x\in V$ we associate an ``extra site'', denoted $\widehat{x}$,
where dual particles will accumulate in the course of time. 
We write the configurations $\bm{\xi} \in \widetilde\Omega$  as
\begin{equation}
    \bm{\xi}=\left(r_{1}^{x},\ldots,r_{N}^{x},m_{2}^{\widehat{x}},\ldots,m_{N}^{\widehat{x}}\right)_{x\in V}
\end{equation}
where the component $r_{k}^{x}$ is interpreted as the number of dual particles of type $k\in \{1,\ldots,N\}$ at site $x$, 
and the component $m_{k}^{\widehat{x}}$  gives the number of dual particles of type $k\in \{2,\ldots,N\}$ at 
the extra-site $\widehat{x}$ connected to $x\in V$.
 The  generator of the dual process is 
 \begin{equation}\label{DualGenerator}
    \widetilde{\mathcal{L}}=\sum_{(x,y)\in \mathcal{E}}\omega_{x,y}\mathcal{L}_{x,y}+\sum_{x\in V}\Gamma_{x}\widetilde{\mathcal{L}}_{x}
\end{equation}
where 
$\mathcal{L}_{x,y}$ is defined in \eqref{siteGenerator} and, for any function $f:\widetilde{\Omega}\to \mathbb{R}$ 
\begin{equation}\label{siteDualGenerator}
    \widetilde{\mathcal{L}}_{x}f(\bm{\xi})=\sum_{i=1}^{N}\alpha_{i}^{x}\sum_{k=2}^{N}r_{k}^{x}\left(f(\bm{\xi}-\delta_{k}^{x}+\delta_{1}^{x}+\delta_{k}^{\widehat{x}})-f(\bm{\xi})\right)
\end{equation}
and the duality function is 
\begin{equation}\label{dualityElements}
	D(\bm{n},\bm{\xi})=\prod_{x\in V}\left(\frac{(2j-\sum_{k=2}^{N}r_{k}^{x})!}{\nu!}\prod_{k=2}^{N}\frac{n_{k}^{x}!}{(n_{k}^{x}-r_{k}^{x})!}\left(\rho_{k}^{x}\right)^{m_{k}^{\widehat{x}}}\,\right)
\end{equation}
where we denote the \textit{density} of the species $k\in \{2,\ldots,N\}$ at the reservoir connected with the site $x\in V$  by 
\begin{equation}
\rho_{k}^{x}:=\frac{\alpha_{k}^{x}}{\sum_{i=1}^{N}\alpha_{i}^{x}}
\end{equation}
\newline
On one hand, the edge part of the dual generator \eqref{DualGenerator} is a copy of \eqref{edgeGenerator}, therefore it performs the stirring dynamics on the graph. On the other hand, the dual generator \eqref{siteDualGenerator} replaces a particle of any type $k\in\{2,\ldots,N\}$ at site $x$ with a particle of type $1$ and creates a particle of the same type $k$ at the extra-site $\widehat{x}$. This last transition is performed with rate $\sum_{i=1}^{N}\alpha_{i}^{x}r_{k}^{x}$. This means that eventually the dual process voids the graph, putting all the dual particles of species $\{2,\ldots,N\}$ in the extra-sites. In other words the extra-sites play the role of absorbing boundaries. 
\newline \newline
\textbf{Remark}: in the reversible situation, i.e. when $\forall x\in V$ $\alpha_{k}^{x}=\alpha_{k}$, the expectation  of the duality function  $D(\bm{n},\bm{\xi})$ with   $\bm{n}$ distributed as  $\mu_{rev} = \bigtimes_{x\in V}\text{Multinomial}\left(\twoj, p_{1},\ldots,p_{N}\right)$ is
\begin{equation}
\mathbb{E}_{\mu^{rev}}\left[D(\bm{n},\bm{\xi})\right]=\prod_{k=2}^{N}\left(\rho_{k}^{x}\right)^{\sum_{x\in V}r_{k}^{x}+\sum_{x\in V}m_{k}^{\widehat{x}}}\qquad \forall \bm{n}\in \Omega,\quad\forall \bm{\xi}\in \widetilde{\Omega}
\end{equation}

\subsection{Proof of duality}
To prove duality between $(\bm{n}(t))_{t\geq 0}$ and $(\bm{\xi}(t))_{t\geq 0}$ we  show that \eqref{dualityIntertwines} is fulfilled.  To show this, we will use the Hamiltonians (linked with the generators by \eqref{Hamiltonian-Generator}) and their Lie algebraic description. Indeed, in this formalism the proof reduces in finding symmetries and exponential transformation of generators of the Lie algebra.\\
The configuration space  \eqref{dualStateSpace} of the dual process is the collection of vectors
\begin{equation}
    |\bm{\xi}\rangle=\bigotimes_{x\in V}\left(|r_{1}^{x},\ldots,r_{N}^{x}\rangle\otimes |m_{2}^{\widehat{x}},\ldots,m_{N}^{\widehat{x}}\rangle\right)
\end{equation}
The Hamiltonian of the dual process reads as
\begin{equation}\label{DualHamiltonian}
    \widetilde{H}=\sum_{x,y\in \mathcal{E}}\omega_{x,y}H_{x,y}+\sum_{x\in V}\Gamma_{x}\widetilde{H}_{x}
\end{equation}
where $H_{x,y}$ is the one defined in \eqref{edgeHamiltonian}, while 
\begin{equation}\label{siteDualHamiltonian}
    \widetilde{H}_{x}=\sum_{i=1}^{N}\alpha_{i}^{x}\sum_{k=2}^{N}\left((a^{\dagger})_{k}^{\widehat{x}}\,E_{1k}^{x}-E_{kk}^{x}\right)
\end{equation}
Here we introduced bosonic creation operator $a^{\dagger}$ acting as $a^{\dagger}|q\rangle=|q+1\rangle$ on a generic vector $|q\rangle$ with $q\in \mathbb{N}_{0}$, so that in \eqref{siteDualHamiltonian} 
$(a^{\dagger})_{k}^{\widehat{x}}$ denotes a copy of $a^{\dagger}$ acting on the extra-site $\widehat{x}$ and on the species $k\in\{2,\ldots,N\}$. \\
The Hamiltonians \eqref{OriginalHamiltonian} and \eqref{DualHamiltonian} are dual in the sense of \eqref{DualityRelation} with respect to the duality matrix $D$ defined as 
\begin{equation}\label{dualityMatrix}
    D=\prod_{x\in V}d_{x}\otimes \dd
\end{equation}
where
\begin{equation}
d_{x}=R_{x}\exp{(E^{x})}
\end{equation}
with 
\begin{equation}\label{EquationEx}
E^{x}=\sum_{a=2}^{N}E_{a1}^{x}
\end{equation}
and
\begin{equation}\label{Rmatrix}
    R_{x}=\sum_{n^{x}\in\Omega_{x}}\frac{\prod_{k=1}^{N}n_{k}^{x}}{\nu!}|n_{1}^{x},\ldots,n_{N}^{x}\rangle\langle n_{1}^{x},\ldots,n_{N}^{x}|
\end{equation}
and where 
\begin{equation}\label{dualityMatrix2}
\dd=\sum_{m_{2}^{\widehat{x}},\ldots,m_{N}^{\widehat{x}}=0}^{\infty}\prod_{k=2}^{N}\left(\rho_{k}^{x}\right)^{m_{k}^{\widehat{x}}}\langle m_{2}^{\widehat{x}},\ldots,m_{N}^{\widehat{x}}|
\end{equation}
The matrix $R_{x}$ is diagonal. Its elements are related to the inverse of the weights of the reversible measure \eqref{reversibleMeasure}. In particular, to obtain these elements, we have considered the weights of \eqref{reversibleMeasure} when all the parameters $p_{i}=\frac{1}{N}$. Then, the constant $\left(\frac{1}{N}\right)^{\nu}$ has been neglected, since it does not change the duality relation. This $R_{x}$ is called the ``cheap'' duality matrix (see \cite{giardina2009duality}). \\
Since \eqref{dualityMatrix} is product over sites, \eqref{DualityRelation} is equivalent to proving that 
\begin{equation}\label{edgeDualRealtion}
    H_{x,y}^{T}D=DH_{x,y}\qquad \forall (x,y)\in \mathcal{E}
\end{equation}
and 
\begin{equation}\label{siteDualRelation}
    H_{x}^{T}D=D\widetilde{H}_{x}\qquad \forall x\in V.
\end{equation}
We perform the proof in three steps: first we will show that matrix \eqref{dualityMatrix} has elements \eqref{dualityElements}; second we will prove \eqref{edgeDualRealtion}; finally we will show\eqref{siteDualRelation}. 
\paragraph{Elements of the duality matrix.}We aim to show that 
\begin{equation}\label{proofDualityElements}
\langle \bm{n}|D|\bm{\xi}\rangle=D(\bm{n},\bm{\xi})\qquad   \forall \bm{n}\in \Omega,\quad \bm{\xi}\in \widetilde{\Omega}
\end{equation}
with $D(\bm{n},\bm{\xi})$ defined in \eqref{dualityElements}. 
Fix an arbitrary site $x\in V$
\begin{align*}
	 &\langle n^{x}|d_{x}\otimes \dd|\xi^{x}\rangle\\&=\langle n_{1}^{x},\ldots,n_{N}^{x}| (\exp{(E_{12}^{x}+\ldots+E_{1N}^{x}}))^{T}R_{x}\otimes\sum_{m_{2}^{\widehat{x}},\ldots,m_{N}^{\widehat{x}}=0}^{\infty}\prod_{k=2}^{N}\left(\rho_{k}^{x}\right)^{m_{k}^{\widehat{x}}}\langle m_{2}^{\widehat{x}},\ldots,m_{N}^{\widehat{x}}|
	 \\&|r_{1}^{x},\ldots,r_{N}^{x}\rangle_{x} \otimes |q_{2}^{\widehat{x}},\ldots,q_{N}^{\widehat{x}}\rangle
\end{align*}
On one hand, on the extra-site $\widehat{x}$ we have 
\begin{align*}
\sum_{m_{2}^{\widehat{x}},\ldots,m_{N}^{\widehat{x}}=0}^{\infty}\prod_{k=2}^{N}\left(\rho_{k}^{x}\right)^{m_{k}^{\widehat{x}}}\langle m_{2}^{\widehat{x}},\ldots,m_{N}^{\widehat{x}}||q_{2}^{\widehat{x}},\ldots,q_{N}^{\widehat{x}}\rangle=\prod_{k=2}^{N}\left(\rho_{k}^{x}\right)^{m_{k}^{\widehat{x}}}
\end{align*}
where we used the orthogonality relation \eqref{ortho}. 
On the other hand, on the site $x$, we have 
\begin{align*}
&\langle n_{1}^{x},\ldots,n_{N}^{x}|(\exp{(E_{12}^{x}+\ldots+E_{1N}^{x})})^{T}R_{x}|r_{1}^{x},\ldots,r_{N}^{x}\rangle\\&= \langle  n_{1}^{x},\ldots,n_{N}^{x}|\left(\sum_{k_{2}=0}^{\infty}\frac{\left\{\left(E_{12}^{x}\right)^{T}\right\}^{k_{2}}}{k_{2}!}\ldots\sum_{k_{N}=0}^{\infty}\frac{\left\{\left(E_{1N}^{x}\right)^{T}\right\}^{k_{N}}}{k_{N}!}\sum_{s\in\chi}\frac{s_{1}^{x}!\ldots s_{N}^{x}!}{\nu!}|s_{1}^{x},\ldots,s_{N}^{x}\rangle\langle s_{1}^{x},\ldots,s_{N}^{x}|\right)|r_{1}^{x},\ldots,r_{N}^{x}\rangle\\&=
\sum_{k_{2}=0}^{n_{1}^{x}}\ldots\sum_{k_{N}=0}^{n_{N}^{x}}\langle n_{1}^{x}+k_{2}+\ldots+k_{N},n_{2}^{x}-k_{2},\ldots,n_{N}^{x}-k_{N}|_{x}\frac{n_{2}^{x}!\ldots n_{N}^{x}!}{(n_{2}^{x}-k_{2})!\ldots(n_{N}^{x}-k_{N})!}
\\&\cdot 
\frac{1}{k_{2}!,\ldots,k_{N}!}\frac{r_{1}^{x}!\ldots r_{N}^{x}!}{\nu!}|r_{1},\ldots,r_{N}\rangle_{x}\\&=
\frac{(2j-\sum_{k=2}^{N}r_{k}^{x})}{\nu!}\prod_{k=2}^{N}\frac{n_{k}^{x}!}{(n_{k}^{x}-r_{k}^{x})!}
\end{align*}
where in the last equality we applied the orthogonality relations \eqref{ortho} and the fact that $r_{1}^{x}=\nu-\sum_{k=2}^{N}r_{k}^{x}$. Finally, by taking the product over $x\in V$ \eqref{proofDualityElements} is proved.
\begin{flushright}
    $\square$
\end{flushright}
\paragraph{Proof of \eqref{edgeDualRealtion}.}To show \eqref{edgeDualRealtion} we need two 'ingredients'. First the existence of a similarity transformation between the Hamiltonian $H_{x,y}$ defined in \eqref{edgeHamiltonian} and its transposed. As we will show, this similarity transformation is $R_{x}$ defined in \eqref{Rmatrix}. Second, the possibility of finding a symmetry for the Hamiltonian \eqref{edgeHamiltonian}. Exploiting the fact that $H_{x,y}$ is a linear function of the coproduct of the second Casimir of $gl(N)$, (see \eqref{hamiltonianCasimir}) this symmetry is $\exp{(E^{x})}\exp{(E^{y})}$ with $E^{x}$ defined in \eqref{EquationEx}. \\
We show that 
\begin{equation}\label{transpositionPropertyR}
(E_{ab}^{x})^{T}=R_{x}E_{ab}^{x}R_{x}^{-1}\qquad \forall x\in V
\end{equation}
Indeed
\begin{align*}
R_{x}E_{ab}^{x}R_{x}^{-1}=&\sum_{r^{x}\in\Omega_{x}}\left(\frac{r_{1}^{x}!\ldots r_{N}!}{\nu!}|r_{1}^{x},\ldots,r_{N}^{x}\rangle \langle r_{1}^{x},\ldots, r_{N}^{x}|\right)
	\\&
	\sum_{s^{x}\in \Omega_{x},}\left(s_{b}^{x}|s_{1}^{x},\ldots,s_{a}^{x}+1,\ldots,s_{b}^{x}-1,\ldots s_{N}^{x}\rangle \langle s_{1}^{x},\ldots,s_{N}^{x}|\right)
	\\&
	\sum_{n_{x}\in\Omega_{x}}\left(\frac{\nu!}{n_{1}!\ldots n_{N}!}|n_{1}^{x},\ldots,n_{N}^{x}\rangle \langle n_{1}^{x},\ldots, n_{N}^{x}|\right)
 \\=&\sum_{r^{x}\in \Omega_{x}}
	r_{a}^{x}|r_{1}^{x},\ldots,r_{N}^{x}\rangle \langle r_{1}^{x},\ldots,r_{a}^{x}-1,\ldots,r_{b}^{x}-1,\ldots,r_{N}^{x}|
	\\=&
	\left(E_{ba}^{x}\right)^{T}
\end{align*}
where in the up to last equation we used the orthogonality relation \eqref{ortho}. 
Equation \eqref{transpositionPropertyR} implies that 
\begin{equation}\label{transpositionPropertyH}
    H_{x,y}^{T}=\left(R_{x}R_{y}\right)H_{x,y}\left(R_{x}R_{y}\right)^{-1}
\end{equation}
We now search for a symmetry of $H_{x,y}$. Given $A,B\in gl(N)$, we say that $A$ is a symmetry of $B$ if 
\begin{equation}
    [A,B]=0
\end{equation}
Moreover, since the coproduct is a Lie algebra homomorphism we have that 
\begin{equation}\label{symmetryCoproduct}
    [A,B]=0\quad \Longrightarrow\quad \left[\Delta(A),\Delta(B)\right]=0
\end{equation}
Since $H_{x,y}$ is a linear function of the coproduct of the second Casimir, as a consequence of \eqref{symmetryCoproduct}, we can search for a symmetry of $C_{2}$. $C_{2}$ is a central for $gl(N)$, i.e. it commutes with all the element of the the algebra. We consider 
\begin{equation}\label{equationE}
    E=\sum_{a=2}^{N}E_{a1}
\end{equation}
To obtain a product structure of the elements \eqref{dualityElements} of the duality matrix, we take the exponential of this matrix, i.e. we introduce the symmetry
\begin{equation}
    S=\exp{(\Delta(E))}
\end{equation}
Since $[C_{2},E]=0$, we have that 
\begin{equation}\label{commutationSC}
    \left[S,\Delta(C_{2})\right]=0
\end{equation}
For fixed $(x,y)\in \mathcal{E}$ we introduce
\begin{equation}
    S_{x,y}=\exp{(\Delta^{x,y}(E))}=\exp{(E^{x})}\exp{(E^{y})}
\end{equation}
where $\Delta^{x,y}(E)$ is a copy of the coproduct acting on sites $x$ and $y$. As a consequence of \eqref{commutationSC}, we have that 
\begin{equation}\label{symmetryH}
    \left[S_{x,y},H_{x,y}\right]=0
\end{equation}
i.e. $S_{x,y}$ is a symmetry of $H_{x,y}$. \\ Finally, we obtain that 
\begin{equation}
    \begin{split}
        H_{x,y}^{T}D&=(R_{x}R_{y})H_{x,y}(R_{x}R_{y})^{-1}\left(d_{x}\otimes\dd\right)\left(d_{y}\otimes\mathcal{D}_{\widehat{y}}\right)\prod_{z\in V\,:\, z\neq x,y}\left(d_{z}\otimes \mathcal{D}_{\widehat{z}}\right)
        \\&=\left(R_{x}\exp{(E^{x})}\otimes \dd\right)\left(R_{y}\exp{(E^{y})}\otimes \mathcal{D}_{\widehat{y}}\right)H_{x,y}\prod_{z\in V\,:\, z\neq x,y}\left(d_{z}\otimes \mathcal{D}_{\widehat{z}}\right)
        \\&=
        DH_{x,y}
    \end{split}
\end{equation}
 where we used \eqref{transpositionPropertyH} and \eqref{symmetryH} in the second equality. Thus, \eqref{edgeDualRealtion} is proved. 
 \newline\newline
 \textbf{Remark}: it is important to notice that the elements of the duality matrix \eqref{dualityElements} are well defined only if at each site $x\in V$ and for every species $k\in \{2,\ldots,N\}$ the number of dual particles $r_{x}^{k}$ is lower or equal than the number of original particles $n_{k}^{x}$. This implies that the dynamics of the dual process is simpler, because described by a lower number of particles. By the way, it is possible to perform computations similar to the one made in this proof choosing instead of  \eqref{equationE} {\color{red} $E=\sum_{a=2}^{N}E_{1a}$ or $E=\sum_{a=2}^{N}E_{aa}$.}
 {\color{blue} That's confusing. What is E?}
 
 
 However, they would lead to duality matrices where the number of dual particles is greater than the number of original ones, loosing the simplification of the dynamics of the dual process. 
 \begin{flushright}
     $\square$
 \end{flushright}
 \paragraph{Proof of \eqref{siteDualRelation}.} To prove \eqref{siteDualRelation} we need two transform via Hadamard formula \eqref{HadamardFormula}, the transposed of the site Hamiltonian \eqref{siteDualHamiltonian} and then introduce properly a creation operator acting on an extra-site $\widehat{x}$ that we connect to every site $x\in V$ of the graph $G$. \\
 Considering $A,B\in gl(N)$, we can write the Hadamard formula as 
 \begin{equation}\label{HadamardFormula}
     \exp{(-B)}A\exp{(B)}=A-\left[B,A\right]+\frac{1}{2!}\left[B,\left[B,A\right]\right]-\frac{1}{3!}\left[B,\left[B,\left[B,A\right]\right]\right]+\ldots
 \end{equation}
For the following of the proof we need the application of the Hadamard formula with $B=E$ defined in \eqref{equationE} and with $A$ equal to some of the generator $E_{ab}$ of the Lie algebra. Therefore we compute 
\begin{enumerate}
    \item for $A=E_{\ell 1}$ with $\ell\in \{2,\ldots,N\}$, we obtain 
    \begin{equation}\label{HT_El1}
        \exp{(-E)}E_{\ell 1}\exp{(E)}=E_{\ell 1}
    \end{equation}
    because $E_{\ell 1}$ commutes with $E$
    \item for $A=E_{\ell\ell}$ with $\ell \in \{2,\ldots,N\}$, we obtain 
    \begin{equation}\label{HT_Ell}
        \exp{(-E)}E_{\ell \ell}\exp{(E)}=E_{\ell \ell}+E_{\ell 1}
    \end{equation}
Indeed, using \eqref{eq:comgl} we have 
    \begin{equation}
       \left[E, E_{\ell\ell}\right]=\sum_{a=2}^{N}\left(E_{a\ell}\delta_{\ell 1}-E_{\ell 1}\delta_{a\ell}\right)=-E_{l1}
    \end{equation}
    Inserting the above commutator in \eqref{HadamardFormula}, we obtain \eqref{HT_Ell}.
    \item for $A=E_{1\ell}$ with $\ell\in \{2,\ldots,N\}$, we obtain 
    \begin{equation}\label{HT-E1l}
        \exp{(-E)}E_{1 \ell}\exp{(E)}=E_{1\ell}+E_{11}-\sum_{j=2}^{N}\left(E_{j1}+E_{j\ell}\right)
    \end{equation}
  \com{Why indices have names $\ell, k, j, a, b$? Fix one set for the generators}
    
    Indeed, using \eqref{eq:comgl} we have 
    \begin{equation}
[E,E_{1\ell}]=\sum_{a=2}^{N}\left(E_{a\ell}\delta_{11}-E_{11}\delta_{a\ell}\right)=\sum_{j=2}^{N}E_{j\ell}-E_{11};
\end{equation}
and 
\begin{equation}
\begin{split}
\left[E,[E,E_{1\ell}]\right]&=\sum_{b=2}^{N}\sum_{j=2}^{N}\left[E_{b1},E_{j\ell}\right]-\sum_{b=2}^{N}\left[E_{b1},E_{11}\right]
\\&=
\sum_{j,b=2}^{N}\left(E_{b\ell}\delta_{j1}-E_{j1}\delta_{b\ell}\right)-\sum_{b=2}^{N}\left(E_{b1}\delta_{11}-E_{11}\delta_{b1}\right)
\\=&
-2\sum_{j=2}^{N}E_{j1};
\end{split}
\end{equation}
Inserting the above commutators in \eqref{HadamardFormula}, we obtain \eqref{HT-E1l}.
\item for $A=E_{k\ell}$ with $k,\ell \in \{2,\ldots,N\}$ we obtain 
\begin{equation}\label{HT-Ekl}
    \exp{(-E)}E_{\ell k}\exp{(E)}=E_{k\ell}+E_{\ell 1}
\end{equation}
   Indeed, using \eqref{eq:comgl} we have 
\begin{equation}
[E,E_{\ell k}]=\sum_{a=2}^{N}\left(E_{ak}\delta_{\ell 1}-E_{\ell 1}\delta_{ak}\right)=-E_{\ell 1};
\end{equation}
\begin{equation}
[E,E_{k\ell}]=\sum_{a=2}^{N}\left(E_{a\ell}\delta_{k1}-E_{k1}\delta_{\ell a}\right)=-E_{k1};
\end{equation}
Inserting the above commutator in \eqref{HadamardFormula}, we obtain \eqref{HT-Ekl}.
\item for $A=E_{11}$ we obtain 
\begin{equation}\label{HT-E11}
    \exp{(-E)}E_{11}\exp{(E)}=E_{11}-\sum_{j=2}^{N}E_{j1}
\end{equation}
  Indeed, using \eqref{eq:comgl} we have 

\begin{align*}
[E,E_{11}]=\sum_{a=2}^{N}\left(E_{a1}\delta_{11}-E_{1 1}\delta_{a1}\right)=\sum_{j=2}^{N}E_{j1};
\end{align*}
  Inserting the above commutator in \eqref{HadamardFormula}, we obtain \eqref{HT-E11}.
\end{enumerate}
Using \eqref{transpositionPropertyR} we write the transpose of site Hamiltonian \eqref{siteHamiltonian} 
\begin{equation}
    \begin{split}
H_{x}^{T}=\sum_{k,l=0}^{N}\alpha_{k}^{x}\left(E_{k\ell}^{x}-E_{\ell\ell}^{x}\right)^{T}=R_{x}\sum_{k,\ell=0}^{N}\alpha_{k}^{x}\left(E_{\ell k}^{x}-E_{\ell\ell}^{x}\right)R_{x}^{-1}
    \end{split}
\end{equation}
First, we multiply both sides by $R_{x}\exp{(E^{x})}$
\begin{equation}
    H_{x}^{T}R_{x}\exp{(E^{x})}=R_{x}\sum_{k,\ell =0}^{N}\alpha_{k}^{x}\left(E_{\ell k}^{x}-E_{\ell\ell}^{x}\right)\exp{(E^{x})}
\end{equation}
then,we insert the identity $I=\exp{(E^{x})}\exp{(-E^{x})}$ in the right hand side
\begin{equation}\label{intermediateTransposeSite}
H_{x}^{T}R_{x}\exp{(E^{x})}=R_{x}\exp{(E^{x})}\exp{(-E^{x})}\sum_{k,\ell=0}^{N}\alpha_{k}^{x}\left(E_{\ell k}^{x}-E_{\ell\ell}^{x}\right)\exp{(E^{x})}
\end{equation}
where $E^{x}$ is a copy of $E$ acting on site $x\in V$ and $E_{ab}^{x}$ with $a,b\in \{1,\ldots,N\}$ are copies of $E_{ab}$ acting at site $x\in V$. Using \eqref{HT_El1}, \eqref{HT_Ell},\eqref{HT-E1l}, \eqref{HT-Ekl}, \eqref{HT-E11} we have that 
\begin{align*}
    &\exp{(-E^{x})}\sum_{k=1}^{N}\sum_{\ell=1}^{N}\alpha_{k}^{x}\left(E_{\ell k}^{x}-E_{\ell\ell}^{x}\right)\exp{(E^{x})}
    \\=&
     \exp{(-E^{x})}\sum_{k=2}^{N}\alpha_{k}^{x}\left(E_{1k}^{x}-E_{11}^{x}\right)\exp{(E^{x})}
     + \exp{(-E^{x})}\sum_{k=1}^{N}\sum_{\ell=2}^{N}\alpha_{k}^{x}\left(E_{\ell k}^{x}-E_{\ell\ell}^{x}\right)\exp{(E^{x})}
     \\=&
     \sum_{k=2}^{N}\alpha_{k}^{x}\left(E_{1k}+E_{11}^{x}-\sum_{a=2}^{N}(E_{ak}^{x}+E_{a1}^{x})-E_{11}^{x}+\sum_{a=2}^{N}E_{a1}^{x}\right)+\sum_{k=1}^{N}\sum_{\ell =2}^{N}\alpha_{k}^{x}\left(E_{\ell k}^{x}+E_{\ell 1}^{x}-E_{\ell\ell}^{x}-E_{\ell 1}^{x}\right)
     \\=&
\sum_{k=2}^{N}\alpha_{k}^{x}E_{1k}^{x}-\sum_{k=1}^{N}\alpha_{k}^{x}\sum_{\ell=2}^{N}E_{\ell\ell}^{x}
=
     \sum_{k=2}^{N}\left(\alpha_{k}^{x}E_{1k}^{x}-\sum_{i=1}^{N}\alpha_{i}^{x}E_{kk}^{x}\right)
=
\sum_{i=1}^{N}\alpha_{i}^{x}\sum_{k=2}^{N}\left(\frac{\alpha_{k}^{x}}{\sum_{i=1}^{N}\alpha_{i}^{x}}E_{1k}^{x}-E_{kk}^{x}\right)
\\=&
\sum_{i=1}^{N}\alpha_{i}^{x}\sum_{k=2}^{N}\left(\rho_{k}^{x}E_{1k}^{x}-E_{kk}^{x}\right)
 \end{align*}
Thus, we rewrite \eqref{intermediateTransposeSite} as
\begin{equation}\label{siteHadamardI}
H_{x}^{T}R_{x}\exp{(E^{x})}=R_{x}\exp{(E^{x})}\sum_{i=1}^{N}\alpha_{i}^{x}\sum_{k=2}^{N}\left(\rho_{k}^{x}E_{1k}^{x}-E_{kk}^{x}\right)
\end{equation}
Making the tensor product on both sides of \eqref{siteHadamardI} by 
\begin{equation}
\sum_{m_{2}^{\widehat{x}},\ldots,m_{N}^{\widehat{x}}=0}^{\infty}\prod_{k=2}^{N}\left(\rho_{k}^{x}\right)^{m_{k}^{\widehat{x}}}\langle m_{2}^{\widehat{x}},\ldots,m_{N}^{\widehat{x}}|
\end{equation}
and we obtain 
\begin{equation}\label{siteHadamardII}
    \begin{split}
&H_{x,y}^{T}d_{x}\otimes\sum_{m_{2}^{\widehat{x}},\ldots,m_{N}^{\widehat{x}}=0}^{\infty}\prod_{k=2}^{N}\left(\rho_{k}^{x}\right)^{m_{k}^{\widehat{x}}}\langle m_{2}^{\widehat{x}},\ldots,m_{N}^{\widehat{x}}|
\\=&
d_{x}\otimes \sum_{m_{2}^{\widehat{x}},\ldots,m_{N}^{\widehat{x}}=0}^{\infty}\prod_{k=2}^{N}\left(\rho_{k}^{x}\right)^{m_{k}^{\widehat{x}}}\langle m_{2}^{\widehat{x}},\ldots,m_{N}^{\widehat{x}}|\sum_{i=1}^{N}\alpha_{i}^{x}\sum_{k=1}^{N}\left(\rho_{k}^{x}E_{1k}^{x}-E_{kk}\right)
    \end{split}
\end{equation}
Recalling the action of the bosonic creation operator acting at site $\widehat{x}$ and on the species $k\in \{2,\ldots,N\}$ we have that 
\begin{equation}\label{bosonicKX}
    \langle m_{2}^{\widehat{x}},\ldots,m_{k}^{\widehat{x}}+1,\ldots,m_{N}^{\widehat{x}}|=  \langle m_{2}^{\widehat{x}},\ldots,m_{k}^{\widehat{x}},\ldots,m_{N}^{\widehat{x}}|(a^{\dagger})^{\widehat{x}}_{k}
\end{equation}
Using \eqref{bosonicKX} by a fixed $k\in \{2,\ldots,N\}$ we rewrite the term of \eqref{siteHadamardII} with Lie generator $E_{1k}^{x}$ on the right hand side of \eqref{siteHadamardII} as 
\begin{equation}
    \begin{split}
&\sum_{m_{2}^{\widehat{x}},\ldots,m_{N}^{\widehat{x}}=0}^{\infty}\prod_{k=2}^{N}\left(\rho_{k}^{x}\right)^{m_{k}^{\widehat{x}}}\langle m_{2}^{\widehat{x}},\ldots,m_{k}^{\widehat{x}},\ldots,m_{N}^{\widehat{x}}|\frac{\alpha_{k}^{x}}{\sum_{i=1}^{N}\alpha_{i}^{x}}E_{1k}^{x}
\\=&\sum_{m_{2}^{\widehat{x}},\ldots,m_{N}^{\widehat{x}}=0}^{\infty}\prod_{k=2}^{N}\left(\rho_{k}^{x}\right)^{m_{k}^{\widehat{x}}+1}\langle m_{2}^{\widehat{x}},\ldots,m_{k}^{\widehat{x}}+1,\ldots,m_{N}^{\widehat{x}}|(a^{\dagger})_{k}^{\widehat{x}}E_{1k}^{x}
\\=&
\sum_{m_{2}^{\widehat{x}},\ldots,m_{N}^{\widehat{x}}=0}^{\infty}\prod_{k=2}^{N}\left(\rho_{k}^{x}\right)^{m_{k}^{\widehat{x}}}\langle m_{2}^{\widehat{x}},\ldots,m_{k}^{\widehat{x}},\ldots,m_{N}^{\widehat{x}}|(a^{\dagger})_{k}^{\widehat{x}}E_{1k}^{x}
    \end{split}
\end{equation}
where, in the last equality, we performed a change of summation variable. Therefore, inserting this last equation in \eqref{siteHadamardII} we obtain 
\begin{equation}
    \begin{split}
     &H_{x,y}^{T}d_{x}\otimes \sum_{m_{2}^{\widehat{x}},\ldots,m_{N}^{\widehat{x}}=0}^{\infty}\prod_{k=2}^{N}\left(\rho_{k}^{x}\right)^{m_{k}^{\widehat{x}}}\langle m_{2}^{\widehat{x}},\ldots,m_{N}^{\widehat{x}}|
\\=&
d_{x}\otimes \sum_{m_{2}^{\widehat{x}},\ldots,m_{N}^{\widehat{x}}=0}^{\infty}\prod_{k=2}^{N}\left(\rho_{k}^{x}\right)^{m_{k}^{\widehat{x}}}\langle m_{2}^{\widehat{x}},\ldots,m_{N}^{\widehat{x}}|\sum_{i=1}^{N}\alpha_{i}^{x}\sum_{k=1}^{N}\left((a^{\dagger})_{k}^{\widehat{x}}E_{1k}^{x}-E_{kk}^{x}\right)   
    \end{split}
\end{equation}
Since the duality matrix \eqref{dualityMatrix} is product over sites, the above equality implies \eqref{siteDualRelation}. 
\begin{flushright}
$\square$
\end{flushright}









\section{Integrability}\label{sectionIntegrabiliy}
\subsection{The integrable chain and the matrix product ansatz}
In this section we specialize the multi-species stirring process to the geometry of the one dimensional chain with sites $\{1,\ldots,L\}$ where two reservoirs are connected with the end site $1$ and $L$. These boundaries, exchanges particle with the external environment, putting the chain out of equilibrium. To use integrability is better to work with Hamiltonians. The Hamiltonian that we consider here is obtained by \eqref{OriginalHamiltonian} assuming that the conducatances are
\begin{equation}
    \omega_{x,y}=\begin{cases}
    1 \quad \text{if}\quad |x-y|=1\\
    0\quad \text{otherwise}
    \end{cases}
\end{equation}
and the local inhomogeneities are
\begin{equation}
    \Gamma_{x}=\begin{cases}
        1\quad \text{if} \quad x\in \{1,L\}\\
        0\quad \text{otherwise}
    \end{cases}
    \end{equation}
We further assume that the maximal occupancy in each site is $\nu=1$, i.e. that there is an \textit{hard-core interaction}.
We will denote the element of the representation of the Lie algebra by lowercase letters, i.e. $e_{ab}$, since we are in the lowest spin case (the fundamental representation of algebra). \\
The state space of the chain is 
\begin{equation}
    \Omega=\bigtimes_{x=1}^{L}\Omega_{x}
\end{equation}
where 
\begin{equation}
	\Omega_{x}=\left\{|n\rangle=|n_{1},\ldots,n_{N}\rangle\;:\;n\in\mathbb{N}_0^N\;\;|n|=1\right\}
\end{equation} 
The Hamiltonian is
\begin{equation}\label{hamiltonian}
	H=B_{1}+H_{bulk}+B_{L}
\end{equation}
where
\begin{equation}
    H_{bulk}=\sum_{x=1}^{L-1}\mathcal{H}_{x,x+1}
\end{equation}
Moreover, for a fixed site $x\in \{1,\ldots,L\}$ we have 
\begin{equation}
	\begin{split}
		\mathcal{H}=P-\id
	\end{split}
\end{equation}
with 
\begin{equation}
	P=\sum_{a,b=1}^Ne_{ab}\otimes e_{ba}
\end{equation} 
such that
\begin{equation}
	\begin{split}
		\mathcal{H}|n\rangle\otimes   |m\rangle&=|m\rangle \otimes |n\rangle-|n\rangle \otimes|m\rangle
	\end{split}
\end{equation}
The boundary terms are given by 
\begin{equation}
	B_{1}=\begin{pmatrix}
		\alpha_{1}-1&\alpha_{1}&\ldots&\alpha_{1}\\
		\alpha_{2}&\alpha_{2}-1&\ldots&\alpha_{2}\\
		\vdots&\vdots&\vdots&\vdots\\
		\alpha_{N}&\alpha_{N}&\ldots&\alpha_{N}-1
	\end{pmatrix}
\end{equation}
\begin{equation}
	B_{L}=\begin{pmatrix}
		\beta_{1}-1&\beta_{1}&\ldots&\beta_{1}\\
		\beta_{2}&\beta_{2}-1&\ldots&\beta_{2}\\
		\vdots&\vdots&\vdots&\vdots\\
		\beta_{N}&\beta_{N}&\ldots&\beta_{N}-1
	\end{pmatrix}
\end{equation}
where we assume that the parameters satisfy
\begin{equation}\label{ratesConditions}
	\sum_{a=2}^{N}\alpha_{a}=1,\qquad\sum_{a=2}^{N}\beta_{a}=1
\end{equation} 
We define also 
\begin{equation}\label{lambdaConditions}
	\lambda_{a}=\alpha_{a}-\beta_{a}\quad\text{with}\quad \sum_{a=2}^{N}\lambda_{a}=0\,.
\end{equation}
Under these assumption the Hamiltonian \eqref{hamiltonian} is integrable and a matrix product anstz (MPA) for the non-equilibrium steady state is available (see\cite{vanicat}). We recall the main points of this ansatz. \\
 Calling $|\mathbb{P}(t)\rangle$ a vector that contains the probability of every configuration of the chain with Hamiltonian \eqref{hamiltonian} at time $t\geq 0$, its evolution equation is given by 
\begin{equation}
    \frac{d|\mathbb{P}(t)}{dt}=H|\mathbb{P}(t)\rangle
\end{equation}
We define the steady state of $H$ defined in \eqref{hamiltonian} a vector, denoted by $|\Psi\rangle$ that contains the probabilities of every configuration of the chain with respect to the stationary distribution, i.e. the probability distribution that the chain reaches when $t\to \infty$. We observe that, since the Markov chain with Hamiltonian \eqref{hamiltonian} is positive recurrent there exists a unique stationary distribution and, moreover, this is distribution is reached when $t\to \infty$ starting for every arbitrary initial distribution. 
 Therefore, to find the stationary state, we aim to compute $|\Psi\rangle$ such that 
\begin{equation}
	H|\Psi\rangle =0
\end{equation}
The MPA states the following
\begin{equation}
	|\Psi\rangle=\frac{1}{Z^{L}}\langle W|\begin{pmatrix}
		X_{1}\\
		\vdots\\
		X_{N}
	\end{pmatrix}\otimes \ldots\otimes \begin{pmatrix}
		X_{1}\\
		\vdots\\
		X_{N}
	\end{pmatrix}|V\rangle
\end{equation}
with the normalization 
\begin{equation}
	Z_{L}=\langle W|(X_{1}+\ldots +X_{N})^{L}|V\rangle
\end{equation}
where $X_{a}$ for $a=1,\ldots,N$ are operators on an auxiliary space $V_{0}$ while $|V\rangle\in V_{0}$ and $\langle W|\in V_{0}^{*}$ ( where $V^{*}$ denotes the algebraic dual space), such that $\langle W|V\rangle=1$. $\langle W|$ and $|V\rangle$ are sometimes called boundary vectors. These operators $(X_{a})_{a\in \{1,\ldots,N\}}$ fulfil the commutators
\begin{equation}\label{bulk}
	\left[X_{a},X_{b}\right]=\lambda_{a}X_{b}-\lambda_{b}X_{a}\qquad\forall a,b=1,\ldots,N
\end{equation}
and their action on the boundary vectors are
\begin{equation}\label{leftBoundary}
	\langle W|\left(\alpha_{a}(X_{1}+\ldots+X_{N})-X_{a}\right)=\lambda_{a}\langle W|\qquad\forall a=1,\ldots,N
\end{equation}
\begin{equation}\label{rightBoundary}
	\left(\beta_{a}(X_{1}+\ldots+X_{N})-X_{a}\right)|V\rangle=-\lambda_{a}|V\rangle\qquad\forall a=1,\ldots,N
\end{equation}
\newline
\textbf{Remark}: only $N-1$ of \eqref{leftBoundary} and \eqref{rightBoundary} are independent, indeed because of \eqref{ratesConditions} and \eqref{lambdaConditions}
	\begin{equation}
		\sum_{a=1}^{N}	\langle W|\left(\alpha_{a}(X_{1}+\ldots+X_{N})-X_{a}\right)=\sum_{a=1}^{N}\lambda_{a}\langle W|
	\end{equation}
	gives $0=0$. Similar computations can be done for the right boundary.
\subsection{The exact steady state for the multi-species stirring process on a chain}\label{subsection-exact}
We aim to find an explicit for for the steady state $|\Psi\rangle$. The MPA states the form of the steady state in function of an abstract algebra of operators. If an explicit representation of $(X_{a})_{a=1,\ldots,N}$ was available, it would be possible to compute explicitly all the correlations and currents of particles for the Hamiltonian \eqref{hamiltonian}. Another way of computing this steady state, without passing through a representation, is using the commutators \eqref{bulk} and the actions \eqref{leftBoundary} and \eqref{rightBoundary}. For the one and two point correlation and for the currents this has been done in \cite{vanicat2017exact}. Inspired by this last reference, in this subsection, we aim to find exact formulas for the steady state of \eqref{hamiltonian} using only the \eqref{bulk}, \eqref{leftBoundary} and \eqref{rightBoundary}. In subsection \eqref{correlation-section}, this result will leads to exact correlation for arbitrary number of particles of type $\{2,\ldots,N\}$. This aim will be achieved by the help of a similarity transformation $\mathcal{S}$ that makes commutation relation simpler, and allows to find the steady state of a transformed Hamiltonian, that will be denoted by $\mathbb{H}$, with lower-triangular left boundary and diagonal right boundary. Therefore, inverting this similarity, one can retrieve the steady state of \eqref{hamiltonian}. A fundamental 'ingredient' of these computation is duality, that has been proved in section~\ref{sectionDuality}. Indeed, the similarity transformation $\mathcal{S}$ relies on an "intermediate transformation" $V$ (see \eqref{transformationV}) that is, up to the diagonal operator $R$, the duality matrix. \com{clarify}
\subsubsection{The similarity transformation}
We aim to construct a similarity transformation that maps the Hamiltonian \eqref{hamiltonian} into an other $\mathbb{H}$ that has left boundary lower-triangular and right boundary diagonal. Consider the transformation
\begin{equation}\label{transformationV}
 V:=\exp{\left(\sum_{a=2}^{N}e_{a1}\right)}=\mathbbm{1}+\sum_{a=2}^{N}e_{a1}
\end{equation}
Observe that it is the the duality matrix at site $x\in \{1,\ldots,L\}$ up to the multiplication by $R_{x}$. By taking the tensor product over the chain we define
\begin{equation}
    \mathcal{V}=V^{\otimes L}
\end{equation}
 This matrix $\mathcal{V}$ is invertible and we introduce 
\begin{equation}\label{hatHamiltonian}
    \widehat{H}:=\widehat{B}_{1}+\mathcal{H}+\widehat{B}_{L}
\end{equation}
such that 
\begin{equation}\label{similarV}
   H=\left(\mathcal{V}^{-1}\right)^{T}\widehat{H}^{T}V^{T}
\end{equation}
Using \eqref{transformationV} we introduce the \textit{similarity transformation} 
\begin{equation}\label{similarity}
    S:=\sum_{a=2}^{N}\beta_{a}\sum_{b=1}^{N}e_{ab}+V=-\sum_{a=2}^{N}\beta_{a}\sum_{b=1}^{N}e_{ab}+\exp{\left(\sum_{a=2}^{N}e_{1a}\right)}
\end{equation}
Taking the tensor product over the sites of the chain we further define
\begin{equation}
    \mathcal{S}:=S^{\otimes L}
\end{equation}
Therefore, we call
\begin{equation}\label{bbHamiltonian}
    \mathbb{H}=\mathbb{B}_{1}+H_{bulk}+\mathbb{B}_{L}
\end{equation}
such that 
\begin{equation}\label{similarS}
    H=\mathcal{S}^{-1} \mathbb{H}\mathcal{S}
\end{equation}
The Hamiltonian $\mathbb{H}$ has lower-triangular left boundary and diagonal right boundary, i.e.
\begin{equation}
    \mathbb{B}_{1}=SB_{1}S^{-1}=e_{11}-\mathbbm{1}+\sum_{a=2}^{N}\left(\alpha_{a}-\beta_{a}\right)e_{a1}=\begin{pmatrix}
		0&0&\ldots&0&0\\
		\alpha_{2}-\beta_{2}&-1&\ldots&0&0\\
		\vdots&\vdots&\ddots&\vdots&\vdots\\
		\alpha_{N-1}-\beta_{N-1}&0&\ldots&-1&0\\
		\alpha_{N}-\beta_{N}&0&\ldots&0&-1
	\end{pmatrix}
\end{equation}
and 
\begin{equation}
    \mathbb{B}_{L}=SB_{L}S^{-1}=e_{11}-\mathbbm{1}=\begin{pmatrix}
		0&0&\ldots&0&0\\
		0&-1&\ldots&0&0\\
		\vdots&\vdots&\ddots&\vdots&\vdots\\
		0&0&\ldots&-1&0\\
		0&0&\ldots&0&-1
  \end{pmatrix}
\end{equation}
The bulk part of $\mathbb{H}$ is left unchanged by this transformation, i.e.
\begin{equation}
    \mathcal{H}_{x,x+1}=\left(S\otimes S\right)\mathcal{H}_{x,x+1}\left(S\otimes S\right)^{-1}
\end{equation}
In the following of our analysis it will be useful to have a relation between \eqref{hatHamiltonian} and \eqref{bbHamiltonian}. This relation reads as
\begin{equation}
    \widehat{H}=\mathcal{T}^{-1}\mathbb{H}\;\mathcal{T}
\end{equation}
where
\begin{equation}
    T=\left(S^{-1}\right)^{T}V=\exp{\left(\sum_{a=2}^{N}\beta_{a}e_{1a}\right)}
\end{equation}
and 
\begin{equation}
    \mathcal{T}:=T^{\otimes L}
\end{equation}
Indeed, by imposing equal \eqref{similarV} and \eqref{similarS} we obtain 
\begin{align*}
    		\left(\mathcal{V}^{-1}\right)^{T}\widehat{H}^{T}\mathcal{V}^{T}&=\mathcal{S}\mathbb{H}\mathcal{S}^{-1}\\
		\mathcal{V}\widehat{H}\mathcal{V}^{-1}&=\mathcal{S}^{T}\mathbb{H}^{T}\left(\mathcal{S}^{-1}\right)^{T}\\
		\widehat{H}&=\mathcal{V}^{-1}\mathcal{S}^{T}\mathbb{H}^{T}\left(\mathcal{S}^{-1}\right)^{T}\mathcal{V}\\
		\widehat{H}&=\mathcal{T}^{-1}\mathbb{H}^{T}\mathcal{T}
\end{align*}
Associated with these new Hamiltonians $\widehat{H}$ and $\mathbb{H}$ we define the following steady states
\begin{equation}
	\mathbb{H}|\widetilde{\Psi}\rangle=0\qquad \widehat{H}^{T}|\widehat{\Psi}\rangle=0
\end{equation}
The steady states $|\Psi\rangle$, $|\widehat{\Psi}\rangle$ and $|\widetilde{\Psi}\rangle$ are linked by 
\begin{align}
	|\Psi\rangle=&\left(\mathcal{S}^{-1}\right)|\widetilde{\Psi}\rangle\label{SteadystateTilde}
 \\|\Psi\rangle =&\left((\mathcal{{V}}^{T})^{-1}\right)|\widehat{\Psi}\rangle \label{SteadystateHat}
 \\|\widehat{\Psi}\rangle=&\left(\mathcal{T}^{T}\right)|\widetilde{\Psi}\rangle\label{SteadyStateHatTilde}
\end{align}
In the following subsections we will first compute the steady state $|\widetilde{\Psi}\rangle$. Then, using the transformation $\mathcal{T}$ we will retrieve $|\widehat{\Psi}\rangle$. Finally, using the transformation $\mathcal{V}$ we will find $|\Psi\rangle$. 
\subsubsection{The steady state of $\mathbb{H}$}
We state the following result. The steady state of the Hamiltonian $\mathbb{H}$ defined in \eqref{bbHamiltonian} is given by 
\begin{equation}\label{ResulsBasis}
	|\widetilde{\Psi}\rangle=\sum_{s_{1},\ldots,s_{L}=1}^{N}\frac{\Gamma\left(2+\sum_{i=1}^{L}\delta_{s_{i},1}\right)}{\Gamma\left(L+2\right)}\prod_{i=1}^{L}\left[\lambda_{s_{i}}\left(1+\sum_{j=i}^{L}\delta_{s_{j},1}\right)\right]^{1-\delta_{s_{i},1}}|\mathbf{\bm{s}}\rangle
\end{equation}
where 
\begin{equation}
    \delta_{s_{i},1}:=\begin{cases}
        1\quad \text{if}\quad s_{i}=1\\
        0\quad \text{otherwise}
    \end{cases}
\end{equation}
and the basis vector $|\bm{s}\rangle =\bigotimes_{i=1}^{L}|s_{i}\rangle$ with $s_{i}\in\{1,\ldots,N\}$ denotes the species that occupies the site $i\in \{1,\ldots,L\}$.
\paragraph{Proof of formula \eqref{ResulsBasis}} 
To prove \eqref{ResulsBasis} we will consider the vector composed by the operators of the MPA $(X_{1},\ldots,X_{N})$ and we will act on it with the similarity transformation \eqref{similarity} obtaining new operators $(\Xt_{1},\ldots,\Xt_{N})$ that will satisfy simpler commutators than \eqref{bulk} and simpler actions than \eqref{leftBoundary} and \eqref{rightBoundary}. This will allow to write explicitly the steady state \eqref{ResulsBasis}.
For the sake of notation, we denote by 
\begin{equation}
    \begin{pmatrix}
		X_{1}\\ 
		X_{a}
	\end{pmatrix}=\begin{pmatrix}
	    X_{1}\\
     \vdots\\
     X_{N}
	\end{pmatrix}\quad \forall a\in \{2,\ldots,N\}
\end{equation}
We obtain the new operators of the MPA that by transforming the original one by \eqref{similarity} 
\begin{equation}\label{Xtildes2b}
	\begin{pmatrix}
		\Xt_{1}\\ 
		\Xt_{a}
	\end{pmatrix} =S\begin{pmatrix}
		X_{1}\\X_{a}
	\end{pmatrix}=\begin{pmatrix} 
		X_{1}+\ldots +X_{N}\\
		X_{a}-\beta_{a}(X_{1}+\ldots+X_{N})\\ 
	\end{pmatrix}=\begin{pmatrix} 
		X_{1}+\ldots +X_{N}\\
		X_{a}-\beta_{a}\Xt_{1}\\ 
	\end{pmatrix}\qquad \forall a=2,\ldots N
\end{equation}
We can also reverse the transformation by $S^{-1}$ and get back: 
\begin{equation}\label{Xes}
	\begin{pmatrix}
		X_{1}\\
		X_{a} 
	\end{pmatrix} =S^{-1}\begin{pmatrix}
		\widetilde{X}_{1}\\
		\widetilde{X}_{a}
	\end{pmatrix}=\begin{pmatrix}
		\beta_1\Xt_{1}-(\Xt_{2}+\ldots+\Xt_{N})\\
		\Xt_{a}+\beta_{a}\Xt_{1}\\ 
	\end{pmatrix}\qquad\forall a\in \{2,\ldots,N\}
\end{equation}
Summing over $b\in \{1,\ldots,N\}$ in \eqref{bulk} we have
\begin{equation} 
	\left[X_{a},\Xt_{1}\right]=\lambda_{a}\Xt_{1}\qquad\forall a=1,\ldots,N
\end{equation}
therefore, we obtain the commutation relations for $\Xt_{a}$ $\forall a\in \{1,\ldots,N\}$
\begin{equation}\label{commutationsBulk}
	\left[\Xt_{a},\Xt_{1}\right]=\lambda_{a}\Xt_{1}\qquad \forall a\in \{2,\ldots,N\}
\end{equation}
Moreover using \eqref{Xtildes2b} and \eqref{Xes}, the action of $\Xt_{a}$ $\forall a\in \{1,\ldots,N\}$ on the boundary vectors are given by 
\begin{equation}\label{commLEFT}
	\langle W|\left(\lambda_{a}\Xt_{1}-\Xt_{a}\right)=\lambda_{a}\langle W|\qquad\forall a=2,\ldots,N
\end{equation}
\begin{equation}\label{commRIGHT}
	\Xt_{a} |V\rangle= \lambda_{a}|V\rangle\qquad\forall a=2,\ldots,N
\end{equation} 
The vector $|\widetilde{\Psi}\rangle$ is written as:
\begin{equation}\label{toDetermineTilde}
	|\widetilde{\Psi}\rangle = \frac{1}{Z_{L}}\sum_{\mathbf{n}\in \Omega}\langle W|\prod_{i=1}^{L}\prod_{a=1}^{N}\widetilde{X}_{a}^{n_{a}^{i}}
	|V \rangle \,|\mathbf{n}\rangle
\end{equation}
where the basis is 
$$
|\mathbf{n}\rangle =|n_{1}^{1},\ldots,n_{N}^{1}\rangle \otimes \ldots\otimes |n_{1}^{L},\ldots,n_{N}^{L}\rangle
$$
such that for each site $x\in \{1,\ldots,L\}$ $|n^{x}\rangle=|n_{1}^{x},\ldots,n_{N}^{x}\rangle\in \Omega_{x}$. 
We also observe that the vector $|\widetilde{\Psi}\rangle $ has dimension $N^{L}$. \\
To determine the steady state \eqref{toDetermineTilde}, we need to compute the coefficient $\langle W|\prod_{i=1}^{L}\prod_{a=1}^{N}\widetilde{X}_{a}^{m_{a}^{i}}
|V \rangle$ and the normalization $Z_{L}$. We use the new commutations relations \eqref{commutationsBulk},\eqref{commLEFT} and \eqref{commRIGHT}.\\
We  rewrite \eqref{commutationsBulk} as:
\begin{equation}\label{UsefulRelation}
	\widetilde{X}_{a}\widetilde{X}_{1}=\lambda_{a}\widetilde{X}_{1}+\widetilde{X}_{1}\widetilde{X}_{a}
\end{equation}
Fixing $ a\in \{2,\ldots,N\}$ and $\forall \ell,n\in \mathbb{N}$ and applying many times \eqref{UsefulRelation} we have
\begin{align*}
	\widetilde{X}_{a}^{n}\widetilde{X}_{1}^{\ell}&=\widetilde{X}_{a}^{n-1}\left(\lambda_{a}\widetilde{X}_{1}+\widetilde{X}_{1}\widetilde{X}_{a}\right)\widetilde{X}_{N}^{\ell-1}
	\\&=\lambda_{a}\widetilde{X}_{a}^{n-1}\widetilde{X}_{1}^{\ell}+\widetilde{X}_{a}^{n-1}\widetilde{X}_{1}\widetilde{X}_{a}\widetilde{X}_{1}^{\ell-1}
	\\&=
	\lambda_{a}\widetilde{X}_{a}^{n-2}\left(\lambda_{a}\widetilde{X}_{1}+\widetilde{X}_{1}\widetilde{X}_{a}\right)\widetilde{X}_{1}^{\ell-1}+\widetilde{X}_{a}^{n-2}\left(\lambda_{a}\widetilde{X}_{1}+\widetilde{X}_{1}\widetilde{X}_{a}\right)\left(\lambda_{a}\widetilde{X}_{1}+\widetilde{X}_{1}\widetilde{X}_{a}\right)\widetilde{X}_{1}^{\ell-1}
	\\&=\ldots\\&=
	\widetilde{X}_{1}^{\ell}\left(\widetilde{X}_{a}+\ell\lambda_{a}\right)^{n}
\end{align*}
therefore 
\begin{align*}
	\widetilde{X}_{1}^{n_{1}}\ldots\widetilde{X}_{N-1}^{n_{N-1}}\widetilde{X}_{N}^{n_{N}}=\widetilde{X}_{1}^{n_{1}}\prod_{a=2}^{N}\left(\widetilde{X}_{a}+n_{1}\lambda_{a}\right)^{n_{a}}
\end{align*}
By taking the product over the sites $i$ from $1$ to L we obtain 
\begin{equation}
	\prod_{i=1}^{L}\prod_{a=1}^{N}\widetilde{X}_{a}^{n_{a}^{i}}=\widetilde{X}_{1}^{\sum_{i=1}^{L}n_{1}^{l}}\prod_{i=1}^{L}\prod_{a=2}^{N}\left(\widetilde{X}_{a}+\lambda_{a}\sum_{j=i}^{L}n_{1}^{j}\right)^{n_{a}^{i}}
\end{equation}
Therefore, multiplying by the boundary vectors we have 
\begin{equation}
	\langle W|\prod_{i=1}^{L}\prod_{a=1}^{N}\widetilde{X}_{a}^{n_{a}^{i}}
	|V \rangle=\langle W|\widetilde{X}_{1}^{\sum_{i=1}^{L}n_{1}^{l}}\prod_{i=1}^{L}\prod_{a=2}^{N}\left(\widetilde{X}_{a}+\lambda_{a}\sum_{j=i}^{L}n_{1}^{j}\right)^{n_{a}^{i}}|V\rangle
\end{equation}
Using \eqref{commRIGHT}
\begin{align*}
	W|\prod_{i=1}^{L}\prod_{a=1}^{N}\widetilde{X}_{a}^{n_{a}^{i}}
	|V \rangle&=\langle W|\widetilde{X}_{1}^{\sum_{i=1}^{L}n_{1}^{i}}\prod_{i=1}^{L}\prod_{a=2}^{N}\left(\lambda_{a}+\lambda_{a}\sum_{j=i}^{L}n_{1}^{j}\right)^{n_{a}^{i}}|V\rangle
	\\&=
	\langle W|\widetilde{X}_{1}^{\sum_{i=1}^{L}n_{1}^{l}}\prod_{i=1}^{L}\prod_{a=2}^{N}\left(\lambda_{a}\right)^{n_{a}^{i}}\left(1+\sum_{j=i}^{L}n_{1}^{j}\right)^{n_{a}^{i}}|V\rangle
	\\&=
	\prod_{i=1}^{L}\prod_{a=2}^{N}\left(\lambda_{a}\right)^{n_{a}^{i}}\left(1+\sum_{j=i}^{L}n_{1}^{j}\right)^{n_{a}^{i}}\langle W|\widetilde{X}_{1}^{\sum_{i=1}^{L}n_{1}^{l}}|V\rangle
\end{align*}
To determine the state we need to compute $\langle W|\widetilde{X}_{1}^{\sum_{i=1}^{L}n_{1}^{l}}|V\rangle$. Calling $\sum_{i=1}^{L}n_{1}^{i}=n_{1}$ and taking an arbitrary $a\in \{2,\ldots,N\}$ we have
\begin{align*}
	\langle W|\widetilde{X}_{1}^{n_{1}}|V\rangle&=\langle W|\widetilde{X}_{1}\widetilde{X}_{1}^{n_{1}-1}|V\rangle=\langle W|\widetilde{X}_{1}^{n_{1}-1}|V\rangle +\langle W|\frac{1}{\lambda_{a}}\widetilde{X}_{a}\widetilde{X}_{1}^{n_{1}-1}|V\rangle
	\\&=
	\langle W|\widetilde{X}_{1}^{n_{1}-1}|V\rangle+\frac{1}{\lambda_{a}}\langle W|\widetilde{X}_{1}^{n_{1}-1}\left(\widetilde{X}_{a}+\lambda_{a}(n_{1}-1)\right)|V\rangle
	\\&=
	\langle W|\widetilde{X}_{1}^{n_{1}-1}|V\rangle+\left(n_{1}+1-1\right)\langle W|\widetilde{X}_{1}^{n_{1}-1}|V\rangle
	\\&=
	\left(2+n_{1}-1\right)\langle W|\widetilde{X}_{1}^{n_{1}-1}|V\rangle
	\\&=
	\frac{\Gamma(2+n_{1})}{\Gamma(2+n_{1}-1)}\langle W|\widetilde{X}_{1}^{n_{1}-1}|V\rangle
\end{align*}
This leads to the recursion relation
\begin{equation}
	\begin{cases}
		\langle W|\widetilde{X}_{1}^{n_{1}}|V\rangle=\frac{\Gamma(2+n_{1})}{\Gamma(2+n_{1}-1)}\langle W|\widetilde{X}_{1}^{n_{1}-1}|V\rangle\\
		\langle W|\widetilde{X}_{1}^{0}|V\rangle=1
	\end{cases}
\end{equation}
that implies 
\begin{align*}
	\langle W|\widetilde{X}_{1}^{n_{1}}|V\rangle&=\frac{\Gamma(2+n_{1})}{\Gamma(2+n_{1}-1)}\langle W|\widetilde{X}_{1}^{n_{1}-1}|V\rangle=\langle W|\widetilde{X}_{1}^{n_{1}}|V\rangle=\frac{\Gamma(2+n_{1})}{\Gamma(2+n_{1}-1)}\frac{\Gamma(2+n_{1}-1)}{\Gamma(2+n_{1}-2)}\langle W|\widetilde{X}_{1}^{n_{1}-2}|V\rangle\\&=
	\frac{\Gamma(2+n_{1})}{\Gamma(2+n_{1}-1)}\frac{\Gamma(2+n_{1}-1)}{\Gamma(2+n_{1}-2)}\ldots \frac{\Gamma(2-1)}{\Gamma(2)}\langle W|\widetilde{X}_{1}^{0}|V\rangle
	\\&=
	\frac{\Gamma(2+n_{1})}{\Gamma(2)}
\end{align*}
By using this result we have 
\begin{equation}
	\langle W|\prod_{i=1}^{L}\prod_{a=1}^{N}\widetilde{X}_{a}^{n_{a}^{i}}
	|V \rangle=\frac{\Gamma(2+n_{1})}{\Gamma(2)}\prod_{i=1}^{L}\prod_{a=2}^{N}\left(\lambda_{a}\right)^{n_{a}^{i}}\left(1+\sum_{j=i}^{L}n_{1}^{j}\right)^{n_{a}^{i}}
\end{equation}
The normalization constant is computed similarly as 
\begin{align*}
	Z_{L}&=\langle W|(X_{1}+\ldots+X_{N})^{L}|V\rangle=\langle W|\widetilde{X}_{1}^{L}|V|\rangle=
	\frac{\Gamma(2+L)}{\Gamma(2)}
\end{align*}
Therefore, we write the coefficients of the linear combination \eqref{toDetermineTilde} as 
\begin{equation}\label{resulEsteady}
	|\widetilde{\Psi}\rangle= \sum_{\mathbf{n}\in \Omega}\frac{\Gamma(2+\sum_{i=1}^{L}n_{1}^{i})}{\Gamma(2+L)}\prod_{i=1}^{L}\prod_{a=2}^{N}\left(\lambda_{a}\right)^{n_{a}^{i}}\left(1+\sum_{j=i}^{L}n_{1}^{j}\right)^{n_{a}^{i}}|\mathbf{n}\rangle
\end{equation}
Calling $s_{i}$ the species present at site $i$, we rewrite the basis vector 
\begin{equation}
	|n_{1}^{i},\ldots,n_{N}^{i}\rangle=|s_{i}\rangle
\end{equation}
where $s_i=\sum_{k=1}^N\delta_{n_k^i,1}$. Finally we take the tensor product of the vectors $|s_{i}\rangle$ and we obtain
\begin{equation}
	|\mathbf{s}\rangle=\bigotimes_{i=1}^{L}|s_{i}\rangle=|\mathbf{s}\rangle
\end{equation}
with $s_{i}\in\{1,\ldots,N\}$. For a fixed site $i$, because of the hard-core interaction that allows at most one particle on type different from $1$ ,at site $i$ we have 
\begin{equation}
	\prod_{a=2}^{N}\left(\lambda_{a}\right)^{n_{a}^{i}}\left(1+\sum_{j=i}^{L}n_{1}^{j}\right)^{n_{a}^{i}}=\left[\lambda_{s_{i}}\left(1+\sum_{j=i}^{L}\delta_{s_{j},1}\right)\right]^{1-\delta_{s_{i},1}}
\end{equation}
Moreover, we have 
\begin{equation}
	\frac{\Gamma(2+\sum_{i=1}^{L}n_{1}^{i})}{\Gamma(2+L)}=\frac{\Gamma\left(2+\sum_{i=1}^{L}\delta_{s_{i},1}\right)}{\Gamma\left(L+2\right)}
\end{equation}
Finally summing over all $\mathbf{s}\in \Omega$ (that is equivalent, $\forall i\in\{1,\ldots,L\}$, to summing over all the possible values of each $s_{i}\in\{1,\ldots,N\}$) we obtain \eqref{ResulsBasis} from \eqref{resulEsteady}.
\begin{flushright}
$\square$
\end{flushright}
\subsubsection{The steady state of $\widehat{H}$}
Knowing the steady state \eqref{ResulsBasis} of the Hamiltonian $\mathbb{H}$ defined in \eqref{bbHamiltonian}, we use \eqref{SteadyStateHatTilde} to retrieve the steady state of the Hamiltonian $\widehat{H}$ defined in \eqref{hatHamiltonian}. The result is the following 
\begin{equation}\label{ABS-vect}
    |\widehat{\Psi}\rangle =\sum_{s_{1},\ldots,s_{L}=1}^{N}\widehat{\Psi}(\bm{s})|\bm{s}\rangle 
\end{equation}
where 
\begin{equation}\label{ABS}
		\begin{split}
			\widehat{\Psi}(\bm{s})=\sum_{i=1}^{L}\sum_{c_{i}=0}^{1-\delta_{s_{i},1}}\frac{\Gamma(2+L-\sum_{k=1}^{L}c_{k})}{\Gamma(L+2)}\prod_{l=1}^{L}\left(\lambda_{s_{l}}\left(2+L-l-\sum_{j=l}^{L}c_{j}\right)\right)^{c_{l}}\beta_{s_{l}}^{(1-c_{l})(1-\delta_{s_{l},1})}
		\end{split}
	\end{equation} 
and where
 \begin{equation}\label{CiEquation}
	c_{i}=\begin{cases}
		1\quad \text{if}\quad s_{i}\neq 1\\
		0\quad \text{if}\quad s_{i}=1
	\end{cases}\qquad \forall i\in \{1,\ldots,N\}
\end{equation}
\paragraph{Proof of formula \eqref{ABS-vect}} As a consequence of \eqref{SteadyStateHatTilde}, we only need to show how the transformation $\mathcal{T}^{T}$ acts on the vector $|\widetilde{\Psi}\rangle$. The matrix $T^{T}$ is given by 
\begin{equation}
    T^{T}=\mathbbm{1}+\sum_{a=2}^{N}\beta_{a}e_{a1}
\end{equation}
We introduce the following notation
\begin{equation}\label{notation1}
    e_{ab}=e_{a}^{b}=\left(\delta_{a,c}\delta_{b,d}\right)_{c,d\in\{1,\ldots,N\}}\qquad \forall a,b\in \{1,\ldots,\}
\end{equation}
and 
\begin{equation}\label{notation2}
    e_{ab}\otimes e_{cd}=e_{a,c}^{b,d}=\left(\delta_{(a,b),(e,f)}\delta_{(c,d),(g,h)}\right)_{e,f,g,h\in \{1,\ldots,N\}}\qquad \forall a,b,c,d\in \{1,\ldots,N\}
    \end{equation}
    that can be generalized for $e_{s_{1},\ldots,s_{L}}^{q_{1},\ldots,q_{L}}$ with $s_{1},\ldots,s_{L},q_{1},\ldots,q_{L}\in \{1,\ldots,N\}$. With this notation we have that 
    \begin{equation}
        T^{T}=\sum_{s_{1}=1}^{N}e_{i}^{i}+Q
    \end{equation}
    where
    \begin{equation}
	Q:=(1-\delta_{s_{1},1})\sum_{s_{1}=2}^{N}\beta_{s_{1}}e_{s_{1}}^{1}
\end{equation}
We compute
\begin{equation}
    \begin{split}
        	T^{T}\otimes T^{T}&=\left(I_{N\times N}+Q\right)\otimes\left(I_{N\times N}+Q\right)
         \\&=
         I_{N\times N}\otimes I_{N\times N}+I_{N\times N}\otimes Q+Q\otimes I_{N\times N}+Q\otimes Q
    \end{split}
\end{equation}
where the addends are 
\begin{equation}
\begin{split}
	&I_{N\times N}\otimes I_{N\times N}=\sum_{s_{1}=1}^{N}e_{s_{1}}^{s_{1}}\otimes \sum_{s_{2}=1}^{N}e_{s_{2}}^{s_{2}}=\sum_{s_{1},s_{2}=1}^{N}e_{s_{1},s_{2}}^{s_{1},s_{2}}\\
	&I_{N\times N}\otimes Q=(1-\delta_{s_{2},1})\sum_{s_{1}=1}^{N}e_{s_{1}}^{s_{1}}\otimes \sum_{s_{2}=2}^{N}\beta_{s_{2}}e_{s_{2}}^{1}=(1-\delta_{s_{2},1})\sum_{s_{1}=1,s_{2}=2}^{N}e_{s_{1},s_{2}}^{s_{1},1}\beta_{s_{2}}\\
	&Q\otimes I_{N\times N}=(1-\delta_{s_{1},1})\sum_{s_{1}=2}^{N}\beta_{s_{1}}e_{s_{1}}^{1}\otimes \sum_{s_{2}=1}^{N}e_{s_{2}}^{s_{2}}=(1-\delta_{s_{1},1})\sum_{s_{1}=2,s_{2}=1}^{N}e_{s_{1},s_{2}}^{1,s_{2}}\beta_{s_{1}}\\
	&Q\otimes Q=(1-\delta_{s_{1},1})(1-\delta_{s_{2},1})\sum_{s_{1}=2}^{N}\beta_{s_{1}}e_{s_{1}}^{1}\otimes \sum_{s_{2}=2}^{N}\beta_{s_{2}}e_{s_{2}}^{1}=(1-\delta_{s_{1},1})(1-\delta_{s_{2},1})\sum_{s_{1},s_{2}=2}^{N}e_{s_{1},s_{2}}^{1,1}\beta_{s_{1}}\beta_{s_{2}}
 \end{split}
\end{equation}
The matrix $T^{T}\otimes T^{T}$ has elements $\left\{\left(T^{T}\otimes T^{T}\right)_{s_{1},s_{2}}^{q_{1},q_{2}}\right\}_{s_{1},s_{2},q_{1},q_{2}\in \{1,\ldots,N\}}$, where the subscripts $s_{1},s_{2}$ denote the raw and the superscript $q_{1},q_{2}$ the column. We fix a raw of this matrix with arbitrary index $s_{1},s_{2}$
\begin{equation}
	\left(T^{T}\otimes T^{T}\right)_{s_{1},s_{2}}=e_{s_{1},s_{2}}^{s_{1},s_{2}}+(1-\delta_{s_{2},1})\beta_{s_{2}}e_{s_{1},s_{2}}^{s_{1},1}+(1-\delta_{s_{1},1})\beta_{s_{1}}e_{s_{1},s_{2}}^{1,s_{2}}+(1-\delta_{s_{1},1})(1-\delta_{s_{2},1})\beta_{s_{1}}\beta_{s_{2}}e_{s_{1},s_{2}}^{1,1}
\end{equation}
Iterating this tensor product for the chain of length $L$, we write the raw with indices $\bm{s}=s_{1},\ldots,s_{L}$
 of the matrix $\mathcal{T}^{T}$ as 
\begin{equation}
	\begin{split}
		\mathcal{T}^{T}_{\bm{s}}&=\left(T^{T}\right)^{\otimes L}_{s_{1},\ldots,s_{L}}=\,e_{s_{1},\ldots,s_{L}}^{s_{1},\ldots,s_{L}}+\sum_{q_{1}=1}^{N}(1-\delta_{s_{q_{1}},1})e_{s_{1},\ldots,s_{q_{1}},\ldots,s_{L}}^{s_{1},\ldots,1,\ldots,s_{L}}\beta_{s_{q_{1}}}\\&+\sum_{q_{1},q_{2}=1\,:\,q_{1}\neq q_{2}}^{L}(1-\delta_{s_{q_{1}},1})(1-\delta_{s_{q_{2}},1})e_{s_{1},\ldots,s_{q_{1}},\ldots,s_{q_{2}}\ldots,s_{L}}^{s_{1},\ldots,1,\ldots,1,\ldots,s_{L}}\beta_{s_{q_{1}}}\beta_{s_{q_{2}}}
  \\&+
  \ldots+\left(\prod_{i=1}^{L}(1-\delta_{s_{i},1})\right)\prod_{i=1}^{L}\left(\beta_{s_{i}}\right)^{1-\delta_{s_{i},1}}e_{s_{1},\ldots,s_{L}}^{1,\ldots,1}
	\end{split}
\end{equation}
The product $\mathcal{T}^{T}|\widetilde{\Psi}\rangle$ gives a vector whose component are denoted by $\widehat{\Psi}(\bm{s})$ that are given by 
\begin{equation}\label{ABS_intermediate}
	\begin{split}
		\widehat{\Psi}(\mathbf{s})&=\mathcal{T}^{T}_{\bm{s}}|\widetilde{\Psi}\rangle\\&= \widetilde{\Psi}(\mathbf{s})+\sum_{q_{1}= 1}^{L}(1-\delta_{s_{q_{1}},1})\widetilde{\Psi}(s_{1},\ldots,\overbrace{1}^{s_{q_{1}}},\ldots,s_{L})\beta_{s_{q_{1}}}\\&+
		\sum_{q_{1},q_{2}= 1}^{L}(1-\delta_{s_{q_{1}},1})(1-\delta_{s_{q_{2}},1})\widetilde{\Psi}(s_{1},\ldots,\overbrace{1}^{s_{q_{1}}},\ldots,\overbrace{1}^{s_{q_{2}}},\ldots,s_{L})\beta_{k_{1}}\beta_{k_{2}}\\&+\ldots+\widetilde{\Psi}(1,\ldots,1)\left(\prod_{i=1}^{L}(1-\delta_{s_{i},1})\right)\prod_{i=1}^{L}\left(\beta_{s_{i}}\right)^{1-\delta_{s_{i},1}}
	\end{split}
\end{equation}
For a fixed configuration $\bm{s}=(s_{1},\ldots,s_{N})$, using equation \eqref{ResulsBasis} and observing that 
\begin{equation}
  \left(1+\sum_{j=i}^{L}\delta_{s_{j},1}\right) =\left(2+L-i-\sum_{j=i}^{L}(1-\delta_{s_{i},1})\right)
\end{equation}
we obtain 
    \begin{equation}
	\widetilde{\Psi}(\bm{s})=\frac{\Gamma(2+\sum_{i=1}^{L}\delta_{s_{i},1})}{\Gamma(L+2)}\prod_{i=1}^{L}\left(\lambda_{s_{i}}\left(2+L-i-\sum_{j=i}^{L}(1-\delta_{s_{i},1})\right)\right)^{1-\delta_{s_{i},1}}
\end{equation}
Because of the exponent $(1-\delta_{s_{i},1})$, the element $\widetilde{\Psi}(\bm{s})$ has, up the term $\frac{\Gamma(2+\sum_{i=1}^{L}\delta_{s_{i},1})}{\Gamma(L+2)}$, a contribution of a factor $\lambda_{s_{i}}\left(1+L-i-\sum_{j=i}^{L}(1-\delta_{s_{i},1})\right)$ if the site is non empty, and a contribution of a factor "$1$" if the site is empty, it is convenient to introduce the $c_{i}$ of defined in \eqref{CiEquation}. Then, equation  \eqref{ABS_intermediate} reads 
\begin{equation}\label{elementsABS}
	\begin{split}
		\widehat{\Psi}(\mathbf{s})=\sum_{c_{1}=0}^{1-\delta_{s_{1},1}}\ldots\sum_{c_{L}=0}^{1-\delta_{s_{L},1}}\frac{\Gamma(2+L-\sum_{k=1}^{L}c_{k})}{\Gamma(L+2)}\prod_{i=1}^{L}\left(\lambda_{s_{i}}\left(2+L-i-\sum_{j=i}^{L}c_{j}\right)\right)^{c_{i}}\left(\beta_{s_{i}}\right)^{(1-c_{i})(1-\delta_{s_{i},1})}
	\end{split}
\end{equation} 
Indeed, when $c_{i}=0$ this formula replaces the contribution in the square brackets with a $\beta_{s_{i}}$. 
Finally, considering all possible occupancy $\bm{s}$, we obtain \eqref{ABS}.
\begin{flushright}
    $\square$
\end{flushright}
\subsubsection{The steady state of $H$}
The steady state $|\Psi\rangle$ of the Hamiltonian $\mathbb{H}$ defined in \eqref{bbHamiltonian} is given by 
\begin{equation}\label{steadyStateH}
    |\Psi\rangle=\sum_{s_{1},\ldots,s_{L}=1}^{N}\Psi(\bm{s})|\bm{s}\rangle
\end{equation}
where for every $(s_{1},\ldots,s_{L})=\bm{s}$
\begin{equation}\label{steadyStateH-coeff}
    \begin{split}
        \Psi(\bm{s})&=\sum_{i=1}^{L}\sum_{c_{i}=0}^{1-\delta_{s_{i},1}}\frac{\Gamma(2+L-\sum_{k=1}^{L}c_{k})}{\Gamma(L+2)}\prod_{l=1}^{L}\left(\lambda_{s_{l}}\left(2+L-l-\sum_{j=l}^{L}c_{j}\right)\right)^{c_{l}}\beta_{s_{l}}^{(1-c_{l})(1-\delta_{s_{l},1})}
        \\&+
        \sum_{\ell=1}^{L}(-1)^{\ell}\sum_{q_{1}=1}^{L}\sum_{q_{2}=q_{1}+1}^{L}\ldots\sum_{q_{\ell}=q_{\ell-1}+1}^{L}\left(\prod_{r=1}^{\ell}\delta_{s_{q_{r}},1}\right) 
        \sum_{r=1}^{\ell}\sum_{d_{q_{r}}=2}^{N}
\\&
\sum_{i=1\,:\, i\neq q_{1},\ldots,q_{l}}^{L}\sum_{c_{i}=0}^{1-\delta_{s_{i},1}}\sum_{p=1}^{\ell}\sum_{c_{q_{p}}=0}^{1}\frac{\Gamma(2+L-\sum_{k=1}^{L}c_{k})}{\Gamma(2+L)}
  \prod_{i=1\,:\, i\neq q_{1},\ldots,q_{\ell}}^{L}\left(\lambda_{s_{i}}\left(2+L-i-\sum_{j=i}^{L}c_{j}\right)\right)^{c_{i}}
 \\ &
 \beta_{s_{i}}^{(1-c_{i})(1-\delta_{s_{i},1})}
 \prod_{t=1}^{\ell}\lambda_{d_{q_{t}}}\left(2+L-q_{t}-\sum_{j=q_{t}}^{L}c_{j}\right)^{c_{q_{t}}}\beta_{d_{q_{t}}}^{(1-c_{q_{t}})}
    \end{split}
\end{equation}
\textbf{Remark}: we observe that if no holes are present in \eqref{steadyStateH-coeff}, i.e. if $s_{i}\neq 1$ $\forall i\in \{1,\ldots,L\}$, then the second addend vanishes and, for these particular configurations, 
\begin{equation}\label{LinkABS-corr}
\Psi(\bm{s})=\widehat{\Psi}(\bm{s})
\end{equation}


\paragraph{Proof of formula \eqref{steadyStateH}} To find $|\Psi\rangle$ we will use \eqref{SteadystateHat}. For the sake of notation we call
\begin{equation}
    F=\left(V^{-1}\right)^{T}
\end{equation}
and thus 
\begin{equation}
    \mathcal{F}=F^{\otimes L}
\end{equation}
By direct computations and using the notation introduced in \eqref{notation1} and \eqref{notation2} we have 
\begin{equation}
    F=\sum_{s_{1}=1}^{N}e_{s_{1}}^{s_{1}}+\delta_{s_{1},1}(-1)\sum_{d_{1}=2}^{N}e_{1}^{d_{1}}
\end{equation}
Taking the tensor product over two sites we obtain 
\begin{equation}
\begin{split}
    F\otimes F&=\sum_{s_{1},s_{2}=1}^{N}\left(e_{s_{1},s_{2}}^{s_{1},s_{2}}+(-1)\delta_{s_{2},1}\sum_{d_{2}=2}^{N}e_{s_{1},1}^{s_{1},d_{2}}
     \right.\\ &+\left. 
    (-1)\delta_{s_{1},1}\sum_{d_{1}=2}^{N}e_{1,s_{2}}^{d_{1},s_{2}}+(-1)^{2}\delta_{s_{1},1}\delta_{s_{2},1}\sum_{d_{1},d_{2}=2}^{N}e_{1,1}^{d_{1},d_{2}}\right)
    \end{split}
\end{equation}
therefore, fixing a raw with indices $s_{1},s_{2}\in \{1,\ldots,N\}$ of the matrix $F\otimes F$ we obtain 
\begin{equation}
\begin{split}
    \left(F\otimes F\right)_{s_{1},s_{2}}&=e_{s_{1},s_{2}}^{s_{1},s_{2}}+(-1)\delta_{s_{2},1}\sum_{d_{2}=2}^{N}e_{s_{1},1}^{s_{1},d_{2}}
    \\&+
    (-1)\delta_{s_{1},1}\sum_{d_{1}=2}^{N}e_{1,s_{2}}^{d_{1},s_{2}}+(-1)^{2}\delta_{s_{1},1}\delta_{s_{2},1}\sum_{d_{1},d_{2}=2}^{N}e_{1,1}^{d_{1},d_{2}}
    \\&=
    \sum_{q_{1}=1}^{2}\prod_{r=1}^{1}\delta_{s_{q_{r}},1}\sum_{r=1}^{1}\sum_{d_{q_{r}}=2}^{N}e_{s_{i},1}^{s_{i}d_{q_{r}}}
    +
    (-1)\sum_{q_{1}=1}^{2}\sum_{q_{2}=q_{1}+1}^{2}\prod_{r=1}^{2}\delta_{s_{q_{r}},1}\sum_{r=1}^{2}\sum_{d_{q_{r}}=2}^{N}e_{s_{i},1}^{s_{i}d_{q_{r}}}
    \end{split}
\end{equation}
where $e_{s_{i},d_{q_{r}}}$ denotes the matrix $e_{s_{1},s_{2}}^{s_{1},s_{2}}$ where the subscripts $s_{q_{r}}$ have been replaced by "1" and the superscript $s_{q_{r}}$ have been replaced by $d_{q_{r}}$, with $r=1,2$. \\
We generalize this tensor product to an arbitrary number of sites $L$ obtaining the matrix $\mathcal{F}$. We write the expression of a fixed raw with indices $s_{1},\ldots,s_{L}\in \{1,\ldots,N\}$ as 
\begin{equation}
    \begin{split}
        \mathcal{F}_{s_{1},\ldots,s_{L}}&=e_{s_{1},\ldots,s_{L}}^{s_{1},\ldots,s_{L}}
        \\&
        +\sum_{\ell=1}^{L}(-1)^{\ell}\sum_{q_{1}=1}^{L}\sum_{q_{2}=q_{1}+1}^{L}\ldots\sum_{q_{\ell}=q_{\ell-1}+1}^{L}\left(\prod_{r=1}^{\ell}\delta_{s_{q_{r}},1}\right)\sum_{r=1}^{\ell}\sum_{d_{q_{r}}=2}^{N}e_{\ldots,s_{i},\ldots,\underbrace{1}_{q_{r}},\ldots}^{\ldots,s_{i},\ldots,\,d_{q_{r}}\,,\ldots}
    \end{split}
\end{equation}
where $e_{\ldots,s_{i},\ldots,\underbrace{1}_{q_{r}},\ldots}^{\ldots,s_{i},\ldots,\,d_{q_{r}}\,,\ldots}$ denotes the matrix $e_{s_{1},\ldots,s_{L}}^{s_{1},\ldots,s_{L}}$ where the subscripts $s_{q_{1}},\ldots,s_{q_{\ell}}$ have been replaced by "1" and the superscripts $s_{q_{1}},\ldots,s_{q_{\ell}}$ have been replaced by $d_{q_{1}},\ldots,d_{q_{\ell}}$. \\
Multiplying the raw with indices $s_{1},\ldots,s_{L}$ of the matrix $\mathcal{F}$ by the vector $|\widehat{\Psi}\rangle$ we obtain 
\begin{equation}
    \begin{split}
        \mathcal{F}_{s_{1},\ldots,s_{L}}|\widehat{\Psi}\rangle&=\widehat{\Psi}(s_{1},\ldots,s_{L})+ 
        \\&
        +\sum_{\ell=1}^{L}(-1)^{\ell}\sum_{q_{1}=1}^{L}\sum_{q_{2}=q_{1}+1}^{L}\ldots\sum_{q_{\ell}=q_{\ell-1}+1}^{L}\left(\prod_{r=1}^{\ell}\delta_{s_{q_{r}},1}\right)\sum_{r=1}^{\ell}\sum_{d_{q_{r}}=2}^{N}\widehat{\Psi}(\ldots,s_{i},\ldots,d_{q_{r}},\ldots)
    \end{split}
\end{equation}
where $\widehat{\Psi}(\ldots,s_{i},\ldots,d_{q_{r}},\ldots)$ denotes the component $\widehat{\Psi}(s_{1},\ldots,s_{L})$ where the occupations  $s_{q_{1}},\ldots,s_{q_{\ell}}$ have been replaced by $d_{q_{1}},\ldots,d_{q_{\ell}}$. Using \eqref{elementsABS} we have 
\begin{equation}
    \begin{split}
  &\widehat{\Psi}(\ldots,a_{i},\ldots,d_{q_{r}},\ldots)=\sum_{i=1\,:\, i\neq q_{1},\ldots,q_{l}}^{L}\sum_{c_{i}=0}^{1-\delta_{s_{i},1}}\sum_{p=1}^{\ell}\sum_{c_{q_{p}}=0}^{1}\frac{\Gamma(2+L-\sum_{k=1}^{L}c_{k})}{\Gamma(2+L)}
\\&  
\prod_{i=1\,:\, i\neq q_{1},\ldots,q_{\ell}}^{L}\left(\lambda_{s_{i}}\left(2+L-i-\sum_{j=i}^{L}c_{j}\right)\right)^{c_{i}}\beta_{s_{i}}^{(1-c_{i})(1-\delta_{s_{i},1})}
\\&
  \prod_{t=1}^{\ell}\lambda_{d_{q_{t}}}\left(2+L-q_{t}-\sum_{j=q_{t}}^{L}c_{j}\right)^{c_{q_{t}}}\beta_{d_{q_{t}}}^{(1-c_{q_{t}})}
    \end{split}
\end{equation}
indeed, in coordinates $q_{r}$ with $r=1,\ldots \ell$ the species $1$ is replaced by all other possible species from $2$ to $N$.  Finally, considering all possible occupations $\bm{s}=(s_{1},\ldots,s_{L})$ we obtain \eqref{steadyStateH}. 
\begin{flushright}
    $\square$
\end{flushright}
\subsection{Exact correlations}\label{correlation-section}
We compute in this section a formula for the stationary steady state correlations between $m$- points of the chain, each of them occupied by a non empty species, i.e. a species with label from $2$ to $N$. This is motivated by interest of understanding how actual particles of the chain are correlated among each others. Consider $m\in \{1,\ldots,L\}$ points with coordinates given by $\bm{y}=(y_{1},\ldots,y_{m})$, such that $y_{i}<y_{i+1}$ $\forall i\in \{1,\ldots,m-1\}$. In each of these points there is a particle of species $s_{i}\in \{2,\ldots,N\}$. Then, with respect to the non-equilibrium stationary distribution, the correlations among these particles are given by
\begin{equation}\label{exactCorrelations}
		\langle n^{y_{1}}_{s_{1}},\ldots,n^{y_{m}}_{s_{m}}\rangle_{\text{stat}}=\sum_{c_{1},\ldots,c_{m}=0}^{1}
			f(c_{1},\ldots,c_{m})\prod_{k=1}^{m}(\alpha_{s_{k}}-\beta_{s_{k}})^{c_{k}}\beta_{s_{k}}^{1-c_{k}}g_{k}(c_{k},\ldots,c_{m})
	\end{equation}
	where 
	\begin{equation}\label{powerCoeffSpecies}
		g_{k}(c_{k},\ldots,c_{m})=\left(L+2-y_{k}-\sum_{i=k}^{m}c_{i}\right)^{c_{k}}
	\end{equation}
	and 
	\begin{equation}\label{powerCoeffNOspec}
		f(c_{1},\ldots,c_{m})=\frac{\Gamma(L+2-\sum_{i=1}^{m}c_{i})}{\Gamma(L+2)}
	\end{equation}
 \paragraph{Proof of formula \eqref{exactCorrelations}}
 We aim to compute the correlations between the sites $\mathbf{y}=\left(y_{1},\ldots,y_{m}\right)$ occupied by the species $\mathbf{s}=\left(s_{1},\ldots,s_{m}\right)$, such that $s_{1},\ldots,s_{m}\in \{2,\ldots,N\}$ i.e. 
\begin{equation}\label{Prob-m}
	\begin{split}
		\langle n_{s_{1}}^{y_{1}},\ldots, n_{s_{m}}^{y_{m}}\rangle_{\text{stat}}&
		=\mathbb{P}_{stat}\left(n_{s_{1}}^{y_{1}}=1,\ldots,n_{s_{m}}^{y_{m}}=1\right)
	\end{split}
\end{equation}
where we denote $\mathbb{P}_{stat}$ the non-equilibrium steady state distribution. 
This probability \eqref{Prob-m} can be seen as the component \eqref{steadyStateH-coeff} of the stationary state \eqref{steadyStateH} of a chain of length $m$ with each coordinates $y_{i}$ occupied by $s_{i}\neq 1$, i.e. 
\begin{equation}
	\mathbb{P}_{stat}\left(n_{s_{1}}^{y_{1}}=1,\ldots,n_{s_{m}}^{y_{m}}=1\right)=\Psi_{\bm{y}}\left(s_{1},\ldots,s_{m}\right)
\end{equation}
where the subscript $\bm{y}$ underlines the fact that we are considering a chain with length $m$. 
As already pointed out in \eqref{LinkABS-corr}, for non-empty configurations we have:
\begin{equation}\label{Corr-mABS}
	\Psi_{\bm{y}}(s_{1},\ldots,s_{m})=\widehat{\Psi}_{\bm{y}}(s_{1},\ldots,s_{m})
\end{equation}
Thus, by applying the formula \eqref{ABS_intermediate} we have 
\begin{equation}
	\widehat{\Psi}_{\bm{y}}(s_{1},\ldots,s_{m})=\sum_{i=1}^{L}\sum_{c_{i}=0}^{1}\frac{\Gamma(2+L-\sum_{k=1}^{m}c_{k})}{\Gamma(2+L)}\prod_{k=1}^{m}\lambda_{s_{k}}^{c_{k}}\beta_{s_{k}}^{1-c_{k}}\left(2+L-y_{k}-\sum_{j=k}^{m}c_{j}\right)^{c_{k}}
\end{equation}
By using the relation \eqref{powerCoeffSpecies} and \eqref{powerCoeffNOspec} we have the equations for arbitrary correlations.
\begin{flushright}
    $\square$
\end{flushright}
\subsubsection{Explicit examples of correlations}
We give some examples of correlations applying formula \eqref{exactCorrelations}.
\paragraph{One point correlations}
The average with respect to the non equilibrium distribution of the occupation variable of the species $s_{1}\in \{2,\ldots,N\}$ at coordinate $y_{1}\in \{1,\ldots,L\}$ reads
\begin{equation}
	\langle n_{s_{1}}^{y_{1}}\rangle_{\text{stat}}=\sum_{c_{1}=0}^{1}(\alpha_{s_{1}}-\beta_{s_{1}})^{c_{1}}\beta_{s_{1}}^{1-c_{1}}g_{s_{1}}(c_{1})f(c_{1})
\end{equation}
Using \eqref{powerCoeffSpecies} and \eqref{powerCoeffNOspec} we obtain 
\begin{equation}\label{one-pts-corr}
\langle n_{s_{1}}^{y_{1}}\rangle_{\text{stat}}=\frac{(L+1-y_{1})}{L+1}(\alpha_{s_{1}}-\beta_{s_{1}})+\beta_{s_{1}}
\end{equation}
\paragraph{Two points correlations}
The two points correlations with respect the non equilibrium distribution of the occupation variable of the species $s_{1},s_{2}\in \{2,\ldots,N\}$ at coordinates $y,x\in \{1,\ldots,L\}$ with $x< y$ reads 
\begin{equation}
	\begin{split}
	\langle n_{s_{1}}^{x},n_{s_{2}}^{y}\rangle_{\text{stat}}= \sum_{c_{1}=0}^{1}\sum_{c_{2}=0}^{1}f(c_{1},c_{2})\prod_{k=1}^{2}(\alpha_{s_{k}}-\beta_{s_{k}})^{c_{k}}\beta_{s_{k}}^{1-c_{k}}g_{k}(c_{k},c_{2})
	\end{split}
\end{equation}
Using \eqref{powerCoeffSpecies} and \eqref{powerCoeffNOspec} we obtain 
\begin{equation}\label{two-pts-corr}
	\begin{split}
	\langle n_{s_{1}}^{x},n_{s_{2}}^{y}\rangle_{\text{stat}}&=\beta_{s_{1}}\beta_{s_{2}}+\frac{(L+1-y_{1})}{L+1}(\alpha_{s_{1}}-\beta_{s_{1}})\beta_{s_{2}}+\frac{(L+1-y_{2})}{L+1}(\alpha_{s_{2}}-\beta_{s_{2}})\beta_{s_{1}}\\&+\frac{(L-y_{1})(L+1-y_{2})}{L(L+1}(\alpha_{s_{1}}-\beta_{s_{1}})(\alpha_{s_{2}}-\beta_{s_{2}})
	\end{split}
\end{equation}
\paragraph{Three points correlations}
The three points correlations with respect the non equilibrium distribution of the occupation variable of the species $s_{1},s_{2},s_{3}\in \{2,\ldots,N\}$ at coordinates $y_{1},y_{2},y_{3}\in \{1,\ldots,L\}$ such that $y_{1}<y_{2}<y_{3}$
\begin{equation}
    \langle n_{s_{1}}^{y_{1}}n_{s_{2}}^{y_{2}},n_{s_{3}}^{y_{3}}\rangle_{\text{stat}}=\sum_{c_{1}=0}^{1}\sum_{c_{2}=0}^{1}\sum_{c_{3}=0}^{1}f(c_{1},c_{2},c_{3})\prod_{k=1}^{2}(\alpha_{s_{k}}-\beta_{s_{k}})^{c_{k}}\beta_{s_{k}}^{1-c_{k}}g_{k}(c_{k},\ldots,c_{3})
\end{equation}
Using \eqref{powerCoeffSpecies} and \eqref{powerCoeffNOspec} we obtain 
\begin{equation}\label{three-pts-corr}
    \begin{split}
        \langle n_{s_{1}}^{y_{1}}n_{s_{2}}^{y_{2}},n_{s_{3}}^{y_{3}}\rangle_{\text{stat}}&=\prod_{k=1}^{3}\beta_{s_{k}}+\sum_{k=1}^{3}\frac{(L-y_{k})}{(L+1)}(\alpha_{s_{k}}-\beta_{s_{k}})\prod_{q=1\,:\,q\neq k}^{3}\beta_{s_{q}}
        \\&+
        \frac{(L-y_{2})(L-y_{3}+1)}{L(L+1)}(\alpha_{s_{2}}-\beta_{s_{2}})(\alpha_{s_{3}}-\beta_{s_{3}})\beta_{s_{1}}
        \\&+
        \frac{(L-y_{1})(L-y_{3}+1)}{L(L+1)}(\alpha_{s_{1}}-\beta_{s_{1}})(\alpha_{s_{3}}-\beta_{s_{3}})\beta_{s_{2}}
        \\&+
        \frac{(L-y_{1})(L-y_{2}+1)}{L(L+1)}(\alpha_{s_{1}}-\beta_{s_{1}})(\alpha_{s_{2}}-\beta_{s_{2}})\beta_{s_{3}}
        \\&+
        \frac{(L-y_{1}-1)(L-y_{2})(L-y_{3}+1)}{L(L-1)(L+1)}\prod_{k=1}^{3}(\alpha_{s_{k}}-\beta_{s_{k}})
    \end{split}
\end{equation}
\textbf{Remark}: the one and two points correlations computed in \eqref{one-pts-corr} and \eqref{two-pts-corr} respectively are exactly the ones found in subsection 4.3 of \cite{vanicat}. 
\subsection{One points correlations of the non-integrable multi-speceis stirring process on a chain}
In subsection~\ref{subsection-exact}, for the case with geometry of one dimensional chain of length $L$ connected at the ends with two reservoris of parameters $(\alpha_{k})_{k=1,\ldots,N}$ and $(\beta_{k})_{k=1,\ldots,N}$, all the correlations have been written by \eqref{exactCorrelations} using integrability. When the maximal occupation of each site $\nu>1$, this nice property is lost. However, something about correlations can be said using duality. In this subsection we still consider the same geometry of \ref{subsection-exact}. As pointed out in section~\ref{sectionDuality}, the dual process voids the chain. Therefore, the problem of computing the average of the duality matrix with respect to the stationary distribution is reduced to computing the absorption probabilities of dual particles, as we will clarify below. In this context it is better to work agin with generators. We first write the generators for the original and the dual process and the duality matrix in the particular case of this geometry. \\
We consider the process $(\bm{n}(t))_{t\geq 0}$ on a chain of length $L$ with the end sites connected with two reservoirs. This interaction with the external environment is described by the generators 
\begin{equation}
    \mathcal{L}_{1}f(\bm{n})=\sum_{k,\ell=0}^{N}\alpha_{k}n_{\ell}^{1}\left(f(\bm{n}-\delta_{\ell}^{x}+\delta_{k}^{x})-f(\bm{n})\right)\qquad \mathcal{L}_{L}f(\bm{n})=\sum_{k,\ell=0}^{N}\beta_{k}n_{\ell}^{L}\left(f(\bm{n}-\delta_{\ell}^{x}+\delta_{k}^{x})-f(\bm{n})\right)
\end{equation}
The bulk generator is 
\begin{equation}
    \mathcal{L}_{\text{bulk}}=\sum_{x=1}^{L-1}\mathcal{L}_{x,x+1}\quad \text{where}\quad \mathcal{L}_{x,x+1}f(\bm{n})=\sum_{k,\ell=0}^{N}n_{k}^{x}n_{\ell}^{x+1}\left(f(\bm{n}-\delta_{k}^{x}+\delta_{\ell}^{x}+\delta_{k}^{x+1}-\delta_{\ell}^{x+1})-f(\bm{n})\right)
\end{equation}
This process is dual to $(\bm{\xi}(t))_{t\geq0} $ on the enlarged configuration space $\widetilde{\Omega}=N_{0}^{N-1}\times\bigtimes_{x=1}^{L} \Omega_{x}\times N_{0}^{L-1}$ with the same $\mathcal{L}_{\text{bulk}}$ and with absorbing boundaries, i.e. 
\begin{equation}
\begin{split}
        \widetilde{\mathcal{L}}_{1}f(\bm{\xi})&=\sum_{i=1}^{N}\alpha_{i}\sum_{k=2}^{N}\xi_{k}^{1}\left(f(\bm{\xi}-\delta_{k}^{1}+\delta_{k}^{0}+\delta_{k}^{1})f(\bm{\xi})\right)\\\widetilde{\mathcal{L}}_{L}f(\bm{\xi})&=\sum_{i=1}^{L}\beta_{i}\sum_{k=2}^{N}\xi_{k}^{L}\left(f(\bm{\xi}-\delta_{k}^{L}+\delta_{k}^{L+1}+\delta_{k}^{L})-f(\bm{\xi})\right)
        \end{split}
\end{equation}
where $0$ and $L+1$ are the extra-sites. The duality function on the chain is
\begin{equation}\label{dualityFunctionChain}
    D(\bm{n},\bm{\xi})=\prod_{k=2}^{N}\left(\frac{\alpha_{k}}{\sum_{i=1}\alpha_{i}}\right)^{\xi_{k}^{0}}\prod_{x=1}^{L}\prod_{k=2}^{N}\frac{\eta_{k}^{x}!}{(n_{k}^{x}-\xi_{k}^{x})!}\prod_{k=2}^{N}\left(\frac{\beta_{k}}{\sum_{i=1}\beta_{i}}\right)^{\xi_{k}^{L+1}}
\end{equation}
Calling $\mu_{stat}$ the stationary measure of the the process $(\bm{n}(t))_{t\geq 0}$ starting from a configuration $\bm{n}$, and considering a generic configuration $\bm{\xi}$ of the dual process, we write the following equalities using \eqref{DualityRelation}
\begin{equation}\label{ExptationSS}
\begin{split}
\mathbb{E}_{\mu^{stat}}\left[D(\bm{n}(t),\bm{\xi})\right]&=\lim_{t\to\infty}\mathbb{E}_{\bm{n}}\left[D(\bm{n}(t),\bm{\xi})\right]=\lim_{t\to\infty}\mathbb{E}_{\bm{\xi}}\left[D(\bm{n},\bm{\xi}(t))\right]
\\=&
\sum_{i_{1}=0}^{|\xi_{1}|}\ldots\sum_{i_{N}=0}^{|\xi_{N}|}\prod_{k=2}^{N}\left(\frac{\alpha_{k}}{\alpha_{1}+\ldots+\alpha_{N}}\right)^{i_{k}}\left(\frac{\beta_{k}}{\beta_{1}+\ldots+\beta_{N}}\right)^{|\xi_{k}|-i_{k}}a_{i_{1},\ldots,i_{k}}(\bm{\xi})
\end{split}
\end{equation}
where 
\begin{equation}\label{Pass}
    a_{i_{k}}(\bm{\xi})=\mathbb{P}\left(\xi(\infty)=\sum_{k=2}^{N}\left(i_{k}\delta_{0}+(|\xi_{k}|-i_{k})\delta_{L+1}\right)| \xi_{0}=\xi\right)
\end{equation}
where $D(\bm{n},\bm{\xi})$ is the duality function defined in \eqref{dualityFunctionChain}. Here, we denote by $|\xi_{k}|=\sum_{x=1}^{L}\xi_{k}^{x}$ and by $i_{k}$ the number of dual particles of species $k\in\{2,\ldots,N\}$ absorbed at left, thus we denote by $|\xi_{k}|-i_{k}$ the number of particles of the same type absorbed at right.\\
We observe that the duality function \eqref{dualityFunctionChain} computed in the dual configurations $\bm{\widehat{\xi}}=\delta_{s_{1}}^{y_{1}}+\ldots+\delta_{s_{m}}^{y_{m}}$ with $y_{i}<y_{i+1}$ $\forall i\in \{1,\ldots,m-1\}$ and $s_{i}\in \{2,\ldots,N\}$ $\forall i=\{2,\ldots,m\}$ is given by 
\begin{equation}
    D(\bm{n},\bm{\widehat{\xi}})= \frac{\prod_{k=1}^{m}n_{s_{k}}^{x_{k}}}{(\nu-m+1)}
\end{equation}
Therefore, up to a constant, $\mathbb{E}_{\mu^{\text{stat}}}\left[D(\bm{n},\bm{\widehat{\xi}})\right]$ gives the non-equilibrium stationary state $m-$points correlations between non-empty species. This expectation is computed by \eqref{ExptationSS} once the absorption probabilities $a_{i_{s_{k}}}(\widehat{\bm{\xi}})$. The issue is that, usually, these absorption probabilities fulfil some difference equations that are not easy to solve. For example, we report here the solution for the one points correlations. 
\paragraph{One point correlations for the non-integrable case}
We have that $\widehat{\bm{\xi}}=\delta_{s}^{x}$, then 
\begin{equation}
	\nu D(\eta,\delta_{s}^{x})=n_{s}^{x}
\end{equation}
Thus
\begin{equation}\label{OnePtsCORR}
\begin{split}
	\langle n_{s}^{x}\rangle_{\text{stat}}&=\frac{1}{\nu}\lim_{t\to\infty}\mathbb{E}_{\bm{n}}\left[D(\bm{n}(t),\delta_{k}^{x})\right]\\&=\frac{1}{\nu}\sum_{i_{s}=0}^{1}\left(\frac{\alpha_{s}}{\alpha_{1}+\ldots+\alpha_{N}}\right)^{i_{s}}\left(\frac{\beta_{s}}{\beta_{1}+\ldots+\beta_{N}}\right)^{1-i_{k}}\mathbb{P}_{\delta_{s}^{x}}\left(\xi_{\infty}=i_{s}\delta_{0}+(1-i_{s})\delta_{L+1}\right)
 \end{split}
\end{equation}
we want to determine $a^{x}_{s}(1)=\mathbb{P}_{\delta_{s}^{x}}\left(\xi_{\infty}=i_{s}\delta_{0}+(1-i_{s})\delta_{L+1}\right)$ is the probability of a random walker on the chain $\{1,\ldots,L\}$ with absorbing boundaries. This $a_{s}^{x}(1)$ fulfils the following equations 
\begin{equation}\label{gamblers-ruin}
	\begin{cases}
		\Delta_{x}^{L}a^{x}_{s}(1)=0\\
		a^{1}_{s}(1)=\frac{A}{1+A}+a^{2}_{s}(1)\frac{1}{1+A}\\
		a^{L}_{s}(1)=a^{L-1}_{s}(1)\frac{1}{1+B}
		\end{cases}
\end{equation}
where $\Delta_{x}^{L}$ is the discrete Laplace operator on the chain of length $L$ and where, for the sake of notation, we call $\alpha_{1}+\ldots+\alpha_{N}=A$ and $\beta_{1}+\ldots+\beta_{N}=B$. The equations \eqref{gamblers-ruin} can be solved.Therefore, using \eqref{OnePtsCORR} and $a^{x}_{s}(0)=1-a^{x}_{s}(1)$ we obtain 
\begin{equation}
\langle n_{s}^{x}\rangle_{\text{stat}}=\frac{1}{\nu}\frac{\alpha_{s}(LB-Bx+1)+\beta_{s}(Ax+1-A)}{ABL-AB+A+B}
\end{equation}
\textbf{Remark}: In case $A=B=1$ we have 
	\begin{equation}
\langle n_{s}^{x}\rangle_{\text{stat}}=\frac{\alpha_{s}(L-x+1)+\beta_{s}x}{L+1}
	\end{equation}
that gives back \eqref{one-pts-corr}.  \\ 
\subsection{Mapping to equilibrium?}
{\color{red}
check also Appendix A in
\cite{Alcaraz:1992zc} }

\cite{Sklyanin:1988yz}
\subsubsection{Quantum inverse scattering method}
We can apply the \textit{quantum inverse scattering method} to this process. We introduce the quantum space as $V=V_{1}\otimes\ldots\otimes V_{j}\otimes\ldots\otimes V_{L}$ and the auxiliary space $V_{0}$. In this specific case $V_{j}=\mathbb{C}^{N}$ $\forall j=0,1,\ldots ,L$. We introduce the following matrices acting on the spaces with indices $a,b=0,1,\ldots,L$
\begin{equation}\label{Rmatrix}
R_{ab}(x)=\frac{(P_{ab}+Ix)}{x+1}
\end{equation}
\begin{equation}\label{KmatrixHAT}
\widehat{K}_{a}(x)=I+\frac{2x}{(x+1)}B_{L}
\end{equation}
\begin{equation}\label{Kmatrix}
K_{a}(x)=\left((2x+N)B_{1}+(x+N)I\right)\frac{1}{(x+1)^{2}(2x+N)}
\end{equation}
The above matrices satisfy the following equations
\begin{itemize}
\item \textit{Yang-Baxter-Equation}
\begin{equation}
R_{ab}(x-y)R_{ac}(x-z)R_{bc}(y-x)=R_{bc}(y-z)R_{ac}(x-z)R_{ab}(x-y)
\end{equation}
\item \textit{Boundary-Yang-Baxter-Equation}
\begin{equation}
R_{ab}(x-y)\widehat{K}_{a}(x)R_{ab}(x+y)\widehat{K}_{2}(y)=\widehat{K}_{b}(y)R_{ab}(x+y)\widehat{K}_{a}(x)R_{ab}(x-y)
\end{equation}
\item \textit{Dual Boundary-Yang-Baxter-Equation}
\begin{equation}
K_{b}(y) R_{ab}(-x-y-N) K_{a}(x)R_{ab}(y-x)=R_{ab}(y-x)K_{a}(x)R_{ab}(-x-y-N)K_{b}(y)
\end{equation}
\end{itemize}
We define the following \textit{transfer matrix}
\begin{equation}
	T(x)=tr_{0}\left(K_{0}(x)U_{0}(x)\right)
\end{equation}
where we used the double row monodromy
\begin{equation}
	U_{0}(x)=M_{0}(x)\widehat{K}(x)\widehat{M}_{0}(x)
\end{equation}
where 
\begin{equation}
	M_{0}(x)=R_{01}(x)\ldots R_{0L}(x)\qquad \widehat{M}_{0}=R_{0L}(x)\ldots R_{01}(x)
\end{equation}
This trasfer matrix has the following properties
\begin{equation}
	\left[T(x),T(y)\right]=0\qquad \frac{\partial}{\partial x}\log (T(0))=C_{1}H+C_{2}I
\end{equation}
with $C_{1},C_{2}>0$. \\
We write $U_{0}(x)$ in the canonical basis of $V_{0}$ and we rename the elements, each of them acting on the quantum space $V$, has follows
\begin{equation}
	U_{0}(x)=\begin{pmatrix}
		A(x)&B_{2}(x)&\ldots&B_{N}(x)\\
		C_{2}(x)&D_{22}(x)&\ldots&D_{2N}(x)\\
		\vdots&\vdots&\ddots&\vdots\\
		C_{N}(x)&D_{N2}(x)&\ldots&D_{NN}(x)
	\end{pmatrix}
\end{equation}
In order to carry out the mapping of non equilibrium onto equilibrium we should consider the K-matrices after the transformation \eqref{Esse}, i.e. when the left boundary is lower triangular a end the right one is diagonal. We properly rescale them and we obtain
\begin{equation}
	\widehat{K}_{0}^{d}(x)=(1+x)\left(I+\frac{2x}{x+1}\widetilde{B}_{L}\right)=\begin{pmatrix}
		1+x&0&\ldots&0\\
		0&1-x&\ldots&0\\
		\vdots&\vdots&\ddots&\vdots\\
		0&0&\ldots&1-x
	\end{pmatrix}
\end{equation}
\begin{equation}
	K_{0}^{\Delta}=\left((2x+N)\widetilde{B}_{1}+(x+N)I\right)\frac{(1+x)^{2}(2x+N)}{(x+1)^{2}(2x+N)}=\begin{pmatrix}
		N+x&0&\ldots&0\\
		(2x+N)(\alpha_{2}-\beta_{2})&-x&\ldots&0\\
		\vdots&\vdots&\ddots&\vdots\\
		(2x+N)(\alpha_{N}-\beta_{N})&0&\ldots&-x
	\end{pmatrix}
\end{equation}
Then we obtain the correspondent transfer matrix as
\begin{equation}
	T^{\Delta}(x)=tr_{0}\left(K_{0}^{\Delta}(x)U_{0}^{d}(x)\right)
\end{equation}
where
\begin{equation}
	U_{0}^{d}(x)=M_{0}(x)\widehat{K}_{0}^{d}(x)\widehat{M}_{0}(x)
\end{equation}
Then we have
\begin{equation}
	\begin{split}
	T^{\Delta}(x)&=tr_{0}\left(\begin{pmatrix}
		N+x&0&\ldots&0\\
		(2x+N)(\alpha_{2}-\beta_{2})&-x&\ldots&0\\
		\vdots&\vdots&\ddots&\vdots\\
		(2x+N)(\alpha_{N}-\beta_{N})&0&\ldots&-x
	\end{pmatrix}\begin{pmatrix}
	A(x)&B_{2}(x)&\ldots&B_{N}(x)\\
	C_{2}(x)&D_{22}(x)&\ldots&D_{2N}(x)\\
	\vdots&\vdots&\ddots&\vdots\\
	C_{N}(x)&D_{N2}(x)&\ldots&D_{NN}(x)
\end{pmatrix}\right)\\&=
(N+x)A(x)-\sum_{i=2}^{N}xD_{ii}(x)+\sum_{i=2}^{N}(2x+N)(\alpha_{i}-\beta_{i})B_{i}(x)\\&=
Q_{2L+2}x^{2L+2}+Q_{2L+1}x^{2L+1}+Q_{2L}x^{2L}+\ldots
\end{split}
\end{equation}
We call these $Q_{k}$, with $k=0,\ldots,2L+2$, the \textit{charges}. \\
We can also introduce 
\begin{equation}
	T_{0}(x)=(N+x)A(x)-\sum_{i=2}^{N}xD_{ii}(x)\quad\text{such that}\quad T_{\Delta}(x)=T_{0}(x)+\sum_{i=2}^{N}(2x+N)(\alpha_{i}-\beta_{i})B_{i}(x)
\end{equation}For large $x$, we compute them by expanding $U_{0}^{d}(x)$ in a polynomial: 
\begin{equation}
	A(x)=\sum_{k=0}^{2L+1}A^{k}x^{k}\quad D_{ii}(x)=\sum_{k=0}^{2L+1}D_{ii}^{k}x^{k}\quad B_{i}(x)=\sum_{k=0}^{2L-1}B_{i}^{k}x^{k}
\end{equation}
The coefficients that we need are the following:
\begin{equation}
	A(x)=A^{2L+1}x^{2L+1}+A^{2L}x^{2L+1}+A^{2L-1}x^{2L-1}+\ldots
\end{equation}
where .
\begin{equation}
	A^{2L+1}=I^{tot}
\end{equation}
\begin{equation}
	A^{2L}=I^{tot}+2e_{11}^{tot}
\end{equation}
\begin{equation}
	\begin{split}
		A^{2L-1}=&2e_{11}^{tot}+\left(e_{11}^{tot}-\sum_{j=2}^{N}e_{jj}^{tot}\right)+2\sum_{a\neq b}e_{11}^{[a]}e_{11}^{[b]}
	\end{split}
\end{equation}
We observe that $\forall i,j=1,2,\ldots,N$
\begin{equation}
	\sum_{a,b=1}^{L}e_{i,j}^{[a]}e_{j,i}^{[b]}=e_{ij}^{tot}e_{ji}^{tot}-e_{ii}^{tot}
\end{equation}
then 
\begin{equation}
	\boxed{A^{2L-1}=e_{11}^{tot}\left(2e_{11}^{tot}+I^{tot}\right)-\sum_{j=2}^{N}e_{jj}^{tot}}
\end{equation}
While, $\forall i=2,\ldots,N$


\begin{equation}
	D_{ii}(x)=D_{ii}^{2L+1}x^{2L+1}+D_{ii}^{2L}x^{2L}+D_{ii}^{2L-1}x^{2L-1}+\ldots
\end{equation}
where the coefficients
\begin{equation}\label{a2lp}
	D_{ii}^{2L+1}=-I^{tot}
\end{equation}\begin{equation}\label{a2l}
	D_{ii}^{2L}=I^{tot}-2e_{ii}^{tot}
\end{equation}
\begin{equation}\label{a2lm1}
	\begin{split}
		D_{ii}^{2L-1}&=2e_{ii}^{tot}+\left(e_{11}^{tot}-\sum_{j=2}^{N}e_{jj}^{tot}\right)-2\sum_{a\neq b}e_{ii}^{[a]}e_{ii}^{[b]}-2\sum_{a\neq b}\sum_{j=2\,:\,j\neq i}^{N}e_{ij}^{[a]}e_{ji}^{[b]}\\
		&=e_{11}^{tot}-\sum_{j=2}^{N}e_{jj}^{tot}+2e_{ii}^{tot}-2\sum_{a\neq b}\sum_{j=2}^{N}e_{ij}^{[a]}e_{ji}^{[b]}\\
	%&=e_{11}^{tot}-\sum_{j=2}^{N}e_{jj}^{tot}+2e_{ii}^{tot}\\
		%&=2e_{11}^{tot}+2e_{ii}^{tot}-C_1
		&=e_{11}^{tot}-\sum_{j=2}^{N}e_{jj}^{tot}+2e_{ii}^{tot}-2\sum_{a\neq b}\sum_{j=2}^{N}e_{ij}^{[a]}e_{ji}^{[b]}\\ 
	\end{split}
\end{equation}

We rewrite
\begin{equation}
	\begin{split}
	D_{ii}^{2L-1}&=2e_{ii}^{tot}+\left(e_{11}^{tot}-\sum_{j=2}^{N}e_{jj}^{tot}\right)-2\sum_{j=2}^{N}e_{ij}^{tot}e_{ji}^{tot}+2(N-1)e_{ii}^{tot}
\end{split}
\end{equation}
then
\begin{equation}
\begin{split}
\boxed{D_{ii}^{2L-1}=	2Ne_{ii}^{tot}+\left(e_{11}^{tot}-\sum_{j=2}^{N}e_{jj}^{tot}\right)-2\sum_{j=2}^{N}e_{ij}^{tot}e_{ji}^{tot}}
	\end{split}
\end{equation}
\textbf{Observation}: it seems to me that, although not diagonal, the $D_{ii}^{2L-1}$ does not chaing the exitation of the chain since we have $e_{ij}^{tot}e_{ji}^{tot}$!\\
%The following \textit{lower diagonal} elements of $U_{0}(x)$, $\forall i=2,\ldots,L$ and for $j=1$
%\begin{equation}
%	C_{i}(x)=C_{i}^{2L-1}x^{2L-1}+\ldots
%\end{equation}
%where the coefficient is
%\begin{equation}\label{c}
%	C_{i}^{2L-1}=2e_{1i}^{tot}+2\sum_{a< b}\left(e_{1i}^{[a]}e_{11}^{[b]}-e_{1i}^{[a]}e_{ii}^{[b]}-\sum_{k=2\,:\,k\neq j}^{N}e_{1k}^{[a]}e_{ki}^{[b]}\right)
%\end{equation}
Moreover, $\forall j=2,\ldots,L$ and for $i=1$
\begin{equation}
	B_{j}(x)=B_{j}^{2L-1}x^{2L-1}+\ldots
\end{equation}
where the coefficient is 
\begin{equation}\label{b}
\begin{split}
	B_{j}^{2L-1}&=2e_{j1}^{tot}+2\sum_{a< b}\left(e_{j1}^{[a]}e_{11}^{[b]}-e_{j1}^{[a]}e_{jj}^{[b]}-\sum_{k=2\,:\,k\neq j}^{N}e_{k1}^{[a]}e_{jk}^{[b]}\right)\\
	&=2e_{j1}^{tot}+2\sum_{a< b}\left(e_{j1}^{[a]}e_{11}^{[b]}-\sum_{k=2}^{N}e_{k1}^{[a]}e_{jk}^{[b]}\right)
\end{split}
\end{equation}
then
\begin{equation}
\begin{split}
	\boxed{B_{j}^{2L-1}=2e_{j1}^{tot}+
	2\sum_{a,b=1\,:\, a<b}^{L}e_{j1}^{[a]}e_{11}^{[b]}-2\sum_{a,b=1\,:\, a>b}^{L}\sum_{k=2}^{N}e_{jk}^{[a]}e_{k1}^{[b]}}
\end{split}
\end{equation}
\textbf{Remark}: it is easy to see that, when restricted to the case of 2 species,  \eqref{a2lp},\eqref{a2l},\eqref{a2lm1} correspond to the one compute in \cite{frassek2020eigenstates}. \\ \\
The interesting charges (we only look for the first non trivial one) take the following form  



\begin{align}
	Q_{2L+2}&=A^{2L+1}-\sum_{i=2}^{N}D_{ii}^{2L+1}\\
	Q_{2L+1}&=NA^{2L+1}+A^{2L}-\sum_{i=2}^{N}D_{ii}^{2L}\\
	Q_{2L\phantom{+}}&=NA^{2L}+A^{2L-1}-\sum_{i=2}^{N}D_{ii}^{2L-1}+\sum_{i=2}^{N}2(\alpha_{i}-\beta_{i})B_{i}^{2L-1}
\end{align}
The charges $Q_{2L+2}$ and $Q_{2L+1}$ are proportional to the identity matrix, while $Q_{2L}$ contains non diagonal matrices and the boundary parameters $\alpha_{i},\beta_{i}$ $\forall i=2,\ldots,N$.\\
It is easy to show that 
\begin{equation}
	\left[Q_{2L},\widetilde{H}\right]=Q_{2L}\widetilde{H}-\widetilde{H}Q_{2L}=0
\end{equation}
The commutation with the Hamiltonian $\widetilde{H}$ for $Q_{2L+2}$ and $Q_{2L+1}$ is trivial. \\
We introduce the following:
\begin{equation}
Q_{2L}^{0}:=NA^{2L}+A^{2L-1}-\sum_{i=2}^{N}D_{ii}^{2L-1}
\end{equation}
that is 
\begin{equation}
\begin{split}
Q_{2L}^{0}&=N\left(I^{tot}+2e_{11}^{tot}\right)+e_{11}^{tot}(2e_{11}^{tot}+I^{tot})-\sum_{j=2}^{N}e_{jj}^{tot}
\\&
+\sum_{i=2}^{N}\left(	2Ne_{ii}^{tot}+\left(e_{11}^{tot}-\sum_{j=2}^{N}e_{jj}^{tot}\right)-2\sum_{j=2}^{N}e_{ij}^{tot}e_{ji}^{tot}\right)
\end{split}
\end{equation}
it has the property that commute with the equilibrium Hamiltonian
\begin{equation}
\widetilde{H}_{0}:=\widetilde{B}_{1}^{0}+\mathcal{H}+\widetilde{B}_{L}
\end{equation}
where 
\begin{equation}
\widetilde{B}_{1}^{0}=\begin{pmatrix}
0&0&\ldots&0\\
0&-1&\ldots&0\\
\vdots&\vdots&\ddots&\vdots\\
0&0&\ldots&-1
\end{pmatrix}
\end{equation}
i.e.
\begin{equation}
\left[Q_{2}^{0},\widetilde{H}_{0}\right]=Q_{2}^{0}\widetilde{H}_{0}-\widetilde{H}_{0}Q_{2}^{0}=0
\end{equation}
We also introduce 
\begin{equation}
Q_{2}^{\Delta}:=\sum_{i=2}^{N}2(\alpha_{i}-\beta_{i})B_{i}^{2L-1}
\end{equation}
These two matrices $Q_{2}^{0}$ and $Q_{2}^{\Delta}$ are such that
\begin{equation}
Q_{2}=Q_{2}^{0}+Q_{2}^{\Delta}
\end{equation}
\subsection{Eigenvector for the diagonal-triangular chain}
In \cite{belliard2011nested}  the eigenvectors and eigenvalues hae been computed for the $gl(N)$ spin chain with diagonal boundaries, therfore we know 
\begin{equation}
T_{0}(x)|\Psi_{m_{2},\ldots,m_{N}}^{0}\rangle =\Lambda_{m_{2},\ldots,m_{N}}(x)|\Psi_{m_{2},\ldots,m_{N}}^{0}\rangle
\end{equation}
where $m_{2},\ldots,m_{N}$ denote the total occupation along the chain of species $2,\ldots,N$. Let us notice that $m_{1}=L-\sum_{k=2}^{N}m_{k}$.\\
For the sake of notation, we restrict to the case with $N=3$ and we denote by $\Delta_{2}=\alpha_{2}-\beta_{2}$ and $\Delta_{3}=\alpha_{3}-\beta_{3}$ and we further assume that $\Delta_{2}=\Delta_{3}=\Delta$. 
We make the following ansatz for the eigevectors of the diagonal-triangular spin chain. Consider the whole chain of dimension $L$. Then, $m=m_{2}+m_{3}$ sites have particles of a type $2$ or $3$ and the remaining $L-m$ are empty. For fixed $m$ there is a one to one correspondence between empty and occupeid sites. Therefore, our ansatz is that $\forall m\in \{0,\ldots,L\}$ 
\begin{equation}\label{ansatzEigenvectorDelta}
|\Psi_{L-m}^{\Delta}\rangle=\mathbbm{1}_{\{m_{2}+m_{3}=m\}}\sum_{k_{2}=0}^{m_{2}}\sum_{k_{3}=0}^{m_{3}}\Delta^{k_{2}+k_{2}}G_{m_{2},m_{3},\epsilon}^{k_{2},k_{3}}(x)|\Psi_{L-m}^{0}\rangle|_{\epsilon=0}
\end{equation}
We fix arbitrary $m,m_{2},m_{3}\in \{0,\ldots,L\}$ such that $m=m_{2}+m_{3}$.  We replace the above eigenvectors in 
\begin{equation}\label{eigDELTA}
T_{\Delta}(x)|\Psi_{L-m}^{\Delta}\rangle=\Lambda_{m_{2},m_{3}}(x)|\Psi_{L-m}^{\Delta}\rangle
\end{equation}
and we obtain
\begin{align*}
&\sum_{k_{2}=0}^{m_{2}}\sum_{k_{3}=0}^{m_{3}}\Delta^{k_{2}+k_{3}}T_{0}(x)G_{m_{2},m_{3},\epsilon}^{k_{2},k_{3}}(x)|\Psi_{L-m}^{0}\rangle|_{\epsilon=0}
\\+&
\sum_{k_{2}=0}^{m_{2}}\sum_{k_{3}=0}^{m_{3}}\Delta^{k_{2}+k_{3}+1}\left(B_{2}(x)+B_{3}(x)\right)G_{m_{2},m_{3},\epsilon}^{k_{2},k_{3}}(x)|\Psi_{L-m}^{0}\rangle|_{\epsilon=0}
\\=&
\sum_{k_{2}=0}^{m_{2}}\sum_{k_{3}=0}^{m_{3}}\Delta^{k_{2}+k_{3}}\Lambda_{m_{2},m_{3}}(x)G_{m_{2},m_{3}\epsilon}^{k_{2},k_{3}}(x)|\Psi_{L-m}^{0}\rangle|_{\epsilon=0}
\end{align*}
We assume that $G_{m_{2},m_{3},\epsilon}^{k_{2},k_{3}}(x)=G_{m_{2},m_{3},\epsilon}^{k_{2}+k_{3}}(x)$, i.e. it only depends on the sum of $k_{2}$ and $k_{3}$. We make an shift of summation indexes such that 
\begin{equation}
k_{2}+k_{3}\to k_{2}+k_{3}+1
\end{equation}
then we obtain
\begin{align*}
&\sum_{k_{2}=0}^{m_{2}}\sum_{k_{3}=0}^{m_{3}}\Delta^{k_{2}+k_{3}}T_{0}(x)G_{m_{2},m_{3},\epsilon}^{k_{2},k_{3}}(x) |\Psi_{L-m}^{0}\rangle|_{\epsilon=0}
\\+&
\sum_{k_{2}=0}^{m_{2}}\sum_{k_{3}=0}^{m_{3}}\mathbbm{1}_{\{k_{2}+k_{3}\neq 0\}}\Delta^{k_{2}+k_{3}}\left(B_{2}(x)+B_{3}(x)\right)G_{m_{2},m_{3},\epsilon}^{k_{2}+k_{3}-1}(x)|\Psi_{L-m}^{0}\rangle|_{\epsilon=0}
\\+&
\Delta^{m_{2}+m_{3}+1}\left(B_{2}(x)+B_{3}(x)\right)G_{m_{2},m_{3},\epsilon}^{m_{2}+m_{3}}(x)|\Psi_{L-m}^{0}\rangle|_{\epsilon=0}
\\=&
\sum_{k_{2}=0}^{m_{2}}\sum_{k_{3}=0}^{m_{3}}\Delta^{k_{2}+k_{3}}\Lambda_{m_{2},m_{3}}(x)G_{m_{2},m_{3}\epsilon}^{k_{2},k_{3}}(x)|\Psi_{L-m}^{0}\rangle|_{\epsilon=0}
\end{align*}
By equalizing the powers of $\Delta$ we obtain the following condition that must be fulfilled to have that \eqref{ansatzEigenvectorDelta} are indeed the eigenvectors for the triangular-diagonal spin chain:
\begin{enumerate}
\item for $k_{2}=k_{3}=0$
\begin{equation}\label{condition1}
T_{0}(x)|\Psi_{m_{2},m_{3}}^{0}|\Psi_{L-m}^{0}\rangle|_{\epsilon=0}=\Lambda_{m_{2},m_{3}}(x)|\Psi_{L-m}^{0}\rangle|_{\epsilon=0}
\end{equation}
\item for $k_{2}+k_{3}\neq 0$
\begin{equation}\label{condition2}
\begin{split}
\left[\left(\Lambda_{m_{2},m_{3}}(x)+\epsilon\right)I-T_{0}(x)\right]^{-1}&G_{m_{2},m_{3},\epsilon}^{k_{2}+k_{3}}(x)|\Psi_{L-m}^{0}\rangle|_{\epsilon=0}\\&=\left(B_{2}(x)+B_{3}(x)\right)G_{m_{2},m_{3},\epsilon}^{k_{2},k_{3}-1}(x)|\Psi_{L-m}^{0}\rangle|_{\epsilon=0}
\end{split}
\end{equation}
\item finally
\begin{equation}\label{condition3}
\left(B_{2}(x)+B_{3}(x)\right)G_{m_{2},m_{3},\epsilon}^{m_{2},m_{3}}(x)|\Psi_{L-m}^{0}\rangle|_{\epsilon=0}=0
\end{equation}
\end{enumerate}
By chosing 
\begin{equation}\label{G-solution}
G_{m_{2},m_{3},\epsilon}^{k_{2}+k_{3}}(x)=\left\{\left[\left(\Lambda_{m_{2},m_{3}}(x)+\epsilon\right)I-T_{0}^{x}\right]^{-1}\left(B_{2}(x)+B_{3}(x)\right)\right\}^{k_{2}+k_{3}}
\end{equation}
conditions \eqref{condition1} and \eqref{condition2} are easilly satisfied. To fulfill condition \eqref{condition3} we observe that, under the condition that $m=m_{2}+m_{3}$ we have that the action of \eqref{G-solution} on $|\Psi_{L-m}^{0}\rangle|_{\epsilon=0}$ leads to a completely occupied state (with both species $2$ and $3$). Therefore, applying to it the rising operator$\left(B_{2}(x)+B_{3}(x)\right)$ we obtain zero.


{\color{blue}We make the following ansatz for the eigenvectors of the chain with left triangular and right diagonal boundaries. For the sake of notation, we restrict to the case with $N=3$ and we denote by $\Delta_{2}=\alpha_{2}-\beta_{2}$ and $\Delta_{3}=\alpha_{3}-\beta_{3}$.
\begin{equation}
|\Psi_{m_{2},m_{3}}^{\Delta}\rangle=\sum_{k_{2}=0}^{m_{2}}\sum_{k_{3}=0}^{m_{3}}\Delta_{2}^{k_{2}}\Delta_{3}^{k_{3}}G_{m_{2},m_{3},\epsilon}^{k_{2},k_{3}}(x)|\Psi_{m_{2},m_{3}}^{0}\rangle|_{\epsilon=0}
\end{equation}

We replace the above eigenvectors in 
\begin{equation}
T_{\Delta}(x)|\Psi_{m_{2},m_{3}}^{\Delta}\rangle=\Lambda_{m_{2},m_{3}}(x)|\Psi_{m_{2},m_{3}}^{\Delta}\rangle
\end{equation}
and we obtain
\begin{align*}
&\sum_{k_{2}=0}^{m_{2}}\sum_{k_{3}=0}^{m_{3}}\Delta_{2}^{k_{2}}\Delta_{3}^{k_{3}}T_{0}(x)G_{m_{2},m_{3},\epsilon}^{k_{2},k_{3}}(x)|\Psi_{m_{2},m_{3}}^{0}\rangle+\sum_{k_{2}=0}^{m_{2}}\sum_{k_{3}=0}^{m_{3}}\Delta_{2}^{k_{2}+1}\Delta_{3}^{k_{3}}B_{2}(x)G_{m_{2},m_{3},\epsilon}^{k_{2},k_{3}}(x)|\Psi_{m_{2},m_{3}}^{0}\rangle
\\&
+\sum_{k_{2}=0}^{m_{2}}\sum_{k_{3}=0}^{m_{3}}\Delta_{2}^{k_{2}}\Delta_{3}^{k_{3}+1}B_{3}(x)G_{m_{2},m_{3},\epsilon}^{k_{2},k_{3}}(x)|\Psi_{m_{2},m_{3}}^{0}\rangle
\\&=
\sum_{k_{2}=0}^{m_{2}}\sum_{k_{3}=0}^{m_{3}}\Delta_{2}^{k_{2}}\Delta_{3}^{k_{3}}\Lambda_{m_{1},m_{2}}(x)G_{m_{2},m_{3},\epsilon}^{k_{2},k_{3}}(x)|\Psi_{m_{2},m_{3}}^{0}\rangle
\end{align*}
after a change of summation indexes we obtain
\begin{align*}
&\sum_{k_{2}=0}^{m_{2}}\sum_{k_{3}=0}^{m_{3}}\Delta_{2}^{k_{2}}\Delta_{3}^{k_{3}}T_{0}(x)G_{m_{2},m_{3},\epsilon}^{k_{2},k_{3}}(x)|\Psi_{m_{2},m_{3}}^{0}\rangle
\\+&
\sum_{k_{2}=1}^{m_{2}}\sum_{k_{3}=0}^{m_{3}}\Delta_{2}^{k_{2}}\Delta_{3}^{k_{3}}B_{2}(x)G_{m_{2},m_{3},\epsilon}^{k_{2}-1,k_{3}}(x)|\Psi_{m_{2},m_{3}}^{0}\rangle+\Delta_{2}^{m_{2}+1}\sum_{k_{3}=0}^{m_{3}}\Delta_{2}^{k_{2}}B_{2}(x)G_{m_{2},m_{3},\epsilon}^{m_{2},k_{3}}(x)|\Psi_{m_{2},m_{3}}^{0}\rangle
\\+&
\sum_{k_{2}=0}^{m_{2}}\sum_{k_{3}=1}^{m_{3}}\Delta_{2}^{k_{2}}\Delta_{3}^{k_{3}}B_{3}(x)G_{m_{2},m_{3},\epsilon}^{k_{2},k_{3}-1}(x)|\Psi_{m_{2},m_{3}}^{0}\rangle+\Delta_{3}^{m_{3}+1}\sum_{k_{2}=0}^{m_{2}}\Delta_{2}^{k_{2}}B_{3}(x)G_{m_{2},m_{3},\epsilon}^{k_{2},m_{3}}(x)|\Psi_{m_{2},m_{3}}^{0}\rangle
\\=&
\sum_{k_{2}=0}^{m_{2}}\sum_{k_{3}=0}^{m_{3}}\Delta_{2}^{k_{2}}\Delta_{3}^{k_{3}}\Lambda_{m_{1},m_{2}}(x)G_{m_{2},m_{3},\epsilon}^{k_{2},k_{3}}(x)|\Psi_{m_{2},m_{3}}^{0}\rangle
\end{align*}
we impose the powers of $\Delta_{2}\Delta_{3}$ to be equal. We obtain the following conditions 
\begin{enumerate}
\item $k_{2}=0$ and $k_{3}=0$
\begin{equation}\label{cond1}
T_{0}(x)G_{m_{2},m_{3},\epsilon}^{0,0}(x)|\Psi_{m_{1},m_{2}}^{0}\rangle=\Lambda_{m_{2},m_{3}}(x)G_{m_{2},m_{3},\epsilon}^{0,0}(x)|\Psi_{m_{1},m_{2}}^{0}\rangle
\end{equation}
\item $k_{2}=m_{2}+1$ and $k_{3}\in \{0,\ldots,m_{3}\}$
\begin{equation}\label{cond2}
\begin{split}
B_{2}(x)G_{m_{2},m_{3},\epsilon}^{m_{2},k_{3}}(x)|\Psi_{m_{2},m_{3}}^{0}\rangle=0
\end{split}
\end{equation}
\item $k_{2}\in\{0,\ldots,m_{2}\}$ and $k_{3}=m_{3}+1$
\begin{equation}\label{cond3}
\begin{split}
B_{3}(x)G_{m_{2},m_{3},\epsilon}^{k_{2},m_{3}}(x)|\Psi_{m_{2},m_{3}}^{0}\rangle=0
\end{split}
\end{equation}
\item $k_{2}\in \{1,\ldots,m_{2}\}$ and $k_{3}\in\{1,\ldots,m_{3}\}$
\begin{equation}\label{cond4}
\begin{split}
&\left(T_{0}(x)G_{m_{2},m_{3},\epsilon}^{k_{2},k_{3}}(x)+B_{2}(x)G_{m_{2},m_{3},\epsilon}^{k_{2}-1,k_{3}}(x)+B_{3}(x)G_{m_{2},m_{3},\epsilon}^{k_{2},k_{3}-1}(x)\right)|\Psi_{m_{2},m_{3}}^{0}\rangle
\\=&
\Lambda_{m_{2},m_{3}}(x)G_{m_{2},m_{3},\epsilon}^{k_{2},k_{3}}(x)|\Psi_{m_{2},m_{3}}^{0}\rangle
\end{split}
\end{equation} 
\item $k_{2}=0$ and $k_{3}\in\{1,\ldots,m_{3}\}$
\begin{equation}\label{cond5}
\begin{split}
&\left(T_{0}(x)G_{m_{2},m_{3},\epsilon}^{0,k_{3}}(x)+B_{3}(x)G_{m_{2},m_{3},\epsilon}^{0,k_{3}-1}(x)\right)|\Psi_{m_{2},m_{3}}^{0}\rangle
=
\Lambda_{m_{2},m_{3}}(x)G_{m_{2},m_{3},\epsilon}^{0,k_{3}}(x)|\Psi_{m_{2},m_{3}}^{0}\rangle
\end{split}
\end{equation}
\item $k_{2}\in\{1,\ldots,m_{2}\}$ and $k_{3}=0$
\begin{equation}\label{cond6}
\begin{split}
&\left(T_{0}(x)G_{m_{2},m_{3},\epsilon}^{k_{2},0}(x)+B_{2}(x)G_{m_{2},m_{3},\epsilon}^{k_{2}-1,0}(x)\right)|\Psi_{m_{2},m_{3}}^{0}\rangle
=
\Lambda_{m_{2},m_{3}}(x)G_{m_{2},m_{3},\epsilon}^{k_{2},0}(x)|\Psi_{m_{2},m_{3}}^{0}\rangle
\end{split}
\end{equation}
\end{enumerate}
To find the function $G_{m_{2},m_{3},\epsilon}^{k_{2},k_{3}}(x)$ we must fulfil the conditions \eqref{cond1},\eqref{cond2},\eqref{cond3},\eqref{cond4}, \eqref{cond5},\eqref{cond6}.
{\color{blue}This second property can be proved by using a lower weight representation }
% \end{itemize}
We prove the ansatz:
\begin{align*}
	T_{\Delta}(x)|\Psi_{m_{2},\ldots,m_{N}}^{\Delta}\rangle&=T_{0}(x)|\Psi_{m_{2},\ldots,m_{N}}^{0}\rangle
	\\&+
	\sum_{k_{2}=1}^{m_{2}}\ldots\sum_{k_{N}=0}^{m_{N}}\Delta_{2}^{k_{2}}\ldots \Delta_{N}^{k_{N}}T_{0}(x)G_{m_{2},\ldots,m_{N},\epsilon}^{k_{2},\ldots,k_{N}}(x)|\Psi_{m_{2},\ldots,m_{N}}^{0}\rangle|_{\epsilon=0}
	\\&+\ldots+	\sum_{k_{2}=0}^{m_{2}}\ldots\sum_{k_{N}=1}^{m_{N}}\Delta_{2}^{k_{2}}\ldots \Delta_{N}^{k_{N}}T_{0}(x)G_{m_{2},\ldots,m_{N},\epsilon}^{k_{2},\ldots,k_{N}}(x)|\Psi_{m_{2},\ldots,m_{N}}^{0}\rangle|_{\epsilon=0} 
	\\&+\sum_{k_{2}=0}^{m_{2}}\ldots\sum_{k_{N}=0}^{m_{N}}\Delta_{2}^{k_{2}+1}\ldots \Delta_{N}^{k_{N}}(2x+N)B_{2}(x)(x)G_{m_{2},\ldots,m_{N},\epsilon}^{k_{2},\ldots,k_{N}}(x)|\Psi_{m_{2},\ldots,m_{N}}^{0}\rangle|_{\epsilon=0}
	\\&+\ldots +
	\sum_{k_{2}=0}^{m_{2}}\ldots\sum_{k_{N}=0}^{m_{N}}\Delta_{2}^{k_{2}}\ldots \Delta_{N}^{k_{N}+1}(2x+N)B_{N}(x)(x)G_{m_{2},\ldots,m_{N},\epsilon}^{k_{2},\ldots,k_{N}}(x)|\Psi_{m_{2},\ldots,m_{N}}^{0}\rangle|_{\epsilon=0}
\end{align*}
we replace $k_{i}\to k_{i}+1$ $\forall i\in\{2,\ldots,N\}$ in the proper summation. Then,
\begin{align*}
	T_{\Delta}(x)|\Psi_{m_{2},\ldots,m_{N}}^{\Delta}\rangle&=T_{0}(x)|\Psi_{m_{2},\ldots,m_{N}}^{0}\rangle
	\\&+
	\sum_{k_{2}=1}^{m_{2}}\ldots\sum_{k_{N}=0}^{m_{N}}\Delta_{2}^{k_{2}}\ldots \Delta_{N}^{k_{N}}T_{0}(x)G_{m_{2},\ldots,m_{N},\epsilon}^{k_{2},\ldots,k_{N}}(x)|\Psi_{m_{2},\ldots,m_{N}}^{0}\rangle|_{\epsilon=0}
	\\&+\ldots+	\sum_{k_{2}=0}^{m_{2}}\ldots\sum_{k_{N}=1}^{m_{N}}\Delta_{2}^{k_{2}}\ldots \Delta_{N}^{k_{N}}T_{0}(x)G_{m_{2},\ldots,m_{N},\epsilon}^{k_{2},\ldots,k_{N}}(x)|\Psi_{m_{2},\ldots,m_{N}}^{0}\rangle|_{\epsilon=0}
	\\&+\sum_{k_{2}=1}^{m_{2}}\ldots\sum_{k_{N}=0}^{m_{N}}\Delta_{2}^{k_{2}}\ldots \Delta_{N}^{k_{N}}(2x+N)B_{2}(x)(x)G_{m_{2},\ldots,m_{N},\epsilon}^{k_{2}-1,\ldots,k_{N}}(x)|\Psi_{m_{2},\ldots,m_{N}}^{0}\rangle|_{\epsilon=0}
	\\&+\ldots
	\sum_{k_{2}=0}^{m_{2}}\ldots\sum_{k_{N}=1}^{m_{N}}\Delta_{2}^{k_{2}}\ldots \Delta_{N}^{k_{N}}(2x+N)B_{N}(x)(x)G_{m_{2},\ldots,m_{N},\epsilon}^{k_{2},\ldots,k_{N}-1}(x)|\Psi_{m_{2},\ldots,m_{N}}^{0}\rangle|_{\epsilon=0}
	\\&+
	\Delta_{2}^{m_{2}+1}\sum_{k_{3}=0}^{m_{3}}\ldots\sum_{k_{N}=0}^{m_{N}}\Delta_{3}^{k_{3}}\Delta_{N}^{k_{N}}(2x+N)B_{2}(x)G_{m_{2},\ldots,m_{N},\epsilon}^{m_{2},\ldots,k_{N}}(x)|\Psi_{m_{2},\ldots,m_{N}}^{0}\rangle|_{\epsilon=0}
	\\&+\ldots +
	\Delta_{N}^{m_{N}+1}\sum_{k_{2}=0}^{m_{3}}\ldots\sum_{k_{N-1}=0}^{m_{N-1}}\Delta_{3}^{k_{2}}\Delta_{N-1}^{k_{N-1}}(2x+N)B_{2}(x)G_{m_{2},\ldots,m_{N},\epsilon}^{k_{2},\ldots,m_{N}}(x)|\Psi_{m_{2},\ldots,m_{N}}^{0}\rangle|_{\epsilon=0}
\end{align*}
We use prperties \eqref{SecondProperty} and \eqref{FirstProperty}:
\begin{align*}
		T_{\Delta}(x)|\Psi_{m_{2},\ldots,m_{N}}^{\Delta}\rangle&=T_{0}(x)|\Psi_{m_{2},\ldots,m_{N}}^{0}\rangle
	\\&+
	\sum_{k_{2}=1}^{m_{2}}\ldots\sum_{k_{N}=0}^{m_{N}}\Delta_{2}^{k_{2}}\ldots \Delta_{N}^{k_{N}}T_{0}(x)G_{m_{2},\ldots,m_{N},\epsilon}^{k_{2},\ldots,k_{N}}(x)|\Psi_{m_{2},\ldots,m_{N}}^{0}\rangle|_{\epsilon=0}
	\\&+\ldots+	\sum_{k_{2}=0}^{m_{2}}\ldots\sum_{k_{N}=1}^{m_{N}}\Delta_{2}^{k_{2}}\ldots \Delta_{N}^{k_{N}}T_{0}(x)G_{m_{2},\ldots,m_{N},\epsilon}^{k_{2},\ldots,k_{N}}(x)|\Psi_{m_{2},\ldots,m_{N}}^{0}\rangle|_{\epsilon=0}
	\\&+\sum_{k_{2}=1}^{m_{2}}\ldots\sum_{k_{N}=0}^{m_{N}}\Delta_{2}^{k_{2}}\ldots \Delta_{N}^{k_{N}}\left[\left(\Lambda_{m_{2},m_{N}}(x)+\epsilon\right)I-T_{0}(x)\right]G_{m_{2},\ldots,m_{N},\epsilon}^{k_{2},\ldots,k_{N}}(x)|\Psi_{m_{2},\ldots,m_{N}}^{0}\rangle|_{\epsilon=0}
	\\&+\ldots +
	\sum_{k_{2}=0}^{m_{2}}\ldots\sum_{k_{N}=1}^{m_{N}}\Delta_{2}^{k_{2}}\ldots \Delta_{N}^{k_{N}}\left[\left(\Lambda_{m_{2},m_{N}}(x)+\epsilon\right)I-T_{0}(x)\right]G_{m_{2},\ldots,m_{N},\epsilon}^{k_{2},\ldots,k_{N}}(x)|\Psi_{m_{2},\ldots,m_{N}}^{0}\rangle|_{\epsilon=0}
\end{align*}
then
\begin{align*}
T_{\Delta}(x)|\Psi_{m_{2},\ldots,m_{N}}^{\Delta}\rangle&=
		\Lambda_{m_{2},\ldots,m_{N}}(x)|\Psi_{m_{2},\ldots,m_{N}}^{0}\rangle
	\\&+
	\sum_{k_{2}=1}^{m_{2}}\ldots\sum_{k_{N}=0}^{m_{N}}\Delta_{2}^{k_{2}}\ldots \Delta_{N}^{k_{N}}\left(\Lambda_{m_{2},m_{N}}(x)+\epsilon\right)(x)G_{m_{2},\ldots,m_{N},\epsilon}^{k_{2},\ldots,k_{N}}(x)|\Psi_{m_{2},\ldots,m_{N}}^{0}\rangle|_{\epsilon=0}
	\\&+\ldots+	\sum_{k_{2}=0}^{m_{2}}\ldots\sum_{k_{N}=1}^{m_{N}}\Delta_{2}^{k_{2}}\ldots \Delta_{N}^{k_{N}}\left(\Lambda_{m_{2},m_{N}}(x)+\epsilon\right)G_{m_{2},\ldots,m_{N},\epsilon}^{k_{2},\ldots,k_{N}}(x)|\Psi_{m_{2},\ldots,m_{N}}^{0}\rangle|_{\epsilon=0}
	\\&=
	\Lambda_{m_{2},\ldots,m_{N}}(x)|\Psi_{m_{2},\ldots,m_{N}}^{\Delta}\rangle
\end{align*}
}
For $i\neq 1$ we have
\begin{equation}
\begin{split}
 \sum_{i,j=2}^N [E_{ij}E_{ji},E_{k1}]&=\sum_{i,j=2}^N E_{ij} [E_{ji},E_{k1}]+\sum_{i,j=2}^N  [E_{ij},E_{k1}]E_{ji}=\sum_{j=2}^N( E_{kj} E_{j1}+ E_{j1} E_{kj})
\\&=\sum_{j=2}^N( 2E_{kj} E_{j1}+ [E_{j1}, E_{kj}])=2\sum_{j=2}^N E_{kj} E_{j1}-(N-1)E_{k1}
 \end{split}
\end{equation} 

\subsection{Bethe ansatz solution}

Describe eigenvalues of $T_0$ following 
\cite{Belliard2} 


\section{Multi-species stirring process with reaction}
\subsection{Duality for a reaction diffusion process}
In \cite{casini2022uphill}, a two-species hard-core reaction diffusion process has been introduced. The feature of this process is that its action on the occupation variable of any species gives a difference-differential equation where both discrete Laplacians and a linear reaction term are present (multiplied for proper constants). In this subsection we write a generalization to $N\in \mathbb{N}$ species with arbitrary maximal occupation per site, again called $\nu\in\mathbb{N}_{0}$. The generator reads as follows
\begin{equation}\label{RDGenerator}
    \mathcal{L}^{rd}=\sum_{x,y\in\mathcal{E}}\omega_{x,y}\mathcal{L}_{x,y}^{rd}+\sum_{x\in V}\Gamma_{x}\mathcal{L}_{x}
\end{equation}
where
\begin{equation}
    \mathcal{L}_{x,y}^{rd}=\sigma_{11}\mathcal{L}_{x,y}+\sigma_{12}\sum_{i=1}^{N}\mathcal{L}_{x,y}^{i}+(\Upsilon-2\nu\sigma_{12})\mathcal{L}_{x,y}^{m}
\end{equation}
with $\mathcal{L}_{x,y}$ and $\mathcal{L}_{x}$ are the generators the generator \eqref{edgeGenerator} and \eqref{siteGenerator} respectively and, for any function $f:\Omega\to \mathbb{R}$
\begin{equation}
    \mathcal{L}_{x,u}^{i}f(\bm{n})=\sum_{k,l=1}^{N}n_{k}^{x}n_{l}^{y}\left(f(\bm{n}-\delta_{k}^{x}+\delta_{g_{i}(l)}^{x}+\delta_{g_{i}(k)}^{y}-\delta_{l}^{y})-f(\bm{n})\right)
\end{equation}
with the mapping 
\begin{equation}
    g_{i}(k)=\begin{cases}
    k+i \quad &\text{if}\quad k+i\leq N\\
    k+i-N\quad &\text{if}\quad k+i>N\\
    0\quad &\text{if}\quad k=0
\end{cases}
\end{equation}
and where
\begin{equation}
    \mathcal{L}_{x,y}^{m}f(\bm{n})=\sum_{l,k=0}^{N}n_{k}^{x}\left(f(\bm{n}-\delta_{k}^{x}+\delta_{l}^{x})-f(\bm{n})\right)
\end{equation}
the action on the occupation variable $n_{\alpha}^{x}$ of particle type $\alpha\in \{2,\ldots,N\}$ at site $x\in V $
is the following
\begin{equation}\label{actionGraphRD}
\begin{split}
    \mathcal{L}_{x,y}^{rd}n_{\alpha}^{x}&=\nu \sigma_{11}(n_{\alpha}^{y}-n_{\alpha}^{x})+\nu\sigma_{12}\sum_{i=1}^{N}(n_{g_{i}(k)}^{y}-n_{\alpha}^{x})+(\Upsilon-2\nu\sigma_{12})\sum_{l=1}^{N}(n_{l}^{x}-n_{\alpha}^{x})
\end{split}
\end{equation}
indeed, 
\begin{equation}
    \begin{split}
\mathcal{L}_{x,y}^{i}n_{\alpha}^{x}&=\sum_{k,l=1}^{N}n_{k}^{x}n_{l}^{y}\left(n_{\alpha}^{x}-\delta_{k}^{x}+\delta_{g_{i}(l)^{x}}+\delta_{g_{i}(x)}^{y}-\delta_{l}^{y}-n_{\alpha}^{x}\right)=\sum_{k=1}^{N}n_{k}^{x}n_{q}^{y}-\sum_{c=1}^{N}n_{\alpha}^{x}n_{c}^{y}
\\&=
\sum_{k=2}^{N}n_{k}^{x}n_{q}^{y}-\left(\nu-\sum_{i=2}^{N}n_{i}^{x}\right)n_{q}^{y}-\sum_{c=2}^{N}n_{\alpha}^{x}n_{c}^{y}-n_{\alpha}^{x}\left(\nu-\sum_{i=2}^{N}n_{i}^{y}\right)
\\&=
\nu(n_{q}^{y}-n_{\alpha}^{x})
    \end{split}
\end{equation}
where we call $q\in \{1,\ldots,N\}$ such that $g_{i}(q)=\alpha$ and $c\in \{1,\ldots,N\}$ such that $g_{i}(c)\neq \alpha$. Summing over all indices $i\in\{1,\ldots,N\}$ we obtain \eqref{actionGraphRD}.\\
Therefore, considering \eqref{actionGraphRD} on the geometry of a chain of finite length $L\in \mathbb{N}$, the action of the generator \eqref{RDGenerator} on the occupation variable $n_{\alpha}^{x}$ gives
\begin{equation}\label{DifferenceEquation}
\mathcal{L}^{rd}n_{\alpha}^{x}=\nu\sigma_{11}\Delta_{L}n_{\alpha}^{x}+\nu\sigma_{12}\sum_{l=2\,:\,l\neq k}^{N}\Delta_{L}n_{l}^{x}+\Upsilon\sum_{l=1}^{N}(n_{l}^{x}-n_{\alpha}^{x})
\end{equation}
where with the symbol $\Delta_{N}$ we denote the discrete Laplacian operator on the chain, i.e. $\Delta_{L}n_{k}^{x}=n_{k}^{x+1}+n_{k}^{x-1}-2n_{k}^{x}$, $\forall x\in \{2,\ldots,L-1\}$ and $\forall k\in \{2,\ldots,N\}$. In the spirit of \cite{casini2022uphill} we can write he the Master's equation for the evolution of the average density occupation, obtaining a system of $N-1$ difference differential equation where the discrete Laplacian of all species are coupled and linear reaction term are present. \\ 
The process defined by generator \eqref{RDGenerator} has reversible product measure when all the reservoirs have the same parameter for each site and each species, i.e. $\alpha_{k}^{x}=\alpha$, $\forall k\in \{1,\ldots,N\}$ and $x\in V$. This measure is given by
\begin{equation}
    \Lambda^{rev}=\bigotimes_{x\in V}\Lambda_{x}^{rev}
\end{equation}
where the marginals are
\begin{equation}
    \Lambda_{x}^{rev}\sim \text{Multinomial}(\nu,p,\ldots,p)
\end{equation}
with $p=\frac{1}{N}$. This can be verified by imposing the detailed balance condition.\\
This generator $\mathcal{L}^{rd}$ can be described by the Hamiltonian $H^{rd}$ that is defined through the basis elements \eqref{actionE} of the $gl(N)$ Lie algebra as 
\begin{equation}
    H^{rd}=\sum_{x,y\in \mathcal{E}}\omega_{x,y}\left(\sigma_{11}H_{x,y}+\sigma_{12}\sum_{i=1}^{N}H_{x,y}^{i}+(\Upsilon-2\sigma_{12})H_{x,y}^{m}\right)+\sum_{x\in V}\Gamma_{x}H_{x}
\end{equation}
where $H_{x,y}$ and $H_{x}$ are \eqref{edgeHamiltonian} and \eqref{siteHamiltonian} respectively, 
\begin{equation}
    H_{x,y}^{i}=\sum_{k,l=0}^{N}\left(E_{g_{i}(l)k}^{x}\otimes E_{g_{i}(k)l}^{y}-E_{kk}^{x}\otimes E_{ll}^{y}\right)
\end{equation}
and 
\begin{equation}
    H_{x,y}^{m}=\sum_{k,l}^{N}\left(E_{lk}^{x}\otimes \mathbbm{1}^{y}-E_{kk}^{x}\otimes \mathbbm{1}^{y}\right)
\end{equation}
For this model, the duality relation is
\begin{equation}\label{DualityRelationRD}
    (H^{rd})^{T}D=D\widetilde{H}^{rd}
\end{equation}
where the duality matrix $D$ is again \eqref{dualityMatrix}. Following subsection~\ref{subsectionDualityProof}, to prove \eqref{DualityRelationRD} we only need tho show that the operator $S_{x,y}$ defined in \eqref{symmetry} commutes with $\sum_{i=1}^{N}H_{x,y}^{i}$ and with $H_{x,y}^{m}$. This result will follow from the commutator between $\sum_{i=1}^{N}H_{x,y}^{i}$ and $\sum_{a=2}^{N}(E_{a1}^{x})\otimes \sum_{b=2}^{N}E_{b,1}^{x}$ and from the commutator between  and $\sum_{a=2}^{N}(E_{a1}^{x})\otimes \sum_{b=2}^{N}E_{b,1}^{x}$. Using \eqref{eq:comgl}, the bilinearity and associativity of the Kronecker product and the bilinearity of the brackets we obtain 
\begin{align*}
    &\left[\sum_{a=2}^{N}E_{a1}^{x}\otimes \sum_{b=2}^{N}E_{b1}^{x},\;\sum_{i=1}^{N}\sum_{k,l=1}^{N}\left(E_{g_{i}(l)k}^{x}\otimes E_{g_{i}(k)l}^{y}-E_{kk}^{x}\otimes E_{ll}^{y}\right)\right]
    \\=&\sum_{a,b=2}^{N}\sum_{i=1}^{N}\sum_{k,l=1}^{N}\left\{\left(E_{ak}^{x}\delta_{g_{i}(l)0}-E_{g_{i}(l)0}^{x}\delta_{ak}\right)\otimes \left(E_{bl}^{y}\delta_{g_{i}(k)0}-E_{g_{i}(k)0}^{y}\delta_{bk}\right)-\left(E_{ak}^{x}\delta_{k0}-E_{k0}^{x}\delta_{ak}\right)\otimes \left(E_{bl}^{y}\delta_{l0}-E_{l0}^{y}\delta_{bl}\right)\right\}
    \\=&
    \sum_{a,b=2}^{N}\sum_{i=1}^{N}\sum_{k,l=1}^{N}\left(E_{ak}^{x}\left(\delta_{g_{i}(l)0}-\delta_{k0}\right)-\left(E_{g_{i}(l)0}^{x}-E_{k0}^{x}\right)\delta_{ak}\right)\otimes \left(E_{bl}^{y}\left(\delta_{g_{i}(k)0}-\delta_{l0}\right)-\left(E_{g_{i}(k)0}^{y}-E_{l0}^{y}\right)\delta_{bl}\right)
    \\=&\;0
\end{align*}
 Again, using \eqref{eq:comgl} and by a similar argument, we obtain
\begin{equation}
\left[\sum_{a=2}^{N}E_{a1}^{x}\otimes \sum_{b=2}^{N}E_{b1}^{x},\;\sum_{i=1}^{N}\sum_{k,l=1}^{N}\left(E_{lk}^{x}\otimes \mathbbm{1}^{y}-E_{kk}^{x}\otimes \mathbbm{1}^{y}\right)\right]=0
\end{equation}
Therefore, arguing similarly to subsection~\ref{subsectionDualityProof}, \eqref{DualityRelationRD} can be proved. 




\section{Conclusion}




\bibliography{ref}
\bibliographystyle{unsrt}
\end{document}


\appendix
\section{Oscillator}

\textbf{Remark}: The basis element the we have defined above, can also be thought as proper compositions of the rising and lowering operators of th Heisenberg Lie algebra. Indeed, \begin{equation}\label{creationOperators}
	\begin{cases}
	\oad_{a}|n_{1},\ldots,n_{a},\ldots,n_{N}\rangle = |n_{1},\ldots,n_{a}+1,\ldots,n_{N}\rangle\\
	\oa_{a}|n_{1},\ldots,n_{a},\ldots,n_{N}\rangle =n_{a}|n_{1},\ldots,n_{a}-1,\ldots,n_{N}\rangle
	\end{cases}
\end{equation}
with commutaion relations
\begin{equation}
	[\oad_{a},\oa_{b}]=-\mathbbm{1}\,\delta_{ab}
\end{equation}
The basis elements of $gl(N)$ can be written as
\begin{equation}
		e_{ab}=\oad_{a}\oa_{b}\qquad \forall a,b=1,\ldots N
\end{equation}  
\section{Representation}
The approach we use to find duality relations is the Lie algebraic one. The main idea is to describe the infinitesimal generator (or better, its transposed, i.e. the Hamiltonian operator) via a proper combination of the basis of a certain Lie algebra. Then, duality will be found by symmetry argument.\\
We chose the $gl(N)$ Lie algebra. This is compact. The abstract basis elements are denoted by $E_{a,b}$ with $a,b\in \{1,\ldots,N\}$ that obeys to the Lie bracket relations given by
\begin{equation}
\left[E_{ab},E_{cd}\right]=E_{ab}E_{cd}-E_{cd}E_{ab}=E_{ad}\delta_{bc}-E_{cb}\delta_{ad}\qquad \forall a,b\in \{1,\ldots,N\}
\end{equation}
We define the the following representation $(\mathcal{C}^{D},\Pi)$, where $D$ is the dimension of the representation and $\Pi:gl(N):\to \mathcal{M}_{D\times D}$ is an homeomorfism between the Lie algebra and the space $\mathcal{M}_{D\times D}$ of $D\times D$ matrices with complex coefficients. For every elements $E_{ab}$ of the abstract Lie algebra, it associates $\Pi(E_{ab})=e_{ab}$ Since this Lie algebra is compact and semi-simple, we can characterize every irreducible representation by defining its own \textit{highest weight state}. We recall that a weight is a vector $\mu=(\mu_{0},\ldots,\mu_{N})$ with $\mu_{i}\in\mathbb{N}_{0}$ such that there exist a state $|\mu\rangle$ with the property $e_{ii}|\mu\rangle=\mu_{i}|\mu\rangle$ $\forall i=0,\ldots,N$ (where $e_{ii}$ is the Cartan element). Let's consider two weights $\mu$,$\nu$, then we say that $\mu$ is positive if the first non zero component is positive. Then, we say that $\mu>\nu$ if $\mu-\nu>0$. Thus, we can define the "highest weight" of a representation the $\mu_{0}$ that is bigger then all the other weights. As already pointed out, every irreducible representation of $gl(N)$ is in a one to one correspondence with its heights weight. That means that we can characterize the representations via the highest weight. A more formal definitions: $\mu^{hws}=(\mu_{1},\ldots,\mu_{N})$ with $\mu_{i}\geq \mu_{i+1}$ is the heights weight of the representation if 
\begin{equation}
	e_{ii}|\mu^{hws}\rangle=\mu_{i}^{hws}|\mu^{hws}\rangle\qquad \text{and}\qquad e_{ij}|\mu^{hws}\rangle=0\quad \forall i<j
\end{equation}
The representation that we choose is characterized by the heighest weight state $|\mu^{hws}\rangle$ such that 
\begin{equation}
	e_{11}|\mu^{hws}\rangle=2j|\mu^{hws}\rangle\qquad \mu^{hws}=(2j,0,\ldots,0)
\end{equation}
The dimension of this representation is given by the combination of $N$ objects in $2j$ positions with repetition:
\begin{equation}
	D= \frac{(N+2j-1)!}{(2j)!(N-1)!}
\end{equation}
For example, in the case of spin $1/2$ (hard-core exclusion) with $N=2$ species we have dim$=\frac{(2+1-1)!}{1!(2-1)!}=2$ and in case of spin $1$, dim$=\frac{(3+2-1)!}{(2)!(2)!}=6$.\\
The basis element of the vector space of the representation $\mathbb{C}^{D}$ are given by 
\begin{equation}
    |n_{1},\ldots,n_{N}\rangle\qquad \forall
    n=(n_{1},\ldots,n_{N})
\end{equation}
$\forall n=(n_{1},\ldots,n_{N})\in \chi$ where $\chi:=\left\{n\;:\;\sum_{k=1}^{N}n_{k}=2j\right\}$ (let us notice that $\Omega=\xi^{|V|}$. This vector can be mapped in a one-to-one correspondence with the canonical basis of $\mathbb{C}^{D}$. By consequence, this basis vector are orthonormal, i.e. 
\begin{equation}
    \langle q_{1},\ldots,q_{N}|_{x}|t_{1},\ldots,t_{N}\rangle_{x}=\prod_{k=1}^{N}\delta_{q_{k},t_{k}}
\end{equation}

The algebra $gl(N)$ is isomorphic to $su(N)$ and we have chosen a $2j$ spin representation of it. It is also possible to see that this representation can be written as a direct sum of representation of $su(2)$, each with representation $2j$ and each associated with a positive root. The generator of this $su(2)$ sub-algebras can be written by:  $J^{+}=e_{ab}$, $J^{-}=e_{ba}$, $J^{0}=1/2(e_{bb}-e_{aa})$ $\forall a,b=1,\ldots,N$. \\
The explicit actions of the basis element of the algebra on the vectors of the space $\mathcal{C}^{D}$ are the following, for non Cartan and Cartan elements respectively: 
\begin{equation}
	\begin{cases}
		e_{ab}|n_{1},\ldots,n_{a},\ldots,n_{b},\ldots,n_{N}\rangle =n_{b}|n_{1},\ldots,n_{a}+1,\ldots,n_{b}-1,\ldots,n_{N}\rangle\\
		e_{aa}|n_{1},\ldots,n_{a},\ldots,n_{b},\ldots,n_{N}\rangle = n_{a} |n_{1},\ldots,n_{a},\ldots,n_{b},\ldots,n_{N}\rangle
	\end{cases}\qquad \forall a,b\in\{1,\ldots,N\}
\end{equation}
In matrix form $\forall a,b\in \{1,\ldots,N\}$
\begin{equation}\label{matrixFormulaE}
	\begin{split}
	&e_{ab}=\sum_{n\in\chi}n_{b}|n_{1},\ldots,n_{a}+1,\ldots,n_{b}-1,\ldots n_{N}\rangle \langle n_{1},\ldots,n_{N}|
	\end{split}
\end{equation}
\textbf{Remark}: in this algebraic setting it is more convenient to use the \textit{Bra-Ket} notation for the states of the process. The mapping with the vectors written in \eqref{stateSpace} is the following
\begin{equation}
(n_{1}^{x},\ldots,n_{N}^{x})^{T}=|n_{1},\ldots,n_{N}\rangle_{x}\qquad(n_{1}^{x},\ldots,n_{N}^{x})=\langle n_{1},\ldots,n_{N}|_{x}
\end{equation}
\begin{flushright}
    $\square$
\end{flushright}
This Lie algebra has second Casimir element given by 
\begin{equation}
    C_{2}=\sum_{a,b=1}^{N}e_{ab}e_{ba}
\end{equation}
This element is diagonal, and its action on a generic state is the following:
\begin{equation}
	C_{2}=2j(2j+N)\mathbbm{1}
\end{equation}
where $\mathbbm{1}$ is the $D\times D$ dimensional identity matrix. \\
By introducing the \textit{coproduct} operator \begin{equation}\begin{split}\Delta:(\mathbb{C}^{D},\Pi)&\to(\mathbb{C}^{D}\otimes \mathbb{C}^{D},\Pi\otimes \Pi) \\
 e_{ab}&\to e_{ab}\otimes \mathbbm{1}+\mathbbm{1}\otimes e_{ab}
\end{split}
\end{equation}
then we can compute 
\begin{equation}\label{coproductC2}
	\begin{split}
	\Delta(C_{2}&)=\sum_{a,b=0}^{N}\Delta(e_{ab})\Delta(e_{ba})
	=
	\sum_{a,b=0}^{N}\left(\mathbbm{1}\otimes e_{ab}+e_{ab}\otimes \mathbbm{1}\right)\left(\mathbbm{1}\otimes e_{ba}+e_{ba}\otimes \mathbbm{1}\right)
	\\&=
	\sum_{a,b=0}^{N}\left(\mathbbm{1}\otimes e_{ab}e_{ba}+e_{ba}\otimes e_{ab}+e_{ab}\otimes e_{ba}+e_{ab}e_{ba}\otimes \mathbbm{1}\right)
	\\&=
	2\left(\sum_{a,b=0}^{N}e_{ab}\otimes e_{ba}\right)+\mathbbm{1}\otimes C_{2}+C_{2}\otimes\mathbbm{1}
	\end{split}
\end{equation}
We now define the equivalent of \eqref{stateSpace} in the vector notation
 \begin{equation}
	\Omega:=\left\{|n_{1},\ldots,n_{N}\rangle \;:\;\sum_{k=1}^{N}n_{k}=2j\right\}^{\otimes|V|}
	\end{equation}
where we denote 
\begin{equation}
|n\rangle=\bigotimes_{x\in V}	\left(|n_{1},\ldots,n_{N}\rangle_{x}\right)\in \Omega
\end{equation}
The Hamiltonian operator can be written in term of the basis element of the Lie algebra as

\begin{equation}
	\begin{split}
		H=\sum_{x,y\in \mathcal{E}}\omega_{x,y}H_{x,y}+\sum_{x\in V}\Gamma_{x}H_{x}
	\end{split}
\end{equation}
where the edge Hamiltonian is
\begin{equation}
H_{x,y}=\sum_{k,l=1}^{N}\left(e_{kl}\otimes e_{lk}-e_{ll}\otimes e_{kk}\right)
 \end{equation}
 and where the site Hamiltonian is
 \begin{equation}
H_{x}=\sum_{k,l=1}^{N}\alpha_{k}^{x}\left(e_{k,l}-e_{ll}\right)
\end{equation}
Moreover, by the \eqref{coproductC2} and by computing the explicit action of the diagonal part, we can write this Hamiltonian in function of the coproduct of the second Casimir
\begin{equation}
	H=\sum_{x,y\in \mathcal{E}}\omega_{x,y}\left\{\frac{1}{2}\Delta(C_{2})-2j(4j+N)\frac{1}{2}\mathbbm{1}\otimes\mathbbm{1}\right\}+\sum_{x\in V}\Gamma_{x}\sum_{k,l=1}^{N}\alpha_{k}^{x}\left(e_{k,l}-2j\mathbbm{1}\right)
\end{equation}
