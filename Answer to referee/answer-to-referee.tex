\documentclass[10pt]{article}
\usepackage[utf8]{inputenc}
\usepackage{mathtools,amsmath,amssymb}
\usepackage{graphicx}
\usepackage{lmodern}
\usepackage[T1]{fontenc}
\usepackage[pagebackref=false]{hyperref}
\usepackage{bbm}
\usepackage{bm}
\usepackage{appendix}
\usepackage{comment}
\usepackage{mathtools}
\usepackage{array}
\usepackage{hyperref}
\usepackage{xcolor}
\numberwithin{equation}{section}
\numberwithin{equation}{subsection}
\usepackage[pagewise]{lineno}
\usepackage{xspace}

\newcommand{\fra}[1]{\textcolor[rgb]{0,0,1}{#1}}





\usepackage[margin=2.5cm]{geometry} 
\title{Answer to the referee
%: \textit{Duality for the multispecies stirring process with open boundaries}
}
	\begin{document}
	%	\maketitle
		
		\section*{Answer to general comments}
		All the page and equation cited in the answers do refer to the new file. 
		\begin{enumerate}
			\item \textit{P3, line 25: Could you provide more detail on the ”choice of the boundary reservoirs”? Is this selection
				the most general case that allows for the determination of an absorbing dual process?}\\ \\
				We expanded the sentence in the introduction to explain our choice of the boundary generators. One could in principle consider other choices but then this would imply that dual particles could possibly change their color before getting absorbed at the boundaries. In view of the application of duality to the computation of the non-equilibrium steady state we are interested in, we limit ourselves to the choice that avoids this color-changing.
				\item \textit{P4, line 33: The rates $\alpha_{A}$ are strictly positive, what happens when they are zero? Does the corresponding duality degenerates to nontrivial duality for closed system?} \\ \\
					When the rates $\alpha_{A}$ are zero, then the generator of the process in (2.2.5) is zero and therefore we get indeed for the closed system the expected duality in product form.
					\item \textit{P14, line 20: In my understanding, the Hadamard formula is a special case of the Baker–Campbell–Hausdorff
						formula, which was first utilized for duality in the paper arXiv:2209.03531. The authors should consider
						citing that paper as well.}\\ \\
						A citation has been added on page 14.
						\item \textit{P20, Section 4.2: “In fact, such simplification can always be achieved if there exists an absorbing dual
							process as established in Section 3”: It looks like this method is general, could you be more precise
							about the conditions for simplification and maybe analogues for (4.2.1)? Additionally, I find equations
							(4.2.2) to (4.2.3) distracting in this context. It’s better to keep the introduction as simple as possible.
							In fact, I find this section redundant with the steps outlined on page 22. Please consider rewriting
							these two parts together.}\\ \\ 	As the computations in section 4 are rather lengthy and technical, we believe it is useful to keep a small introductory subsection where we explain the main ideas and show that the matrix product ansatz simplies after the similarity transformations.
							We considered the option of moving section 4.2 at the end of 4.1, but we believe it is useful to explain the strategy of the proof right at the beginning of the section.							
		\end{enumerate}
		\section*{Minor issues}
		\begin{enumerate}
			\item Done.
			\item Done.
			\item Done.
			\item Done.
			\item Done.
			\item We fixed it. $N$ is the number of species. 
			\item We deleted line 52.
			\item Done.
			\item Done.
			\item The reason is that when $\nu=1$ the we have the fundamental (defining) representation of the $gl(N)$ Lie algebra. We added a footnote at page 8. %Therefore, the matrices $E_{AB}$ coincide with the elementary matrices $e_{AB}$ such that $(e_{AB})_{CD}=\delta_{AC}\delta_{BD}$ for all $A,B,C,D\in\{1,\ldots,N\}$. A footnote has been added.
			\item Done.
			\item Done.
			\item Done.
			\item  $E$ and $\mathcal{D}_{u(x)}$ have been added in the suggested equations.
			\item Done.
			\item Equations (3.3.30), (3.3.31) and (3.3.32) have been added at page 15. Moreover, they have been used to add an intermediate step in equation (3.3.35) at page 15.
			\item Done.
			\item We made a more precise statement at page 16. 
			\item Done.
			\item We replaced everywhere the symbol $t_{i}$ with $s_{i}$. Notice also that, to avoid confusion, the $s_{a}^{x}$ in equation (3.3.18) have been replaced by $r_{a}^{x}$.
			\item Done. 
			\item Details about $\langle \langle W|$ have been added at page 21. 
			\item Details and equations (4.4.17), (4.4.18) have been added at page 25. 
			\item Done.
			\item Done.
			\item Fixed, the index is now $A$.
			\item Done, probably the sentence was confusing by the fact that we used (4.4.32) in the second equality and (4.4.37) in the third and not just (4.4.37) in the second equality (as it was mistakenly written). We clarified which equations are used. 
			\item Done.
			\item Done.
			\item $c_{j}=\delta_{\sigma_{j},N}$ has been added at page 29.
			\item Details have been added in Remark 9 at page 30. See the bracket in the first sentence and the reference to equation (4.4.48) just below equation (4.4.63).
			\item Words removed in the sentence just above Remark 12 at page 32. 
			\item Words removed in the sentence just above the paragraph with title "Correlations of SSEP" at page 33.
			\item In the third line of the paragraph with title "Proof of formula (4.5.14)" at page 34, in the second curly bracket there was $|\tau|$ instead of $|\bm{\tau}|$. We fixed it.
			\item Words removed in the sixth line after the title of Section 5 at page 35. 
			\item Done.
			\item Done.
			\item More details added in the paragraph just above equation (5.2.2) at page 37. Bold removed in equation (5.2.7) at page 38.
			\item Brackets added in equation (5.2.11), page 38.
			\item Subscripts fixed in equation (5.3.20) page 41.
			\item We put a consistent notation for the absorption probabilities in Appendix C, pages 44-45. 
			\item We put a consistent notation for the boundary densities in Appendix B, pages 43-44.
			\item Done.
			\item For the sake of clarity we prefer to keep this part like it was.
		\end{enumerate}
\section*{Further corrections}
\begin{enumerate}
\item \textbf{Equation (5.2.10)}: the boundary dual generator of the thermalized process was wrong. Compare with the old (ArXiv) equation (5.2.10) with the new equation (5.2.10).
\item \textbf{Equations (5.1.5), (5.1.6)}: the absorption probabilities of the dual particles should be written for $s_{1},\ldots,s_{N-1}$ and not until $s_{N}$ as they were before. Indeed, $s_{a}$ denotes the number of particles of species $a$ that we absorbed at $0$ and we can only absorb particles, not the holes. 
\item \textbf{Section 5.1}: we can talk about $m$-point correlations only if we choose $1\leq x_{1}<\ldots<x_{m}\leq L$ otherwise  we obtain the mixed moments in the non-equilibrium steady state. We have fixed equation (5.1.3) and above and we have added Remark 13 for  clarification.
\end{enumerate}
	\end{document}
